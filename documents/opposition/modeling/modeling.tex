\section{Modeling}
\label{sec:modeling}
	Three different models are described and evaluated in the thesis:
	\begin{itemize}
		\item The Physical model,
		\item The Simple model,
		\item The Virtual model.
	\end{itemize}
	
	The naming of the model is somewhat confusing - why, for example, can't
	the virtual model be a simple model? A common font style for the models 
	could increase the readability.
	
	It is nonetheless an interesting addition to the thesis to study different
	variants of models. It would be interesting to for instance continue 
	the efforts to improve the performance of the physical model for comparison
	with the \textit{simple} and \textit{virtual} models, which were the ones that
	received the most focus in the thesis.
	
	One discrepancy in the thesis is the use of rotational velocity of
	the propellers in the modeling, whereas the controller later sees this
	variable as a measure of PWM pulse width. 
	The variable may be used somewhat interchangably in the case of a linear
	relation between the two, although the connection is very unclear in the
	thesis.
	
	There also seems to be a slight misconception as to the interpretation 
	of the roll ($\phi$), pitch ($\theta$) and yaw ($\phi$) angles, as these
	are presented as the rotation around the Earth-fixed base vectors, which is not
	generally true - see for instance Tait-Bryan angles common in avionics.
	
	The inclusion of the virtual model, and its comparison to the simple model,
	is an interesting addition to the thesis.
	
	The model identification performed in the thesis is well described
	and the verification experiments are relevant.
