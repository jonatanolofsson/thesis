\section{Discussion}
\label{sec:discussion}
The following questions are relevant for discussion of the report, following
the formal presentation.

\subsection{Modeling}
\begin{itemize}
	\item The grey-box physical model that was identified in the thesis
			could not properly describe the motions of the vehicle, do you have any ideas as to why?
			
	\item When the simple model was used, the amplitude of the control signals
			are unbalanced between the motors. 
			Is this behaviour ``optimal'' in the MPC optimization, or are
			there e.g. model errors that introduce this behaviour?
			
	\item The matrix of inertia is assumed to be diagonal. Are there symmetry reasons for this,
			or is it just a simplification?
\end{itemize}

\subsection{Filtering}
\begin{itemize}
	\item In the thesis, a Low-Pass filter is described as a way for state estimation. What was it used for,
			and to what benefits?
			
	\item For the estimation and validation data set, \textsc{Matlab} 
			was used to concatenate different datasets. How was the data merged and fitted in the joints?
\end{itemize}

\subsection{Control}
\begin{itemize}
	\item In the thesis, it is mentioned that an MPC without active constraints is
			equivalent to an LQ controller if the weight of the end state is chosen
			as the solution to the Riccati equation. Can you, briefly, explain why this is
			so, and how this can be a general result for all prediction horizons?
			
	\item In the explicit MPC, the prediction horizon is only two steps - that is, $0.04$ seconds.
			How does this affect the performance of the system?
			
	\item In the explicit MPC, a tree search is used to find the active polyhedra. 
			Could the performance of the algorithm be improved by using other methods
			of indexing? Is the tree search a relevant time-factor?
			
	\item The control variable in the MPC is the PWM pulse width, whereas in the
			model identification, the rotational velocity of
			the propellers is used. How are these two related?
\end{itemize}
