\section*{Introduction}
This paper contains the description of the tasks outlined for the
master thesis work performed by the author.

Due to physical size constraints, a Gantt chart and relevant statistics
are separated from this document, but are relevant for the reader.

The numbering used in this document corresponds directly to the WBS\footnote{Work Breakdown Structure} numbering
in the activity list appended separately.

    \subsection*{Comment regarding the timespan}
        The project timespan is limited by the LiU semester VT2012.
        The semester spans from 2012-01-16 to 2012-06-02\footnote{\url{http://www.lith.liu.se/kalender/terminstider/}}
        which corresponds to 20 weeks.
        The suggested plan also takes into account a reasonable amount
        of time for the opposition for preparation, and suggests that a
        more or less final report should be available as of one month
        before the final presentation.

        This limitation in turn imposes a deadline to the project goal -
        the report is set due two weeks after the milestone where autonomous
        landing will be presented.

    \subsection*{Workload}
        As shown in in the appended plot, the workload has been roughly evenly
        distributed over the semester. It should be noted in the plot that
        400 hours - 4 hours a day - have been assigned to ''unspecified writing
        of report''. Since this time is distributed evenly over the entire
        semester in the plot, the exact amount of working time is somewhat incorrect.
        The (unrealistic) even distribution should however assure that time
        is available to the writing of the report throughout the entire project.

    \subsection*{Availability of resources}
        The proposed schedule is quite narrow, and even if I have both tried not
        to underestimate the work required, and left about $10~\%$ of the time
        unplanned, there is still little room for delays because of inavailability
        of hardware. Inavailability of tutoring time is less critical, since
        the shedule is laid in such a way that work always can be made
        simultaneously on several activities. It should thus be possible
        to continue work on other activities while waiting for the availability
        of a tutor.
