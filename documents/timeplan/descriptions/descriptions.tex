\section{Primary Linux Computer}
    The primary Gumstix module will be central to the state estimation
    and control of the LinkQuad. A lot of the work has been done previously
    that can be used on this computer, and a lot of the work is therefore
    focused on assembling existing parts and get it to work together.
    Also, the time for tuning the parameters should not be underestimated.

    \subsection{Kalman Filter}
        \subsubsection{Implementation}
            There exists today an implementation in C++ of a
            non-linear Unscented Kalman Filter that will be used.
            There is some implementation details to get it running on the
            Gumstix module, as well as a few minor theoretical questions
            in the implementation that I would like to adress.

        \subsubsection{Tuning}
            The tuning of filter covariances can be both cumbersome and
            time-consuming. The work is eased with the availability of
            the VICON-lab, where a ground truth can be obtained to
            enable an off-line tuning loop.

    \subsection{Controller}
        A study have to be performed here to decide upon control type,
        control variables and control model.
        \subsubsection{Implementation}
            After the type of controller has been decided upon, the work
            with implementing it is quite easy work.

        \subsubsection{Modelling}
            The controlled system should be modelled in some way.
            In case LQ control is used, a simple state-space model should
            be devised. If a PID controller is used, using an IMC tuning model
            is beneficial and a model should thus be constructed in either case.

        \subsubsection{Parameter Tuning}
            Here the use of a simulation environment is much beneficial.
            It might be of interest to introduce a disturbance model in the
            simulator to be able to test more realistic flight cases.

    \subsection{Communication}
        \subsubsection{Internal Communication}
            The internal communication on the Gumstix module will be
            made using a simple signaling framework.
            There should not be much work here except compiling it on the
            Gumstix.

        \subsubsection{Serial MCU Communication}
            The serial communication on the Gumstix should be implemented
            already through the existing LinkBoard communication library.

        \subsubsection{Ground Ctation Communication}
            Communication with the ground station should be implemented
            already through the existing LinkBoard communication library.

        \subsubsection{Gumstix Communication}
            Communication with the secondary Gumstix should be implemented
            already through the existing LinkBoard communication library.

    \subsection{Logic}
        \subsubsection{State Machine Implementation}
            The behaviour of an UAV can generally be described as actions
            performed in one of several states. A general state machine
            should therefore be constructed with easily defined actions
            and transition rules. This will be beneficial as one can
            focus on implenting each given action independently and
            let the rules define when each action should be selected.

        \subsubsection{Hover Mode}
            When the LinkQuad is in hover mode, is only control goal
            is to stand still in the air. This is the first mode to be
            implemented, and acts as a test case for further development
            of more advanced flight modes.

        \subsubsection{Milestone: Stable Hovering}
            Here, stable hovering with the LinkQuad should be demonstrated.
            Optionally, reference hovering positions may be changed to
            demonstrate movement.

        \subsubsection{Landing mode}
            \paragraph{State Identification \newline}
                When performing a full landing, there are several substeps
                that need to be identified.

            \paragraph{State Transition Rules \newline}
                Each substep of the landing have to be defined with its
                own set of transition rules that define which control
                state should be activated.
                This work also includes the detection of a completed landing.

            \paragraph{Reference Generation \newline}
                A lot of the work after implementing reference following
                for the LinkQuad will be spent on creating viable route
                references that should be followed.

            \paragraph{Milestone: Autonomous landing \newline}
                Here, an autonomous landing should be demonstrated.

            \paragraph{Milestone: Repeated Autonomous Landing \newline}
                A short while after the previous milestone, repeated landing
                should be performed where time has been available to fix
                issues that arised during the first demonstration.
                This should allow for repeated successful landings of the LinkQuad.

    \subsection{Gumstix Integration}
        Since the Gumstix is a full-fledged Linux computer,
        compiling existing code is not expected to be a big issue.

\section{Secondary Linux Computer}
    The second Gumstix will be running the video processing algorithms
    and communicate results to the primary Gumstix.

    \subsection{PTAM Integration}
        PTAM is a library for monocular SLAM developed at the University of Oxford.
        Is is in a very experimental state, but is what was used in e.g.
        \citep{klein07parallel,weiss11monocular}.
        Since getting it to run seems non-trivial, a lot of time has been assigned
        to getting it integrated.

    \subsection{Gumstix Integration}
        Since PTAM may be severely platform-dependent, the implementation
        will be done directly on the Gumstix.

\section{Sensory MCU}
    \subsection{Sensor Interfacing}
        This should be mostly implemented already with code from IDA

    \subsection{Milestone: Plotted Sensordata}
        This milestone is a confirmation that the sensor interfacing is working.
        Live sensordata should be plotted - filterer or unfiltered.

\section{Control MCU}
    \subsection{Control Servos}
        This should be mostly implemented already with code from IDA
    \subsection{Milestone: Moving Servos}
        This milestone should display the availability of control over servos from the Gumstix
        computer.

\section{Ground Station}
    \subsection{Interfacing the LinkQuad}
        Some kind of interface is needed with the LinkQuad to be able to communicate
        commands to the logic on the primary Gumstix module.
        This may be fairly easy to implement, thanks to IDA code, but should
        nonetheless be well thought through.

\section{Peripheral}
    \subsection{Simulation Environment Integration}
        The LinkQuad has recently been extended with a ROS based simulator.
        It would be of much value to interface the new developments with
        this simulator for control parameter tuning etc. This may be non-trivial however
        and has, as it is deemed an important task, been assigned quite some time.

\section{Report}
    \subsection{Unspecified Work}
        Half the work on the thesis have been assigned to this ''unspecified work'' on the
        report. Hopefully, this will give enough time for a good report and not
        least, time is assigned to this task from the very beginning of the project.

    \subsection{Environment Setup}
        Starting a report of this size requires some setup-time
        where a formal LaTeX directory structure, figure generation tools
        and document generation scripts are setup to work together.

    \subsection{Milestone: Chapter Outline}
        On this milestone, a preliminary outline of the report's table of
        contents should be presented.

    \subsection{Milestone: Filtering and Control Chapters}
        Ready for critique.

    \subsection{Milestone: Video Processing Chapters}
        Ready for critique.

    \subsection{Milestone: Hover State Evaluation Chapter}
        Ready for critique.

    \subsection{Milestone: All Chapters Ready for Critique}
        The aim is that the report will be largely finished by the
        time of this milestone. Some fine-tuning will remain, but the
        report should be finished to be critiqued.

\section{Administration}
    Some time in the thesis project should be devoted to administration

\section{Opposition}
    This activity contains the preparation time for opposing a fellow student's thesis.

\section{Presentation}
    This activity is dedicated to the preparation of the final presentation of the thesis work.
