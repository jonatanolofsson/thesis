\documentclass{beamer}
\usetheme{Marburg}%Madrid,Boadilla,default,Warsaw,Bergen,Frankfurt,Darmstadt
%~ \setbeamertemplate{footline}[page number]
%~ \setbeamercovered{transparent}
\setbeamercovered{invisible}
% To remove the navigation symbols from the bottom of slides
\setbeamertemplate{navigation symbols}{}
%\logo{\includegraphics[height=0.6cm]{yourlogo.eps}}

%% Specify report metadata
\title[A Half-time Report]{Autonomous UAV Landing Using Monocular Vision}
\date{\today}
\newcommand{\lang}{swedish}
\newcommand{\pdfauthor}{Jonatan Olofsson}
\newcommand{\pdftitle}{Autonomous UAV Landing Using Monocular Vision}
\newcommand{\pdfsubject}{}
\newcommand{\pdfkeywords}{}



%% Include helper files
\usepackage[usenames]{color}
\usepackage{amsfonts}
%\usepackage{amsmath}
%~ \usepackage[fleqn]{amsmath}
\usepackage{amsmath}
\usepackage{amssymb}
\usepackage{graphicx}
\usepackage{float,multicol,epsfig}
\usepackage{verbatim}
\usepackage{rotating}
\usepackage{newclude}
\usepackage{include/mcode}
\usepackage{tabularx}
\usepackage{colortbl}
\usepackage{xifthen}
\usepackage{natbib}
\usepackage{tikz}
\usepackage{framed}
\usepackage{xparse}
\usepackage{url}

\input{include/pdf.inc}
\newcommand{\currentchapter}{}
\newcommand{\fig}[4]{%
  \begin{figure}[H]
    \centering
    \includegraphics[width=#1\linewidth]{\currentchapter/figures/#2}
    \caption{#3}
    \label{#4}
  \end{figure}
}

%% Math functions
\renewcommand{\sin}[1]{\text{sin}\left(#1\right)}
\renewcommand{\cos}[1]{\text{cos}\left(#1\right)}
\renewcommand{\arctan}[1]{\text{arctan}\left(#1\right)}
\renewcommand{\arccos}[1]{\text{arccos}\left(#1\right)}

%% Initialize new chapter
\newcommand{\newchapter}[1]{
    \renewcommand{\currentchapter}{#1}
    \include*{#1/#1}
    \clearpage
}

\newcommand{\matlab}{\textsc{Matlab}\xspace}

\newtheorem{algorithm}{Algorithm}


\def\ifmonospace{\ifdim\fontdimen3\font=0pt }
\def\C++{%
\ifmonospace%
    C++%
\else%
    C\kern-.1667em\raise.30ex\hbox{\smaller{++}}%
\fi%
\spacefactor1000 }

%c C sharp
\def\Csharp{%
\ifmonospace%
    C\#%
\else%
    C\kern-.1667em\raise.30ex\hbox{\smaller{\#}}%
\fi%
\spacefactor1000 }



\def \crap {\textit{CRAP}\xspace}

%~ \setlength{\parindent}{0pt}
%~ \setlength{\itemsep}{1pt}
%~ \setlength{\parskip}{0pt}
%~ \setlength{\parsep}{0pt}
%~ \oddsidemargin 0.0pt
%~ \evensidemargin 0.0pt
\newcommand{\tablewidth}{0.9\linewidth}
\newcommand{\plotwidth}{0.5}
\newcommand{\splotwidth}{0.4}
\newcommand{\imagewidth}{0.8}

\definecolor{dkgreen}{rgb}{0,0.6,0}
\definecolor{gray}{rgb}{0.5,0.5,0.5}
\definecolor{mauve}{rgb}{0.58,0,0.82}

\lstset{ %
  language=C++,                % the language of the code
  basicstyle=\scriptsize,           % the size of the fonts that are used for the code
  numbers=left,                   % where to put the line-numbers
  numberstyle=\tiny\color{gray},  % the style that is used for the line-numbers
  stepnumber=1,                   % the step between two line-numbers. If it's 1, each line
                                  % will be numbered
  numbersep=10pt,                  % how far the line-numbers are from the code
  backgroundcolor=\color{white},      % choose the background color. You must add \usepackage{color}
  showspaces=false,               % show spaces adding particular underscores
  showstringspaces=false,         % underline spaces within strings
  showtabs=false,                 % show tabs within strings adding particular underscores
  frame=single,                   % adds a frame around the code
  rulecolor=\color{black},        % if not set, the frame-color may be changed on line-breaks within not-black text (e.g. commens (green here))
  tabsize=2,                      % sets default tabsize to 2 spaces
  captionpos=b,                   % sets the caption-position to bottom
  breaklines=true,                % sets automatic line breaking
  breakatwhitespace=false,        % sets if automatic breaks should only happen at whitespace
  title=\lstname,                   % show the filename of files included with \lstinputlisting;
                                  % also try caption instead of title
  keywordstyle=\color{blue},          % keyword style
  commentstyle=\color{dkgreen},       % comment style
  stringstyle=\color{mauve},         % string literal style
  escapeinside={\%*}{*)},            % if you want to add a comment within your code
  morekeywords={*,...}               % if you want to add more keywords to the set
}


\begin{document}

\begin{frame}
%\frametitle{}

\Huge Autonomous UAV Landing Using Monocular Vision

\LARGE A Half-time Report
\newline
\newline
Jonatan Olofsson

\end{frame}

\begin{frame}
    \frametitle{What should be done?}
    \begin{itemize}
        \item Develop a platform for real-time control of a UAV,
        \item Integrate the PTAM library as a component in the state estimation,
        \item Implement a simple landing procedure using the pose estimate.
    \end{itemize}

    \fig{LinkQuad}{0.3}
\end{frame}


\begin{frame}
    \frametitle{What has been accomplished so far?}
    \begin{itemize}
        \item Internal/serial communication,
        \item Visualization and evaluation tools,
        \item Physical modelling,
        \item Pose/state estimation (some problems still),
        \item Control,
        \item PTAM.
    \end{itemize}
\end{frame}


\begin{frame}
    \frametitle{CRAP}
    \begin{itemize}
        \item A framework for internal communication.
        \begin{itemize}
            \item Minimal overhead using C++ function pointers and minimizing copies
        \end{itemize}
        \item Serial communication library;
        \begin{itemize}
            \item Some connections to previoius implementation,
            \item C++ templates moves calculations to compile-time,
            \item Simplistic API.
        \end{itemize}
        \item Integrated Matlab-style real-time plotting (enabled compile-time),
        \item Easily configurable, modular design.
    \end{itemize}
\end{frame}

\begin{frame}
    \frametitle{Physical Modelling}
    Detailed physical model of forces and moments acting on the quadrotor.
    \begin{itemize}
        \item Gravity,
        \item Wind
        \begin{itemize}
            \item Stochastic model - estimated by the filter,
            \item Models forces and moments on the quadrotor.
        \end{itemize}
        \item Thrust;
        \begin{itemize}
            \item Propeller flapping,
            \item Ground Effect.
        \end{itemize}
        \item Additional forces,
        \begin{itemize}
            \item Gyroscopic effects, hub forces etc.
        \end{itemize}
    \end{itemize}
\end{frame}


\begin{frame}
    \frametitle{State Estimation}
    Estimates the full set of 23 states.
    \begin{itemize}
        \item UKF,
        \item EKF.
    \end{itemize}
\end{frame}


\begin{frame}
    \frametitle{Control}
    Basic idea: Linearize the physical model and use linear control theory.
    \tiny
    \begin{equation}
        \dot{x} = f(x,u) \approx f(x_{0},u_{0})
            + \underbrace{\left. \frac{\partial f}{\partial x} \right|_{
                \begin{array}{l}
                    x=x_{0} \\
                    u=u_{0}
                \end{array}
            }}_{A}
                \underbrace{\left( x-x_{0} \right)}_{\Delta x}
            + \underbrace{\left. \frac{\partial f}{\partial u} \right|_{
                \begin{array}{l}
                    x=x_{0} \\
                    u=u_{0}
                \end{array}
            }}_{B}
                \underbrace{\left( u-u_{0} \right)}_{\Delta u}
    \end{equation}
    \begin{equation}
    \label{eq:controller:affinelq}
        \dot{X} = \left[
        \begin{array}{c}
            \dot{x} \\
            0
        \end{array}\right] =
        \underbrace{\left[
        \begin{array}{cc}
            A & f(x_{0},u_{0})-Ax_{0} \\
            0 & 0
        \end{array}\right]}_{\bar{A}}
        \underbrace{\left[
        \begin{array}{c}
            x \\
            1
        \end{array}\right]}_{\bar{X}}
        +
        \underbrace{\left[
        \begin{array}{c}
            B \\
            0
        \end{array}\right]}_{\bar{B}}
        \Delta u.
    \end{equation}
    \begin{itemize}
        \item SDRE
    \end{itemize}
\end{frame}

\begin{frame}
    \frametitle{PTAM}
    \begin{itemize}
        \item Parallell Tracking And Mapping,
        \item Developed at the University of Oxford,
        \item \url{http://www.youtube.com/watch?v=Y9HMn6bd-v8&feature=player_embedded},
        \item Compiled for the Gumstix but not yet tested,
        \item Camera settings needs adjustment for good tracking.
    \end{itemize}
\end{frame}
%~ %~
\begin{frame}
    \frametitle{Personal Goals}
    \begin{itemize}
        \item Create a general lightweight platform for robot automation.
        \item Learn about flight dynamics.
        \item Use quaternions.
        \item Evaluate the UKF and the EKF non-linear filters.
        \item Use and evaluate SDRE for non-linear control.
        \item Take the time to write really good and re-usable code.
    \end{itemize}
\end{frame}

\begin{frame}
    \frametitle{Time Schedule}
    \begin{itemize}
        \item[-] Approx. two weeks behind schedule,
        \item[-] Expected hover by now,
        \item[+] About two weeks of buffer left,
        \item[+] Ketchup effect expected as soon as the observer is working.
    \end{itemize}
\end{frame}

\begin{frame}
    \frametitle{Next Steps}
    \begin{itemize}
        \item Get the \#!\#?! observer working,
        \item Compile for gumstix and evaluate speed,
        \item Update the report;
        \begin{itemize}
            \item Results from the evaluation of the UKF,
            \item Further describe the model,
            \item Chapter about PTAM and video tracking.
        \end{itemize}
        \item Verify the model against flight data and put in real parameter values,
        \item Perform test flight.
    \end{itemize}
\end{frame}
\end{document}
