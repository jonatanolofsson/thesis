\section{State Estimation}
    \subsection{Sensor Fusion}
    \begin{frame}{Sensor Fusion}
        Uses information from all sensors with models of expected behavior
        to estimate the movements of the vehicle.

        The Kalman filter is the standard choice for high-level state estimation.

        \textbf{Non-linear extensions:}
        \begin{itemize}
            \item EKF - Extended Kalman Filter
            \item UKF - Unscented Kalman Filter
        \end{itemize}
        The UKF has issues with the applied motion model in high uncertainty simulations.
        The EKF was selected as the more reliable filter.
    \end{frame}
    \note{
        The measurements from the camera need to be fused with the rest
        of the available sensor data, as well as the motion model.
        In the linear case, the Kalman filter is provably the optimal bayesian filter
        given some assumptions, but the theory can also with good results be extended
        to the nonlinear case.
        The novel approach is the Extended Kalman filter which simply
        linearizes the model as nescessary and applies the linear theory.
        The Unscented Kalman Filter approaches the field of particle filtering
        by testing variations of the estimated state and look for the best Gaussian fit.

        Both of these are studied in the report, although in the end, the
        Extended Kalman filter proved to be the more reliable of the two.
    }

    \subsection{Filtering Framework}
    \begin{frame}{Filtering Framework}
        The standard formulation of a high-level (Bayesian) state estimation framework
        relies on two separate steps.

        \vspace{0.5cm}

        \textbf{Measurement update: } Sample the sensors and weigh their likelihood against the current estimate.

        \textbf{Time update: } Use the motion model to predict the vehicle's motions.


        \vspace{0.5cm}
        Through discrete instances of time, the steps are independent.
        %~ We may e.g. perform one m.u. for the IMU and one for the camera before
        %~ performing a time update.
    \end{frame}
    \note{
        A bayesian filtering framework is generally equipped with two independent steps;

        The measurement update incorporate sensorvalues to improve the
        state estimate. Since both the state estimate and the sensor measurements
        are associated with noise and uncertainties, these two have to be
        weighed against each other.

        The time update uses a model to anticipate how the states progresses
        through time, based on reasonable assumptions.
        One can here, for instance, assume that the velocity is constant, and
        adjust the vehicle's position according to this velocity.
        In this thesis, a physical motion model was developed and presented.
    }

    \subsection{Physical Motion-model}
    \begin{frame}{Motion Model}
        A nonlinear physical model of the quadrotor
        is studied in the thesis.
        \begin{itemize}
            \item Detailed physical model.
            \item Same model is used for state estimation, control and simulation.
            \item May easily be replaced by simpler model, e.g. for state estimation.
        \end{itemize}

        The model also includes sensor-models for all used sensors.
        %~ \begin{itemize}
            %~ \item Used both for simulation and state estimation.
            %~ \item New sensor types are easily added.
        %~ \end{itemize}
    \end{frame}
    \note{
        The physical quadrotor model of the thesis is in fact
        developed in some detail, and while some model identification
        remain, it describes in detail the forces acting on the quadrotor
        in flight.

        The physical model is arguably unnescessary complex for many purposes,
        but the fact is that this model actually fits both simulation,
        state estimation and nonlinear control, which is a desirable,
        but seldom exploited, feature.

        In fact, the structure of the model allows for its re-use in all
        model-dependent parts of the implementation, removing the need for
        multiple models to be developed.
    }
    % EKF vs. UKF
    % Kalman filter framework with time- and measurement updates

%~ \begin{frame}
    %~ \frametitle{State Estimation}
    %~ Estimates the full set of 23 states.
    %~ \begin{itemize}
        %~ \item UKF,
        %~ \item EKF.
    %~ \end{itemize}
%~ \end{frame}
%~
%~ \begin{frame}
    %~ \frametitle{Physical Modelling}
    %~ Detailed physical model of forces and moments acting on the quadrotor.
    %~ \begin{itemize}
        %~ \item Gravity,
        %~ \item Wind
        %~ \begin{itemize}
            %~ \item Stochastic model - estimated by the filter,
            %~ \item Models forces and moments on the quadrotor.
        %~ \end{itemize}
        %~ \item Thrust;
        %~ \begin{itemize}
            %~ \item Propeller flapping,
            %~ \item Ground Effect.
        %~ \end{itemize}
        %~ \item Additional forces,
        %~ \begin{itemize}
            %~ \item Gyroscopic effects, hub forces etc.
        %~ \end{itemize}
    %~ \end{itemize}
%~ \end{frame}
