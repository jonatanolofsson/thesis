\section{Non-linear Control}
\subsection{LQ control}
\begin{frame}{Linear Quadratic control}
    Linear Quadratic Control utilizes a linear motion model to optimally
    control the system.

    \textbf{Definition: } Find $u=-Lx$ s.t.
    \scriptsize
    \begin{equation*}
        \mathcal{J} = \int_{0}^{\infty} x^{T}Qx + u^{T}Ru dt.
    \end{equation*}
    \normalsize
    is minimized. A feedback-form closed solution exists, having solved the CARE\footnote{Continous Algebraic Riccati Equation};
    \scriptsize
    \begin{equation*}
        A^{T}S + SA + M^{T}QM - SBR^{-1}B^{T}S = 0
    \end{equation*}
    \begin{equation*}
        L = R^{-1}B^{T}S
    \end{equation*}
\end{frame}
\begin{frame}{Linear Quadratic control}
    \begin{itemize}
        \item Appealingly simple.
        \item Requires a linear motion model
    \end{itemize}

    \vspace{0.5cm}

    The motions of a quadrotor is non-linear, but can locally
    be described by a linear approximation.

    \begin{itemize}
        \item The motion model needs linearization.
        \item The linearization needs a linearization point.
    \end{itemize}
\end{frame}

\subsection{SDRE}
\begin{frame}{State-Dependent Riccati Equation}
    \textbf{Basic idea: } Linearize the physical model and use LQ theory.
    \tiny
    \begin{equation*}
        \dot{x} = f(x,u) \approx f(x_{0},u_{0})
            + \underbrace{\left. \frac{\partial f}{\partial x} \right|_{
                \begin{array}{l}
                    x=x_{0} \\
                    u=u_{0}
                \end{array}
            }}_{A}
                \underbrace{\left( x-x_{0} \right)}_{\Delta x}
            + \underbrace{\left. \frac{\partial f}{\partial u} \right|_{
                \begin{array}{l}
                    x=x_{0} \\
                    u=u_{0}
                \end{array}
            }}_{B}
                \underbrace{\left( u-u_{0} \right)}_{\Delta u}
    \end{equation*}
%~ \end{frame}
%~ \begin{frame}{State-Dependent Riccati Equation}
    \normalsize
    By adding a homogeneous state, the linear property of the equation is regained.
    The result is locally valid in every differentiable point in the state space.
    \tiny
    \begin{equation*}
        \dot{X} = \left[
        \begin{array}{c}
            \dot{x} \\
            0
        \end{array}\right] =
        \underbrace{\left[
        \begin{array}{cc}
            A & f(x_{0},u_{0})-Ax_{0} \\
            0 & 0
        \end{array}\right]}_{\bar{A}}
        \underbrace{\left[
        \begin{array}{c}
            x \\
            1
        \end{array}\right]}_{\bar{X}}
        +
        \underbrace{\left[
        \begin{array}{c}
            B \\
            0
        \end{array}\right]}_{\bar{B}}
        \Delta u.
    \end{equation*}
\end{frame}
