\section{Nonlinear Control}
    \subsection{LQ control}
    \begin{frame}{Linear Quadratic control}
        \begin{itemize}
            \item Optimal control of a linear motion model.
            \item Control signal minimizes quadratic criterion, weighing control error against agressive control.
        \end{itemize}
        \begin{align*}
            \dot{x} &= Ax + Bu \\
            \min_{u=-Lx}\mathcal{J} &= \int_{0}^{\infty} x^{T}Qx + u^{T}Ru\, dt
        \end{align*}
        Feedback solution, given the solution of the \textbf{Riccati Equation}.
        \begin{equation*}
            A^{T}S + SA + M^{T}QM - SBR^{-1}B^{T}S = 0
        \end{equation*}
    \end{frame}
    \note{
        Having aquired a pose estimate, this is fed to a controller
        that is able to control the flight and descent of the quadrotor.
        Linear Quadratic Control is one of the simplest to use optimal control techniques,
        and makes use of the motion model I just briefly presented.

        The basic principle of Linear Quadratic Control is to weigh the
        system response, as predicted by the model, against the cost of
        using the control signals, u.
        That is; using this mathematical criterion, the LQ decides which
        is better - to rapidly reach the desired velocity, or
        to be gentle on the motors.

        An added advantage for the Linear Quadratic Control is that
        a feedback solution exists, having solved the Riccati equation exemplified here.
    }

    \begin{frame}{Linear Quadratic control}
        \begin{itemize}
            \item Appealingly simple, when you know it.
            \item Requires a linear motion model, $A$.
        \end{itemize}

        \vspace{0.3cm}

        The motions of a quadrotor are nonlinear, but can locally
        be described by a linear approximation.

        \vspace{0.3cm}

        \begin{itemize}
            \item The motion model needs linearization.
            \item The linearization needs a linearization point.
        \end{itemize}
    \end{frame}
    \note{
        The Linear Quadratic controller may be appealingly simple when you know it, although,
        as the name suggests, it requires a linear motion model.
        The problem is that, a general model of a quadrotor
        is nothing but nonlinear,
        so it does not quite fit the original LQ formulation.

        One can consider linearizing the model in various stationary points,
        although such points are generally difficult to find unless we apply
        the beforementioned clever trick.
    }

    \subsection{SDRE}
    \begin{frame}{State-Dependent Riccati Equation}
        \textbf{Basic idea: } Linearize the physical model and use LQ theory.
        \tiny
        \begin{equation*}
            \dot{x} = f(x,u) \approx f(x_{0},u_{0})
                + \underbrace{\left. \frac{\partial f}{\partial x} \right|_{
                    \begin{array}{l}
                        x=x_{0} \\
                        u=u_{0}
                    \end{array}
                }}_{A}
                    \underbrace{\left( x-x_{0} \right)}_{\Delta x}
                + \underbrace{\left. \frac{\partial f}{\partial u} \right|_{
                    \begin{array}{l}
                        x=x_{0} \\
                        u=u_{0}
                    \end{array}
                }}_{B}
                    \underbrace{\left( u-u_{0} \right)}_{\Delta u}
        \end{equation*}
        \normalsize
        By adding a homogeneous state, the linear property of the equation is regained.
        The result is locally valid in every differentiable point in the state space.
        \tiny
        \begin{equation*}
            \dot{X} = \left[
            \begin{array}{c}
                \dot{x} \\
                0
            \end{array}\right] =
            \underbrace{\left[
            \begin{array}{cc}
                A & f(x_{0},u_{0})-Ax_{0} \\
                0 & 0
            \end{array}\right]}_{\bar{A}}
            \underbrace{\left[
            \begin{array}{c}
                x \\
                1
            \end{array}\right]}_{\bar{X}}
            +
            \underbrace{\left[
            \begin{array}{c}
                B \\
                0
            \end{array}\right]}_{\bar{B}}
            \Delta u.
        \end{equation*}
    \end{frame}
    \note{
        The trick proposed in this thesis, with good results, is to
        extend the Taylor expansion linearisation with a new virtual state.
        This virtual state, shown in the second equation, allows us to regain the
        linear property of an affine system approximation that is the
        result of a general Taylor expansion, as in the first equation.

        By using this formulation, it is possible to control a general
        nonlinear system with the same simplicity as the linear case.

        This general case can be solved repeatedly to generate the control
        signals, a techinique called State-Dependent Riccati Equation which
        is studied and evaluated in the thesis.
    }
