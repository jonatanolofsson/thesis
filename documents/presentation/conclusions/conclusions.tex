\section{Conslusions}
    \begin{frame}{Conclusions}
        \begin{itemize}
            \item General algorithms and control structure were fully implemented.
            \item PTAM modifications enables full autonomousity.
            \item Simulated advanced control and landing performed in simulation.
        \end{itemize}
        \vspace{-0.5cm}
        \begin{figure}[h]
            \includegraphics[trim = 0mm 35mm 0mm 35mm,clip,width=0.5\textwidth]{\currentchapter/figures/referencefollowing}
        \end{figure}
        \vspace{-0.5cm}
        \begin{itemize}
            \item Implementation covers advanced control.
            \item Tuning of filtering and control remain.
            \item Results suggest the system is viable to perform landing.
        \end{itemize}
    \end{frame}
    \note{
        In the work I present in this thesis covers the implementation and
        simulational evaluation of the algorithms presented here, as well
        as a lot of groundwork to connect these in a high-level control system
        for a quadrotor.
        I have extended the PTAM monoslam localisation library and incorporated
        these measurements into an extendable filtering framework to provide
        the state estimate needed to achieve stable control needed to perform
        precision landing with a quadrotor.

        Results are still simulational, but with continued parameter tuning
        and enough computational power, they firmly suggest that full control
        and landing could be performed using the implementation.
        The work has been focused on implementing state-of-the-art techniques
        to provide landing and general control for the LinkQuad and, by extension
        general quadrotors.
    }
