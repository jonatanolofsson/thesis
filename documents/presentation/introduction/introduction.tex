\section{Introduction}

    \subsection{Background}
        \begin{frame}{Background}
            \begin{itemize}
                \item Increased interest in civilian applications.
                \item For many applications, a small scale vehicle is desired.
                \item MAV - Micro Air Vehicle; UAV weighing $5$~kg or less\footnote{Definitions differ}.
            \end{itemize}
            \fig{0.5}{skeldar}
        \end{frame}
        \note{
            During the past few years, there has been a significant increase
            in the interest in the field of unmanned aerial vehicles.
            While the military industry is the major developer of UAVs,
            the maturity and availability of technology has enabled applications
            in search-and-rescue operations and even hobby-level implementations.

            Many UAVs are of considerable size to carry load and to increase flight-time.
            As new applications develop however, the interest has increased in small scale UAVs
            such as the LinkQuad, towards which this thesis have been targeted.
            This type of UAV is known as MAV, Micro Air Vehicle.
        }

    \subsection{The LinkQuad Platform}
        \begin{frame}{The LinkQuad Platform}
            Quadrotor research platform developed by AIICS at the
            Department of Computer and Information Science of Linköping University.

            Used sensors: Accelerometers, gyroscopes, pressure sensor, camera

            \fig{0.5}{linkquadoutside}
        \end{frame}
        \note{
            The LinkQuad platform is a quadrotor developed here at IDA.
            It is equipped with a standard set of inertial sensors - accelerometers and gyroscopes -
            as well as pressure sensors, magnetometers and GPS.
            It can also be optionally equipped with a camera, which was used in this thesis.
        }
        \begin{frame}{The LinkQuad Platform}
            \begin{itemize}
                \item Dual gumstix micro-computers (Linux).
                \item Sensor-board with dual microcontrollers for sensor sampling, data logging and low-level control.
            \end{itemize}
            \begin{tikzpicture}[auto,>=latex']
            \tikzstyle{mcu} = [rectangle, draw=none, rounded corners=1mm, fill=Emerald,
                text centered, anchor=north, text=white,font=\small]
            \tikzstyle{sensor} = [rectangle, draw=none, fill=black,
                text centered, anchor=north, text=white,font=\tiny, node distance=0.8cm]
            \tikzstyle{camera} = [rectangle, draw=none, rounded corners=1mm, fill=red,
                text centered, anchor=north, text=white,font=\small]
            \tikzstyle{framework} = [rectangle, draw, fill=YellowGreen,
                text centered, anchor=north, text=white,font=\small,thin, minimum width=2cm,node distance=0.7cm]
            \tikzstyle{gumstix} = [rectangle, draw=none, fill=LimeGreen, rounded corners=2mm,
                text centered, anchor=north, text=white,font=\small]
            \tikzstyle{gumstixtext} = [rectangle, draw=none,
                text centered, align=center, anchor=north, text=white,font=\small]

            \node [gumstixtext] (primarytext) {Primary computer \\ \tiny{State-estimation and control}};
            \node [framework, below of=primarytext] (crap) {CRAP};
            \node [gumstixtext, right of=primarytext, node distance=5cm] (secondarytext) {Secondary computer \\ \tiny{Video processing}};
            \node [framework, below of=secondarytext] (ptam) {PTAM};

            \begin{pgfonlayer}{background}
                \node [gumstix, fit=(primarytext) (crap)] (primary) {};
            \end{pgfonlayer}
            \begin{pgfonlayer}{background}
                \node [gumstix, fit=(secondarytext) (ptam)] (secondary) {};
            \end{pgfonlayer}

            \node [mcu, below of=primary, xshift=-1.5cm, node distance=1.5cm] (smcu) {Sensor MCU};
            \node [mcu, right of=smcu, node distance=3cm] (cmcu) {Control MCU};
            \node [sensor, below of=smcu, xshift=-1.2cm] (accelerometer) {Acc.};
            \node [sensor, right of=accelerometer] (gyro) {Gyro};
            \node [sensor, right of=gyro] (magnetometer) {Mag.};
            \node [sensor, right of=magnetometer] (pressure) {hPa};
            \node [camera, below of=secondary, node distance=1.5cm] (camera) {Camera};
            \node [sensor, below of=cmcu] (esc) {Speed controller};
            \node [sensor, right of=esc, node distance=1.8cm] (sdcard) {SD-card};

            \path[-]
                        (primary) edge (secondary)
                        (primary.south) edge (smcu.north)
                        (primary.south) edge (cmcu.north)
                        (cmcu.south) edge (esc.north)
                        (smcu.south) edge (accelerometer.north)
                        (smcu.south) edge (gyro.north)
                        (smcu.south) edge (magnetometer.north)
                        (smcu.south) edge (pressure.north)
                        (secondary) edge (camera)
                        (smcu.east) edge (cmcu.west)
                        (cmcu.south) edge (sdcard.north)
                        ;
            \end{tikzpicture}
        \end{frame}
        \note{
            One particularly notable feature on the LinkQuad is the dual
            gumstix Linux microcomputers. This allows us to dedicate one
            each of these to video processing and state estimation respectively.
        }

    \subsection{Problem Formulation}
        % Why was a high-level filtering and control framework needed?
        \begin{frame}[c]{Problem Formulation}

            \textbf{Primary goal:} Develop a control system
                for the LinkQuad that use video-based positioning
                to provide stable landing.

            \textbf{Breakdown:}
            \begin{itemize}
                \item Video-based SLAM.
                \item Sensor Fusion with available sensors.
                \item Use state-estimate for control.
                \item Generate control reference for landing procedure.
            \end{itemize}
        \end{frame}
        \note{
            One of the most important features of an independent UAV is
            the landing procedure. Therefore, the goal of this thesis was
            to develop a control system that, with the help of camera positioning,
            could be used to land the LinkQuad.
        }

    \subsection{Method}
        \begin{frame}{Method: Video-based Positioning}
            Extract 3D information from 2D video-stream.
            \begin{itemize}
                \item vSLAM - Visual SLAM.
                \item Simultaneous Localisation And Mapping to track features in the video-stream.
                \item Gives orientation and position relative to the features.
                \item High computational demands.
                \item Complex sensor measurements.
                %~ \item Fits nicely into a state-space filtering framework.
            \end{itemize}
        \end{frame}
        \note{
            One of the main technologies that were studied in this thesis
            is video based SLAM, Simultaneous Localization And Mapping using
            video-stream data.
            A vSLAM algorithm - and I say ``a'', 'cause there are several, which
            work in simiilar ways - extracts trackable features from the video images,
            and tracks their changes in position between the video frames.
            This is used to extract the camera's position relative to the
            features.
            Video processing generally puts high computational demands, and
            the resulting measurements are quite complex to interpret.
            A high-level filtering framework needed to be implemented to
            integrate the video positioning with the state estimation of the quadrotor.
        }

        \begin{frame}{Method: Filtering}
            Current implementation is based on
            complementary filtering.
            \scriptsize
            \begin{equation*}
                \text{angle} = (0.98)*(\text{angle} + \text{gyro} * \text{dt}) + (0.02)*(a_{\hat{x}})
            \end{equation*}
            \normalsize
            Performance is adequate, but the c.f is difficult to extend.

            Also, camera measurements
            \begin{itemize}
                \item cannot be used directly in the current filter,
                \item fits nicely into a standard state-space filter framework.
            \end{itemize}

            A high-level filtering framework with an advanced motion model
            was implemented and applied for state-estimation.
        \end{frame}
        \note{
            The current observer - that is, implementation of state estimation - is based on the
            complementary filter, and while the performance of the filter
            is quite adequate for its current purpose, adding more sensors - especially
            such as the camera positioning - is quite difficult.

            The de facto standard techique for nonlinear state estimation,
            the Extended Kalman Filter, was ultimately applied for the state estimation.
            Extensive work was in fact put into creating a physical motion model,
            and while a simpler model probably would have sufficed for the state estimation,
            the same model could in fact thereby be re-used later, for instance in the nonlinear
            control that was implemented.
        }

        \begin{frame}{Method: Control}
            With a motion model available, control signals can be computed optimally.
            Linear Quadratic control offers
            \begin{itemize}
                \item simple implementation,
                \item simple tuning,
                \item optimal control.
            \end{itemize}
            Motion model needs to be linear, which is not the case for a quadrotor.
            This constraint can be circumvented by an extension to LQ control using the
            \textbf{State-dependent Riccati Equation}.
        \end{frame}
        \note{
            Having a motion model ment that the LQ control methodology was near
            at hand to be applied. The problem is, the LQ requires the motion model to be linear.
            In the thesis, I study a technique to bypass this constraint
            called State-dependent Riccati Equations, which basically solves the
            linear problem repeatedly online.
        }

    \tikzstyle{state} = [rectangle, draw=none, rounded corners=1mm, fill=YellowGreen,
                    text centered, anchor=north, text=white, minimum width=2cm, minimum height=1cm, node distance=2cm, align=center]
    \tikzstyle{condition} = [diamond, draw=none, fill=red!80,
                    text centered, anchor=north, text=white, minimum height=1cm, node distance=2.3cm, align=center]
    \tikzstyle{choice}=[font=\scriptsize]

        \begin{frame}{Method: Reference Generation}
            With properly implemented control, landing is a matter of descending
            steadily until landing is detected.
            \begin{figure}[h]
            \footnotesize\centering
                \begin{tikzpicture}[auto]
                    \node (entry) {Entry};
                    \node [state, right of=entry] (halt) {Halt to hover\\ {\tiny Set $V_{ref}=0$}};
                    \node [condition,right of=halt] (halted) {$|V| < \epsilon$};
                    \node [state, below of=halt, node distance=1.8cm] (descend) {Descend \\ {\tiny Set $V_{ref,z} > 0$}};
                    \node [condition,right of=descend] (landed) {Landed};
                    \node [state, below of=descend, node distance=1.8cm] (spindown) {Spin down};
                    \path[->]
                        (entry.east) edge (halt.west)
                        (halt.east) edge (halted.west)
                        (halted) edge[loop right,in=90,out=-135] (descend.north)
                        (descend.east) edge (landed.west)
                        (landed) edge[loop right,in=90,out=-135] (spindown.north)
                    ;
                \end{tikzpicture}
            \end{figure}
        \end{frame}
        \note{
            Finally, I also implemented a state-machine engine to generate
            reference signals for the controlled descent and landing of a quadrotor.
            The basic idea is to use the positioning for a stabilized descent
            and to detect the event of landing when the vehicle has hit the ground.
        }

        \begin{frame}{Method: Landing Detection}
            Landing is generally associated with a lack of descent.
            The observer will explain this with upward winds.

            \vspace{0.7cm}

            \textbf{Two interesting states to monitor:}
            \begin{itemize}
                \item Altitudinal velocity
                \item Altitudinal wind velocity
            \end{itemize}

            \vspace{0.7cm}

            %~ Beause only simulated landing was performed in this thesis,
            %~ the implementation needs further work.
        \end{frame}
        \note{
            As the quadrotor lands, the observer's current motion model will fail
            to accurately describe this sudden loss of altitudinal velocity and will
            instead, after a while's settling time, explain the behaviour
            of the vehicle by an increased upward force from the modeled wind.
            Thus, the altitudinal velocity and estimated altitudinal wind velocity
            could both be monitored for increased detection reliability.
        }

%~ \begin{frame}{The LinkQuad Platform}
    %~ \fig{LinkQuad}{0.3}
%~ \end{frame}
