\section{Previous work}
\label{sec:previouswork}
    Here, previous known similar work is referenced and discussed.
    Previous work can be used for both reference implemetations, but also to
    recognize limitations and restrictions posed on the systems studied.

    It turns out that the problem of navigating and landing an UAV using visual
    feedback has been, albeit sparsely,solved.
    Even fewer have however succeeded using strictly on-board sensors and
    real-time algorithms running on the limited processing power that standard
    UAV's of the size of the LinkQuad generally are equipped with.
    The LinkQuad may, with its distributed processing
    power, be well suited for a full implementation and solution to the problem.

    \subsection{Autonomous landing}
    The problem of landing a quadrocopter can to a large extent be boiled
    down to achieving good pose estimates using available information.
    This problem is studied e.g. in \citep{DM:MS:10,brockers:803111},
    but the most interesting results are obtained in \citep{DBLP:conf/icra/BloschWSS10,weiss11monocular},
    where the ideas from \citep{klein07parallel} are implemented on a UAV platform
    and excellent results are obtained.

    \citep{brockers:803111} implements a landing control reference scheme
    which is summarized in \ref{ssec:waypointgeneration}.

    \subsection{vSLAM}
    A first take on a vSLAM algorithm is presented in \citep{Karlsson05thevslam}, resting on the
    foundation of a Rao-Blackwellized particle filter with Kalman filter banks.
    It might possibly be of gain to use the theory studied in \citep{vandermerwe:upf},
    which extends the work in \citep{Montemerlo+al:AAAI02,Montemerlo_2003_4434} to use a prior distribution
    which can be tuned for heavier tailed distributions than the standard gaussian\citep{Merwe04sigma-pointkalman}, since
    it is explicitly noted in \citep{Karlsson05thevslam} that the measurement errors seem to
    be drawn from such a distribution.

    The algorithm presented in \citep{Karlsson05thevslam} also extends to
    the case of multiple cameras, which could be interesting in a longer perspective.
    An implementation has been made in a ROS project\citep{rosvslam}, which may be used for
    reference.

    Another source which could prove interesting in the area of SLAM is \citep{Davison:2007:MRS:1263144.1263479}, where
    the MonoSLAM algorithm is introduced and real-time code samples are referenced.

    \citep{klein07parallel} uses an alternative approach and splits the tracking problem of SLAM from the mapping problem.
    This means that the mapping can run more advanced algorithms at a slower
    pace than required by the tracking.

    The algorithm proposed in \citep{klein07parallel} uses selected keyframes
    from which offsets are calculated continously in the tracking thread, while
    the mapping problem is addressed separately as fast as possible using information only
    from these keyframes. This opposed to the traditional vSLAM filtering
    solution where each frame has to be used for continuous filter updates.

    By for instance not considering uncertainties in either camera pose or feature location,
    the complexity of the algorithm is reduced and the number of studied points
    can be increased to achieve better robustness and performance than
    when a filtering solution is used \citep{DBLP:conf/icra/StrasdatMD10}.

    The mapping implementation in \citep{DBLP:conf/icra/StrasdatMD10} uses the SBA libary \citep{sba,lour09}
    to compute the Bundle Adjustment step in the mapping algorithm.
    Again, this is the method on which \citep{weiss11monocular} bases its implementation.


    \subsection{Filtering}
    Using the work presented e.g. in \citep{Julier95anewapproach,Julier97anew,vandermerwe:upf}
    along with the implementation in \citep{bayesclasses}, a fairly
    efficient implemetation of the filter exists today from previous projects by
    the author of this document.
    The same code could also, with little effort, be used in the vSLAM algorithm.

    In a 3D-environment, it is desirable to select an appropriate representation
    of the orientation, e.g. the quaternion representation.
    To cope with the restrictions in such a representation, \citep{DBLP:journals/ras/Ude99}
    could be studied, for instance.
