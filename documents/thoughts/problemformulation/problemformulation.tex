\section{Problem formulation}
\label{sec:problemformulation}
    In this section, the thesis subject is introduced and the problem
    formulation is presented. After the main goal of the thesis is presented, 
    the section continues with a proposed means of implementation.

    Selected for having a large potential and expected payoff in terms of
    both performance and from an educational standpoint, it is proposed
    that a vSLAM algorithm is implemented for estimation of absolute
    generalized pose (position and orientation).
    This approach have a long-term advantage over the optical flow
    algorithm previously discussed for the LinkQuad system, and may also possibly benefit from
    previous work done by on the LinkQuad by \citep{Barac11}.
    
    \subsection{Main goal}
        The main goal of the thesis work is to develop and implement
        an autonomous landing mode for the LinkQuad quadrocopter.
        The following sensors are available onboard as of today:
        \begin{itemize}
            \item Accelerometers (3DOF acceleration),
            \item Gyroscopes (3DOF angular velocity),
            \item Pressure sensor (height above ground, HOG),
            %~ \item Magnetometer (3DOF angular orientation)
            \item Camera.
        \end{itemize}

        Sensors of optical flow have been discussed in the forming of the
        thesis but since optical flow can be computed from the camera feed,
        this source will be investigated foremost in the event that optical flow
        i used in the state estimation. Other sensors have been discussed
        as well and may be added if possible and if needed.

        The problem of landing autonomously can be divided into three main
        subproblems that all need to be solved for full online autonomous landing:
        \begin{enumerate}
            \item State estimation,
            \item Landing site designation / waypoint generation,
            \item Control.
        \end{enumerate}

        With access to the VICON-lab at IDA, it is possible to perform
        work on the latter two subproblems without the need for fully
        functional state estimation, thus allowing progress on multiple
        frontiers at once.

        The subsystems should also be made pluggable, so that
        for instance the landing site designation could initially be
        implemented in a simple manner, later
        to be replaced by a more advanced algorithm for finding a suitable
        landing site.

    \subsection{State estimation}
        The problem of estimating the full state of the controlled LinkQuad
        will be tackled by implementing an unscented Kalman filter 
        (\citep{Julier95anewapproach,Julier97anew,vandermerwe:upf,Merwe04sigma-pointkalman}) 
        and fusing all available sensor data.
        The first three of the above mentioned onboard sensors are
        straightforward to implement, while the camera will require more
        work and will play a significant role in the thesis.
        
        As the algorithm for the vSLAM, it is proposed that a keyframe solution
        similar to that of \citep{weiss11monocular} is pursued, as it seems to yield
        good results to a tractable computational cost.

    \subsection{Landing site designation / waypoint generation}
        \label{ssec:waypointgeneration}
        The problem of finding a suitable landing site for the LinkQuad
        can, with a control logic suitably interfaced, easily be scaled.
        Examples of designation schemes are
        \begin{itemize}
            \item Descend-where-you-are,
            \item Visual marker,
            \item Virtual marker,
            \item Surface finding within specified area.
        \end{itemize}

        For the landing, following the example of for instance \citep{brockers:803111,DM:MS:10}, it
        may be suitable to implement a strategy as follows:
        \begin{enumerate}
            \item Landing site detection
            \item Estimate refinement
            \item Approach
            \item Descend
            \item (Recovery)
        \end{enumerate}
        where the last step brings the helicopter back to a stable hovering state
        in case the landing failed (e.g. if it falls of the landing platform).

        Each of these steps can be implemented as modes in a general
        mode-selected controller framework where mode transition conditions
        can be generally implemented.

    \subsection{Control}
        The outer loop controller will be implemented as a, possibly scheduled,
        LQG controller using a linearized dynamic model of the quadrocopter.
        The controller will be used to follow a trajectory reference as computed
        given a set of waypoints.
