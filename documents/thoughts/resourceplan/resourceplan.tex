\section{Resource plan}
\label{sec:resourceplan}
This section describes the initial plan on how to use the processing
power available on the LinkQuad.
    \subsection{Available resources}
        Each LinkQuad is equipped with a LinkBoard.
        A LinkBoard is in turn equipped with
        \begin{itemize}
            \item Sensor MCU,
            \item Control MCU,
            \item two Gumstix\textsuperscript{\textregistered} modules running Linux.
        \end{itemize}

    \subsection{Resource usage}
        \paragraph{Sensor MCU: } The sensor MCU will be used to sample
        and compile sensor data from accelerometers, gyroscopes and pressure sensor.
        If possible, the MCU will oversample the measurements, and at
        a fixed rate of e.g. 100 Hz send the data to the observer, implemented on
        one of the Gumstix\textsuperscript{\textregistered} modules.
        The module should also be able to send filtered measurements to the
        backup control MCU.
    
        %~ \begin{figure}[h]
            %~ \noindent\makebox[\textwidth]{%
                %~ \begin{tikzpicture}[node distance = 3cm]
                    %~ % Place nodes
                    %~ \node [block] (init) {initialize};
                    %~ \node [block, right of=init] (sample) {sample sensors};
                    %~ \node [block, right of=sample] (send) {send sensor data};
%~ 
%~ 
                    %~ \path [->]
                        %~ (init) edge (sample)
                        %~ (sample) edge (send)
                        %~ (send) edge [out=-90, in=-90] (sample)
                    %~ ;
                %~ \end{tikzpicture}
            %~ }
            %~ \caption{Rough outline of the Sensor MCU structure}
            %~ \label{fig:sensorMcuFlow}
        %~ \end{figure}


        \paragraph{Control MCU: } The control MCU is important in the aspect
        that it acts as a watchdog, should a failure occur with the primary
        controller on one of the Gumstix\textsuperscript{\textregistered} modules.
        On failure, it reverts control to the lower level stabilizing PID loop,
        which can then be used to safely manually land the LinkQuad.
        The control MCU is also used to actuate the higher-level controls 
        given by the controller on the Gumstix\textsuperscript{\textregistered} module.
        
    
        %~ \begin{figure}[h]
            %~ \noindent\makebox[\textwidth]{%
                %~ \begin{tikzpicture}[node distance = 3cm]
                    %~ % Place nodes
                    %~ \node [block] (init) {initialize};
                    %~ \node [decision, right of=init] (timeout) {timeout};
                    %~ \node [block, right of=timeout] (stabilize) {stabilize};
                    %~ \node [block, right of=stabilize] (actuate) {actuate};
                    %~ \node [block, below of=stabilize] (command) {command};
%~ 
%~ 
                    %~ \path [->]
                        %~ (init) edge (timeout)
                        %~ (timeout) edge[in=180,out=0] node[near start] {yes} (stabilize)
                        %~ (timeout) edge[in=180,out=-90] node[near start] {no} (command)
                        %~ (stabilize) edge (actuate)
                        %~ (command) edge [out=0,in=-90] (actuate)
                        %~ (actuate) edge [out=90, in=90] (timeout)
                    %~ ;
                %~ \end{tikzpicture}
            %~ }
            %~ \caption{Rough outline of the Control MCU structure}
            %~ \label{fig:controlMcuFlow}
        %~ \end{figure}

        \paragraph{Controller and observer Gumstix\textsuperscript{\textregistered} module: }
        One of the Gumstix\textsuperscript{\textregistered} modules will be primarily
        used for logic, state observation - i.e. run the UKF algorithm - and
        controller. All modules measuring some kind of sensory percept
        should report this to the observer which will fuse the information
        to a single best estimate of the full state. The state can then
        be immediately used by the controller to stabilize the LinkQuad
        and follow the reference.

        \paragraph{Video processing Gumstix\textsuperscript{\textregistered} module: }
        The second Gumstix\textsuperscript{\textregistered} module will be dedicated
        to the evaluation of the video feed in terms of running the vSLAM
        algorithm and keeping track of the map.
        This will require intimate two-way communication with the observer.
