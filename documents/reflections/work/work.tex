\section{Work}
\label{sec:work}
	During the entire thesis, the ambition to create a well designed, advanced 
	implementation was set high. 
	The planning and preparations for the thesis were extensive, and included
	a preparatory study of the required theory and previous work. 
	The planning was made in detail in Gantt charts and time for unexpected 
	delays were included in the planning.
	
	The high ambition of the project unfortunately resulted in a constantly high
	workload, and since there were quite an amount of components to the thesis - 
	each taking just a bit too long to finish, far too many nights were devoted to work.
	
	The work with the thesis was planned to follow the implementation of each
	implemented part of the thesis and in general, this plan was followed. 
	The writing, and the work, was made in sprints for each of the major parts
	of the thesis - state estimation, control and vSLAM. Each chapter
	was drafted after and during the implementation of each module, so
	the report should not fall behind.
	Nonetheless, the process of proof-reading became stressful, leaving
	some easy mistakes for quite a while before I was able to devote the time
	to do a proper read-through myself. By that time, however, I had open-mindedly
	listened to the comments of my tutors and were able to apply their general 
	comments better into the thesis.
	
	Above all, the thesis allowed me to deepen my knowledge in the field
	that I have spent the last five years studying for, and while the primary
	goal was not fully achieved, the implementation and results confirm that
	an engineer from the Y-program is capable of both the overview and detail
	needed to construct the control system of an autonomous flying vehicle, 
	and incorporate, implement and extend advanced techniques for use in 
	this application and others.
