\section{Goals}
\label{sec:goals}
	The thesis was performed in a very independent manner, and while
	assistance with both practical and theoretical aspects of the thesis 
	was available, much could be sorted out by thinking twice before asking.
	The tutors I received at IDA were very helpful throughout the work, even though
	the thesis -with its focus on SLAM, state estimation and control -
	did not entirely match their main field.
	
	During the thesis, I was impressed by the assistance I received from
	people entirely unrelated to my project. Especially the people of ISY
	were very helpful with their assistance in theoretical problems and issues
	along the way.
	
	Throughout the thesis, the focus was very fixed towards providing state-of-the-art
	control over the quadrotor to perform precision landing. 
	Several approaches to e.g. state estimation and SLAM were tried 
	and while simpler solutions in retrospect should have gained some
	precedence, the primary focus eventually resulted in the design of a modular decoupled
	system architecture capable of performing simulated landing using advanced,
	generally applicable, techniques. The implementation was fully backed up
	by theoretical derivations as necessary.
	
	Due to the amount of groundwork that had to be done before 
	evaluating real flight, the project never reached the point where
	real flight could truly be evaluated, although the implementation
	was actually able to perform in real time when run on more powerful hardware
	than what was available on the LinkQuad quadrotor that was targeted in the thesis.
	
	The system was evaluated using recorded data and simulation, although parts of the
	model would need continued model identification and verification to gain 
	further scientific credibility to the developed physical quadrotor motion model.
