\section{LQ Gain-Scheduling}
\label{sec:controller:gainscheduling}
    Even though any system could be described at any point by its linearization,
    the linear nature of the LQ control poses a limitation in that
    a general system such as the one studied in this thesis - the LinkQuad - will
    sooner or later leave the vicinity of the linearization point and no
    longer adhere to the physical circumstances of the linearization point.

    This will lead to sub-optimal control and possibly even to system failure.
    A common approach is to switch between pre-calculated control gains
    which has been calculated for selected linearization points.

    The approach used in this thesis is closely related to gain scheduling,
    but instead of using pre-calculated gains, the linearization is done
    in-flight at each time instant from the analytical expression of
    the system's Jacobian.

    \begin{equation}
        \dot{x} = f(x,u) \approx f(x_{0},u_{0})
            + \underbrace{\left. \frac{\partial f}{\partial x} \right|_{
                \begin{array}{l}
                    x=x_{0} \\
                    u=u_{0}
                \end{array}
            }}_{A}
                \underbrace{\left( x-x_{0} \right)}_{\Delta x}
            + \underbrace{\left. \frac{\partial f}{\partial u} \right|_{
                \begin{array}{l}
                    x=x_{0} \\
                    u=u_{0}
                \end{array}
            }}_{B}
                \underbrace{\left( u-u_{0} \right)}_{\Delta u}
    \end{equation}

    In the standard formulation of LQ, the linearization is made
    around a stationary point $(x_{0},u_{0})$, where $f(x_{0},u_{0}) = 0$.
    In a more general formulation, it is possible to lift this constraint
    using homogenous equations \citep{Rantzer99piecewiselinear};
    \begin{equation}
    \label{eq:controller:affinelq}
        \dot{X} = \left[
        \begin{array}{c}
            \dot{x} \\
            \ast
        \end{array}\right] =
        \underbrace{\left[
        \begin{array}{cc}
            A & f(x_{0},u_{0}) \\
            0 & 0
        \end{array}\right]}_{\bar{A}}
        \underbrace{\left[
        \begin{array}{c}
            \Delta x \\
            1
        \end{array}\right]}_{\bar{X}}
        +
        \underbrace{\left[
        \begin{array}{c}
            B \\
            0
        \end{array}\right]}_{\bar{B}}
        \Delta u.
    \end{equation}
    Equation \ref{eq:controller:affinelq} is a linear system for which
    the ordinary LQ problem can be solved, using eq.
    \eqref{eq:controller:u}-\eqref{eq:controller:Lr}.

    Generally the linearized output signal, $\Delta u$, would be added
    to $u_{0}$ to form the controller outout, as per
    \begin{equation*}
        u = u_{0} + \Delta u = u_{0} -L\Delta x + L_{r}\Delta r.
    \end{equation*}
    However, the approach to re-linearize the system continously
    leads to the property that, since the controller will always work
    at its linearization point, and we can note that $\Delta x$ will
    thus always be zero, giving the following expression for the
    calculation of the control output;
    \begin{equation}
        u_{t+1} = u_{t} + \Delta u = u_{0} + L_{r}(r-x).
    \end{equation}
