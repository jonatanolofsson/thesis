\section{The Linear Quadratic Controller}
\label{sec:controller:lq}
    In this section, the theory of Linear Quadratic control is presented,
    considering the continuous linear plant in \eqref{eq:controller:lq:plant},
    with control signal $u$ and reference $r$.
    \begin{subequations}
        \label{eq:controller:lq:plant}
        \begin{align}
            \dot{x} &= Ax + Bu \\
%
            z &= Mx \\
%
            e &= z - r
        \end{align}
    \end{subequations}

    The basic LQ controller, described in e.g. \citep{glad2003reglerteori},
    uses a linear state-space system model and costs on the states ($Q$) and control
    signals ($R$) respectively to calculate the control signals that would
    - given a starting state, a motion model and a constant reference -
    minimize the integral \eqref{eq:controller:lq:j}, the cost
    scaling quadratically with the deviation from reference.

    \begin{equation}
    \label{eq:controller:lq:j}
        \mathcal{J} = \int_{0}^{\infty} e^{T}(t)Qe(t) + u^{T}(t)Ru(t) dt.
    \end{equation}

    By varying the elements of the cost matrices $Q$ and $R$
    respectively, the solution to the optimization will yield control
    signals that will control the system in a fashion that the amplitude
    of the control signals and the errors are balanced.
    By e.g. increasing the costs of the control signals, the system
    LQ controller will issue smaller control signals, protecting the
    engines but slowing the system down.

    In the linear case, Eq.~\eqref{eq:controller:lq:j} can be solved analytically,
    resulting in a linear feedback gain, given in Eqs.~\eqref{eq:controller:u1}-\eqref{eq:controller:L}.
    \begin{align}
        u_{t} &= -L\hat{x}_{t} \label{eq:controller:u1} \\
%
        L &= R^{-1}B^{T}S \label{eq:controller:L}
    \end{align}
    $S$ is the Positively Semi-Definite (PSD) solution to the
    Continuous Algebraic Riccati Equation (CARE) \citep{glad2003reglerteori},
    stated in Eq.~\eqref{eq:controller:lq:care}.
    \begin{equation}
        \label{eq:controller:lq:care}
        A^{T}S + SA + M^{T}QM - SBR^{-1}B^{T}S = 0
    \end{equation}

    To improve the reference following abilities of the controller,
    the reference may be brought into the control signal by a scaling
    matrix $L_{r}$, describing the inverse system kinematics.
    $L_{r}$ is chosen so that the static gain of the
    system equals the identity matrix \citep{glad2003reglerteori}.
    In the case of equal number of control signals as controlled
    states, Eqs.~\ref{eq:controller:ufull}-\ref{eq:controller:Lr} are obtained.
    \begin{align}
        u_{t} &= -L\hat{x}_{t} + L_{r}r_{t} \label{eq:controller:ufull} \\
%
        L_{r} &= \left[M(BL - A)^{-1}B\right]^{-1} \label{eq:controller:Lr}
    \end{align}

%FIXME: Discrete results are used in the code!
