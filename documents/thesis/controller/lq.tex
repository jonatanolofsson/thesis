\section{The Linear Quadratic Controller}
\label{sec:controller:lq}
    The basic LQ controller, described in e.g. \citep{glad2003reglerteori},
    uses a linear state-space system model and weights on the states ($Q$) and control
    signals ($R$) respectively to calculate the control signals that would
    minimize the integral
    \begin{equation}
    \label{eq:controller:lq:j}
        \mathcal{J} = \int_{0}^{\infty} e^{T}(t)Qe(t) + u^{T}(t)Ru(t) dt.
    \end{equation}

    Thus, by varying the elements of the cost matrices $Q$ and $R$
    respectively, the solution to the optimization will yield control
    signals that will steer the system in a fashion that the amplitude
    of the control signals and the errors are balanced.
    By e.g. increasing the costs of the control signals, the system
    LQ controller will issue smaller control signals, protecting the
    engines but slowing the system down.

    In the linear case, \eqref{eq:controller:lq:j} can be solved analytically,
    resulting in a linear feedback
    \begin{equation}
    \label{eq:controller:u}
        u_{t} = -L\hat{x}_{t} + L_{r}r_{t}
    \end{equation}
    \begin{equation}
    \label{eq:controller:L}
        L = R^{-1}B^{T}S,
    \end{equation}
    where $S$ is the Positively Semi-Definite (PSD) solution to the
    Continuous Algebraic Riccati Equation (CARE)\citep{glad2003reglerteori},
    \begin{equation}
        A^{T}S + SA + M^{T}QM - SBR^{-1}B^{T}S = 0.
    \end{equation}

    To improve the reference following abilities of the controller,
    the reference is brought into the control signal by a scaling
    matrix $L_{r}$, which is chosen so that the static gain of the
    system is equal to identity \citep{glad2003reglerteori}.
    In the case of equal number of control signals as controlled
    states, the following result is obtained;
    \begin{equation}
    \label{eq:controller:Lr}
        L_{r} = \left[M(BL - A)^{-1}B\right]^{-1}.
    \end{equation}
