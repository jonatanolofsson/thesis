\chapter{Nonlinear Control}
\label{cha:controller}
    The state estimate from the algorithms discussed in Chapter~\ref{cha:observer}
    can be used to control the quadrotor's movements according to a desired
    reference.

    In this thesis, a nonlinear controller is proposed to be applied to the
    control the movements of the
    physical system, using a model of the system to calculate the best
    (in a sense well defined in this chapter) signals of control to each
    of the motors driving the propellers.

    The controller approach proposed in this thesis is extended from the Linear Quadratic (LQ)
    controller, the theory of which is presented in Section~\ref{sec:controller:lq}.
    An extension to the technique of \textit{gain-scheduling} is discussed
    in Section~\ref{sec:controller:gainscheduling}.

    The physical model of the system was derived in Section~\ref{sec:observer:motionmodel}.
    In Section~\ref{sec:controller:model}, this is further developed
    and adapted for compatibility as a model for the controller.
    The controller is interfaced by providing references
    for the NED-frame velocities and the, body-fixed, yaw rate.
    The controller outputs reference angular rates for each propeller.

    \section{The Linear Quadratic Controller}
\label{sec:controller:lq}
    In this section, the theory of Linear Quadratic control is presented,
    considering the continous linear plant in \eqref{eq:controller:lq:plant},
    with control signal $u$ and reference $r$.
    \begin{subequations}
        \label{eq:controller:lq:plant}
        \begin{equation}
            \dot{x} = Ax + Bu
        \end{equation}
        \begin{equation}
            z = Mx
        \end{equation}
        \begin{equation}
            e = z - r
        \end{equation}
    \end{subequations}

    The basic LQ controller, described in e.g. \citep{glad2003reglerteori},
    uses a linear state-space system model and weights on the states ($Q$) and control
    signals ($R$) respectively to calculate the control signals that would
    - given a starting state, a motion model and a constant reference -
    minimize the integral \eqref{eq:controller:lq:j}.
    \begin{equation}
    \label{eq:controller:lq:j}
        \mathcal{J} = \int_{0}^{\infty} e^{T}(t)Qe(t) + u^{T}(t)Ru(t) dt.
    \end{equation}

    Thus, by varying the elements of the cost matrices $Q$ and $R$
    respectively, the solution to the optimization will yield control
    signals that will steer the system in a fashion that the amplitude
    of the control signals and the errors are balanced.
    By e.g. increasing the costs of the control signals, the system
    LQ controller will issue smaller control signals, protecting the
    engines but slowing the system down.

    In the linear case, \eqref{eq:controller:lq:j} can be solved analytically,
    resulting in a linear feedback,
    \begin{equation}
    \label{eq:controller:u1}
        u_{t} = -L\hat{x}_{t}
    \end{equation}
    \begin{equation}
    \label{eq:controller:L}
        L = R^{-1}B^{T}S,
    \end{equation}
    where $S$ is the Positively Semi-Definite (PSD) solution to the
    Continuous Algebraic Riccati Equation (CARE)\citep{glad2003reglerteori}, stated in \eqref{eq:controller:lq:care}.
    \begin{equation}
        \label{eq:controller:lq:care}
        A^{T}S + SA + M^{T}QM - SBR^{-1}B^{T}S = 0
    \end{equation}

    To improve the reference following abilities of the controller,
    the reference may be brought into the control signal by a scaling
    matrix $L_{r}$, which is chosen so that the static gain of the
    system is equal to identity \citep{glad2003reglerteori}.
    In the case of equal number of control signals as controlled
    states, Eqs.~\ref{eq:controller:ufull}-\ref{eq:controller:Lr} are obtained.
    \begin{equation}
    \label{eq:controller:ufull}
        u_{t} = -L\hat{x}_{t} + L_{r}r_{t}
    \end{equation}
    \begin{equation}
    \label{eq:controller:Lr}
        L_{r} = \left[M(BL - A)^{-1}B\right]^{-1}.
    \end{equation}

%FIXME: Discrete results are used in the code!

    \section{State-Dependent Riccati Equations and LQ Gain Scheduling}
\label{sec:controller:gainscheduling}
    Even though any system could be described at any point by its linearization,
    the linear nature of the LQ control poses a limitation in that
    a general system such as the one studied in this thesis - a quadrotor - will
    sooner or later leave the vicinity of the linearization point and no
    longer adhere to the physical circumstances valid there.

    This will lead to sub-optimal control and possibly even to system failure.
    A common approach in non-linear control is to switch between pre-calculated
    control gains which has been calculated for selected linearization points.

    The approach used in this thesis is closely related to gain scheduling,
    but instead of using pre-calculated gains, the linearization is done
    in-flight.

    The problem of solving of the Riccati equation  online is treated under
    the subject of \textit{State-Dependent Riccati Equations}, SDRE's.
    While beeing computationally heavier than the standard LQ formulation
    due to having to solve the Riccati equation each time step, the need
    for finding valid linearization points is removed and more general problems can be treated.
    The basic formulation of the problem is covered in e.g. \citep{Rantzer99piecewiselinear},
    and an extensive survery is presented in \citep{Tayfun08sdresurvey}.
    Implementation details are detailed and evaluated in e.g \citep{Erdem_analysisand,Benner98acceleratingnewton's,10.1109/MED.2006.328740}.

    The LQ theory is extended to the non-linear case using the Taylor
    expansion of a general motion model is exploited to form the general
    result of Equation \eqref{eq:controller:affinelq}.
    Similar to several other modifications to the LQ methodology - such as
    Model Predictive Control (MPC) - the cost associated with the control signal
    is often applied relative to its current level, $\Delta u$, to avoid
    forcing all control signals to zero in the optimization problem that is solved.
    In each time-step the local linearization of the motion model, as presented in
    Eq. \eqref{eq:controller:gainscheduling:xdot}, is used to solve the LQ-equations for $\Delta u$.

    \begin{equation}
        \label{eq:controller:gainscheduling:xdot}
        \dot{x} = f(x,u) \approx f(x_{0},u_{0})
            + \underbrace{\left. \frac{\partial f}{\partial x} \right|_{
                \begin{array}{l}
                    x=x_{0} \\
                    u=u_{0}
                \end{array}
            }}_{A}
                \underbrace{\left( x-x_{0} \right)}_{\Delta x}
            + \underbrace{\left. \frac{\partial f}{\partial u} \right|_{
                \begin{array}{l}
                    x=x_{0} \\
                    u=u_{0}
                \end{array}
            }}_{B}
                \underbrace{\left( u-u_{0} \right)}_{\Delta u}
    \end{equation}

    In the standard formulation of LQ, the linearization is made
    around a stationary point $(x_{0},u_{0})$, where $f(x_{0},u_{0}) = 0$.
    In a more general formulation, it is possible to lift this constraint
    using a homogenous state \citep{Rantzer99piecewiselinear};
    \begin{equation}
    \label{eq:controller:affinelq}
        \dot{X} = \left[
        \begin{array}{c}
            \dot{x} \\
            0
        \end{array}\right] =
        \left[
        \begin{array}{cc}
            A & f(x_{0},u_{0})-Ax_{0} \\
            0 & 0
        \end{array}\right]
        \underbrace{\left[
        \begin{array}{c}
            x \\
            1
        \end{array}\right]}_{X}
        +
        \left[
        \begin{array}{c}
            B \\
            0
        \end{array}\right]
        \Delta u.
    \end{equation}
    Eq.~\ref{eq:controller:affinelq} is a linear system for which
    the ordinary LQ problem can be solved, using Eq.
    \eqref{eq:controller:u}-\eqref{eq:controller:Lr}.

    The linearized output signal, $\Delta u$, is then added
    to $u_{0}$ to form the controller output, as in Equation \eqref{eq:controller:u}.
    \begin{equation}
    \label{eq:controller:u}
        u = u_{0} + \Delta u = u_{0} - L\bar{X} + L_{r}r
    \end{equation}

    However, the linearizing extension of the affine controller in \eqref{eq:controller:affinelq}
    introduces a non-controllable constant state, with
    an associated eigenvalue inherently located in the
    origin.
    This poses a problem to the traditional solvers of the Riccati equation, which
    expects negative eigenvalues to solve the problem numerically, even though
    the weights of \eqref{eq:controller:lq:j} theoretically could be chosen to
    attain a well defined bounded integral.

    The problem is circumvented by adding slow dynamics to the theoretically
    constant state, effectively nudging the eigenvalue to the left of the
    imaginary axis to retain stability.
    This guarantees that \eqref{eq:controller:lq:j} tends to zero.

    Equation \eqref{eq:controller:affinelq} is thus really implemented as in \eqref{eq:controller:affinelqimplementation}.

    \begin{equation}
    \label{eq:controller:affinelqimplementation}
        \dot{X} = \left[
        \begin{array}{c}
            \dot{x} \\
            0
        \end{array}\right] =
        \left[
        \begin{array}{cc}
            A & f(x_{0},u_{0})-Ax_{0} \\
            0 & \mathbf{-10^{-9}}
        \end{array}\right]
        \underbrace{\left[
        \begin{array}{c}
            x \\
            1
        \end{array}\right]}_{X}
        +
        \left[
        \begin{array}{c}
            B \\
            0
        \end{array}\right]
        \Delta u.
    \end{equation}

    \section{Control Model}
\label{sec:controller:model}
    In Chapter~\ref{cha:observer}, a physical model of the system
    was derived. To incorporate the information from the physical model
    into the governing control law, the model needs to be fitted into Eq.
    \eqref{eq:controller:affinelq} by providing the Jacobi matrices with
    regards to $x$ and $u$, and removing states which will not be
    used in the controller.

    The Jacobians of the system are aquired numerically by using
    central difference
    \begin{equation}
        f'(x) \approx \frac{f(x+h) - f(x-h)}{2h}
    \end{equation}

    While all states are of importance to the dynamics of the quadrotor,
    it should be noted that only a subset of the states in $x$ is used for control.
    We thus need to form new matrices containing the relevant states.
    \paragraph{Notation}
    \begin{description}
        \item[] $\bar{x}$ denotes the rows in $x$ that are used for control.
        \item[] $\bar{x}^{\dagger}$ is used to denote the rows in $x$ that are \textit{not} used for control.
        \item[] $\bar{A} = [ \bar{A}^{\square} \bar{A}^{\dagger}]$
        defines the separation of the square part ($\square$) of $\bar{A}$ with columns
        associated with the controller states, from the other columns ($\dagger$).
    \end{description}

    The new matrices need to be formed only with the rows that are to be used
    for control. Extracting those rows from Equation \eqref{eq:controller:gainscheduling:xdot}
    yields Equation \eqref{eq:controller:model:derivation}.
    \begin{equation}
        \label{eq:controller:model:derivation}
        \begin{array}{rl}
             \dot{\bar{x}} = \bar{f}(x,u) \approx & \bar{f}(x_{0},u_{0}) + \bar{A}(x - x_{0}) + \bar{B}\Delta u \\
             = & \bar{f}(x_{0},u_{0}) - \bar{A}^{\square}\bar{x}_{0} - \bar{A}^{\dagger}\bar{x}^{\dagger}_{0} + \bar{A}^{\square}\bar{x} + \bar{A}^{\dagger}\bar{x}^{\dagger} + \bar{B}\Delta u \\
             = & \bar{f}(x_{0},u_{0}) - \bar{A}^{\square}\bar{x}_{0} + \bar{A}^{\square}\bar{x} + \bar{B}\Delta u
        \end{array}
    \end{equation}
    In the last equality of Eq.~\eqref{eq:controller:model:derivation}, it
    is assumed that the states not described in the controller are invariant
    of control and time, giving $\bar{x}^{\dagger}_{0} = \bar{x}^{\dagger}$.
    The results of Equation \eqref{eq:controller:model:derivation} can be
    directly fitted into Equation \eqref{eq:controller:affinelq} to form
    the new matrices for the controller.
    Note also that the derivation in \eqref{eq:controller:model:derivation} is
    completely analogous in the discrete-time case.

    %~ \section{Solving the Riccati Equation}
\label{sec:controller:riccati}

