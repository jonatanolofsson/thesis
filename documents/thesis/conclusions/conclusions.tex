\chapter{Concluding Remarks}
\label{cha:conclusions}
    This thesis covers the theory and implementation of the most
    important aspects of autonomous flight. The implementation is
    still in need of tuning, both for state estimation accuracy and control,
    but the foundations for high-level control and estimation using
    modern and effective algorithms have been laid.

    While the implementation has yet to see real flight, it is believed
    that relatively little work remain before test-flights can be performed,
    the most important work being tuning of state-estimation and control.

\section{Conclusions}
    The system proposed in this thesis has been implemented and verified
    using the framework presented in Appendix~\ref{app:crap}.
    The verification suggests that with further work, the proposed approach
    is a feasible solution for landing a quadrotor autonomously.
    The advanced algorithms used in the thesis allow for
    simple scaling of the solution, which also makes the method applicable
    for other types of platforms, or platforms with other sensor configurations.

    Positioning using Monocular SLAM has been shown to improve the estimation of the states
    of the quadrotor significantly. By extending the PTAM library,
    it has been possible to automate the initialization and utilization
    while also providing the coupling to a real-world metric system needed
    for positioning applications.

    Even in an untuned state, the positioning provides a state estimate
    that is adequate for the nonlinear control methodology presented
    in Chapter~\ref{cha:controller} to perform well in simulation.
    The controller extends the well-studied Linear Quadratic controller
    to a far more general case, while retaining the appreciated simplicity of tuning.
    The controller successfully lands a quadrotor in a simulated environment,
    and performs well in symbiosis with the suggested filtering algorithms.


\section{Further work}
\label{sec:conclusions:furtherwork}
    %~ Wind model, drag model, wind momentum on body
    While the implementation still needs tuning, it shows promise
    and one can see many fields which would prove interesting for
    future in-detail study - filter, model and control tuning included.

    \subsection{Filtering and Control}
        As a future study, it would be of interest to make a comparison between
        a fully tuned advanced model, compared to simpler motion models that
        can be applied in the filtering framework.
        Also, after the proposed model has been properly verified and tuned, further
        modeling may be of interest.

        As more processing power becomes available one can, in the filtering
        process, consider using even more state-of the art filtering techniques,
        such as GPU-implemented Particle filtering.
        Similarly, simpler control models could be compared with the
        advanced SDRE-solution proposed in this thesis.

        Future work could also investigate the applications of other types of
        optimal control, such as Model Predictive Control, MPC.

    \subsection{Monocular SLAM}
        One of the main results of this thesis is the relating of the PTAM
        coordinate system to the world coordinate system.
        This result could be relevant for any future work including real-world
        positioning using video feedback.

        There is also a potential to improve the PTAM library.
        The PTAM library was specifically developed for hand-held cameras
        without any additional sensors, yet the algorithm would benefit
        from such. The state estimate and motion model of the
        observer could be utilized to improve the quality and stability of
        the camera tracking.

        The PTAM library also exhibit performance issues when the internal map of features
        grows. This could be improved by using informed search,
        for instance using R-trees%\footnote{\url{http://en.wikipedia.org/wiki/R-tree}}
        , to index features according to their position and limit the search for
        relevant features to re-project into the image.
