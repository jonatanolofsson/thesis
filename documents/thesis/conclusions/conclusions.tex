\chapter{Concluding Remarks}
\label{cha:conclusions}
    This thesis covers the theory and implementation of the most
    important aspects of autonomous flight. The implementation is
    still in need of tuning, both for state-estimation accuracy and control,
    but the foundations for high-level control and estimation using
    modern and effective algorithms have been laid.

    While the implementation has yet to see real flight, it is believed
    that little work remain before test-flights can be performed,
    the most important work beeing tuning of state-estimation and control.

    With extensions to the state-machine logic, the system is prepared to
    perform advanced missions, including such planned by a future on-board
    artificially intelligent mission planner.

\section{Further work}
\label{sec:conclusions:furtherwork}
    %~ Wind model, drag model, wind momentum on body
    While the implementation still needs tuning, it shows promise
    and one can see many fields which would prove interesting for
    future in-detail study - filter, model and control tuning included.

    \subsection{Filtering and Control}
        As a future study, it would be of interest to make a comparison between
        a fully tuned advanced model, compared to simpler motion models that
        can be applied in the filtering framework.
        Also, after the proposed model has been properly verified and tuned, further
        modelling may be of interest.

        As more processing power becomes available one can, in the filtering
        process, consider using even more state-of the art filtering techniques,
        such as GPU-implemented Particle filtering.
        Similarly, simpler control models could be compared with the
        advanced SDRE-solution proposed in this thesis.

        Future work could also investigate the inclusion of other types of
        optimal control, such as Model Predictive Control, MPC.

    %~ \subsection{Mission Planning}
        %~ With relatively little work, the state-machine logic could
        %~ be extended with autonomous mission planning, using parallel state
        %~ machines and more advanced planning tools.
        %~ The graphical front-end could be replaced or extended with tools
        %~ suitable for mission control.

    \subsection{Monocular SLAM}
        One of the main results of this thesis is the relating of the PTAM
        coordinate system to the world coordinate system.
        This result could be relevant for any future work including real-world
        positioning using video feedback.

        There is also a potential to improve the PTAM library.
        One could, for instance, consider closing the loop, using the observer's
        pose estimate and motion model to improve the quality of the camera tracking.
        The PTAM library was specifically developed for hand-held cameras
        without any other sensors available, yet the algorithm would benefit
        of such information.
        Care would of course have to be taken to avoid the problems of
        information looping between the filters.

        There are also some performance issues when the internal map of PTAM features
        grows. This could be improved by using informed search,
        for instance using R-trees%\footnote{\url{http://en.wikipedia.org/wiki/R-tree}}
        , to index features according to their position and limit the search for
        relevant features to reproject into the image.


\section{Conclusions}
    %~ The achievements and theory presented in this thesis represent a set
    %~ of higher level algorithms than is common in the MAV field.
    %~ This is highly intentional, as I believe it is the untested solutions
    %~ that need to be tried in order to advance research in the area.
    %~ While one may argue that a simpler solution is ``good enough'',
    %~ superiority is not achieved by mimicking results but by expanding the ideas
    %~ of the cutting edge technology presented by a world of hobbyists,
    %~ developers and scientists.
    The system proposed in this thesis has been implemented and verified
    using the framework presented in Appendix~\ref{app:crap}.
    The verification suggests that with further work, the suggested approach
    is a feasible solution for landing a quadrotor autonomously.
    The advanced algorithms used in the thesis allow for
    simple scaling of the solution, which also makes the method applicable
    for other types of platforms, or platforms with other sensor configurations.

    Positioning using video-feed has been shown to improve the estimation of the states
    of the quadrotor significantly. By extending the PTAM library,
    it has been possible to automate the initialization and utilization
    while also providing the coupling to a real-world metric system needed
    for positioning applications.

    The positioning provides a state estimate that is adequate for the
    non-linear control methodology presented in Chapter~\ref{cha:controller}
    to perform well.
    The controller extends the well-studied Linear Quadratic controller to a far more general case,
    while retaining the appreciated simplicity of tuning.
    The controller successfully lands a quadrotor in a simulated environment,
    and performs well in symbiosis with the suggested filtering algorithms.

    %~ The algorithms and implementations that form the filtering framework allows
    %~ for simple addition of new future sensors and models.
    %~ This high-level filtering approach allows the information from all sensors
    %~ to be fused to form a common, better, estimate which with proper tuning.
    %~ This approach, standard in modern state estimation, is provably beneficial.

    \vspace{0.5cm}


    \noindent To be able to stand on the shoulders of previous work it is important
    that research is made publically available, both as theoretical reports
    but also their programmatic implementation.
    It is my firm belief that a scientific field will shine only
    with implementations available for others to extend.
    Therefore, the full implementation of this thesis is hereby
    released under the GNU General Public License\footnote{Full license is available online at \url{http://www.gnu.org/licenses/gpl.html}.}.
