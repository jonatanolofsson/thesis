\chapter{Discussion}
\label{cha:discussion}
    In this chapter, the results of the thesis are discussed,
    analyzing the contents of Chapter~\ref{cha:results} and providing
    a basis a for later discussion on future work.

    \section{Camera Localization}
    \label{sec:discussion:cameralocalization}
        Two libraries were evaluated for the camera localization part of this thesis,
        namely Scenelib and PTAM. Scenelib is the older library of the two,
        and is also the library to which PTAM was compared in its original paper \citep{klein07parallel}.
        The Scenelib library, using the filter approach described in Section~\ref{sec:video:filtersolutions},
        has the advantage to use metric measures.
        This was believed to potentially simplify the integration with the
        observer's measurement update, although in the end, difficulties
        with camera calibration rendered the Scenelib library subpar.

        The PTAM library is designed to be used with a wide-angle camera lens,
        which unfortunately was impossible with the available resources.
        While good performance could be achieved in the restricted testing
        environment, a wide-angle lens is likely to improve the tracking
        performance when larger scenes are explored.

        One can note that tracking was quite stable in recorded test cases
        with the low capture rate of $15$~Hz. A faster rate is nonetheless advisable.

    \section{Filtering}
        The results for the state estimation algorithms of this thesis
        were presented in Sections~\ref{sec:results:modelling}-\ref{sec:results:filtering} of Chapter~\ref{cha:results}.
        In the evaluation, the parameters of the motion model were set to reasonable values
        but otherwise untuned. Similarly, filter covariances were only roughly
        tuned. Video was recorded for $20$ seconds ($30$~Hz mode)
        or $40$ seconds ($15$~Hz mode). The Vicon ground truth was used only for
        initialization and plot reference.

        \subsection{EKF vs. UKF}
            Two nonlinear filter algorithms were studied in this thesis.
            After an initial evaluation period of the Unscented Kalman Filter,
            one weakness of the algorithm was exposed that rendered the
            algorithm very difficult to use further.

            Since the distribution of sigma points in the UKF is scaled with
            the covariance, they move farther away as the positioning gets
            less confident.
            With a fragile model such as that of a quadrotor,
            evaluating the model in too large offsets from reasonable ranges
            will cause effects large enough to destabilize the filter.
            Excessive covariance in angle estimates may for instance
            cause the filter to evaluate and compare two cases of the quadrotor
            being upside down, both irrelevant to the mean case of stable, vertical, flight.
            Hence, the implementation was changed to use the more stable EKF algorithm.

            With further tuning, the UKF may be usable in a later stage 
            of development, but with the inherent instability of the model, 
            the UKF is still problematic in the development stage, 
            as measurements are not always available to reduce the covariance.

        \subsection{Performance}
            The quality of the tracking, despite lack of tuning and previously
            mentioned problems, is in some cases remarkable given the
            preconditions - the angle estimates in particular.
            The most important factor for this is the high precision achieved by the
            camera positioning. By experimenting with filter tuning, it
            is apparent that noise from the pressure sensor
            in Figure~\ref{fig:results:modelling:pressure} affects the 
            altitude positioning negatively. 
            The trends in measured and predicted pressure of Figure~\ref{fig:results:modelling:pressure}
            seem to correlate with the ground truth altitude in Figure~\ref{fig:results:filtering:position},
            so further tuning and modeling may increase the utility of the pressure sensor.
            Due to its problematic behavior, the pressure sensor was disabled in the filter verification.

            With the proposed filter structure in place, the single
            most rewarding factor to increase the quality of the filtering
            would be simply to further tune it. Since this can be quite time-consuming,
            it could not be included in the thesis, but tools for simulation
            and evaluation are provided in the implementation.

            Currently, the integration of the continuous model from Chapter~\ref{cha:observer}
            is performed using numerical Euler integration.
            This is a simple and fast integration method, but lacks in accuracy.
            Applying a more advanced solution, e.g. a standard Runge-Kutta solver,
            could be expensive for a complex model, but a study to determine
            a reasonable trade-off using different integration techniques could be performed.
            The cost of integration may very well outweigh
            the performance gained by applying the advanced motion model,
            and the precision loss caused by the Euler integration might
            exceed the gain from the finer details of the said model.
            A simpler model could remove the need for numerical integration
            entirely and provide a significant speed-up. The complex model
            would however still be highly relevant for validation and
            development purposes, and the complexity of the modeling could
            be varied for the different applications of the model.

    \section{Modeling}
        It is arguable whether the detail of the physical modeling presented
        in Chapter~\ref{cha:observer} is tenable, as performance similar or better
        than presented in Section~\ref{sec:results:filtering} is very likely
        to have been achieved using a well tuned, far simpler, motion model.
        These results are not surprising for the untuned state of the model, however.
        Aside from requiring accurate parameter settings, the full potential of the
        proposed model is dependent on the availability of control reference
        signals and more accurate control over the true rotational velocity of
        the propellers. Feedback of the measured motor RPM is likely to appear in
        a later stage of the development of the LinkQuad, and may thus
        be available in the future.

        The model, as presented in this thesis, needs further in-detail
        verification and tuning. The flapping equations presented in
        Section~\ref{ssec:observer:thrust} were in fact even disabled during
        the verification in Chapter~\ref{cha:results} due to the
        difficulty of verification. In its current state, the model
        adequately describes the accelerations of the system - and with tuning it might become great -
        but the performance of the model must also be weighed against
        the computational complexity.

        The structure and extensiveness of the model does however open
        for new applications, including future outdoor applications, due
        to its handling of wind and other disturbances.
        The state-space type of model also allows for simple addition
        of further modeling and added - or removed - states.

        The proposed model also bear the advantage that the
        linearization performed in the Extended Kalman Filter
        may be cleverly re-used in the control algorithm, removing duplication
        of computations. In fact, by clever use of the model,
        only a single model needed to be developed in the thesis, then re-used for
        state estimation, control, simulation of reality and sensor measurements.
        The model can, as future work, be simplified, yet still benefit from the
        added detail where suitable.

        One can note that the number of physical constants can be reduced 
        using model identification for some of the equations of Chapter~\ref{cha:observer}.
        The importance of each force discussed should be investigated,
        and forces of lesser importance could be removed to reduce the complexity of the model.

    \section{Controller}
        The advantages and disadvantages of different control techniques
        and its applications to quadrotor control have been previously
        discussed and evaluated in \citep{bouabdallah04pid}.
        The conclusions are ambiguous, but the state-space approach is believed to
        excel with continued modeling.

        One disadvantage with the state-space approach is its - in comparison
        to PID control - lack of easily implemented error integration.
        This adds demands on a physically correct model, which may
        not always be available in all flight-conditions.

        The SDRE approach used in this thesis is notably rare - if at
        all existing - in the application of quadrotor control
        and with its ease of use, it is an interesting topic for further research and applications.
        The control implementation is very versatile although the computational
        burden could be somewhat lifted by comparing and applying other techniques for solving
        the Riccati equation of Eq.~\eqref{eq:controller:lq:care} and its discrete equivalent.
        Several numerical approaches exist to solving the Riccati equation,
        although an iterative approach could prove beneficial since
        the equation is solved repeatedly and the previous solution would make a
        good starting guess.

        While the control would need to be further verified and tuned before applied in real flight,
        the tools and structure have been verified in simulation with good results.
        It should be noted that the control model is fundamentally the same
        as the observer model and, more importantly, the underlying simulation
        model. Even though noise was added to both the system and the measurements,
        the control evaluation suffers from the fact that the control is ``perfect'' for the
        simulated model, and not necessarily for real flight.

        One clear advantage with the LQ control methodology is its simple
        tuning, and even though very little of the work was focused on
        tuning the control, the performance of the simulated control was acceptable.
        The control handled moderate simulated winds and disturbances without trouble, and
        followed velocity references adequately.
        Since the actuating of the yaw rate control is very weak in
        configurations such as the LinkQuad, slow control
        is to be expected here, especially without more tuning of the control
        parameters.
        
        An entirely different approach for control would be to replace the
        proposed control interface for a more subsidiarized control approach, using
        existing structure for controlling rotational velocities of the
        quadrotor. This would, however, remove the benefits of using
        the theory of optimal control to control the physical model directly.

    \section{State-machine Logic}
        The implementation of the state-machine engine uses a simple, 
        programmatically elegant, solution, detailed in Appedix~\ref{app:crap}. 
        The implemented modes are quite trivial, yet the structure is in place to implement
        far more advanced modes. In the future, the implementation could
        be extended to support concurrent state-machines as has been
        extensively studied in projects closely related to the LinkQuad.

        A text-book solution to detecting landing would be to introduce
        an estimated state of the ground force, which could be thresholded
        in the detection of landing. One finds, however,
        that such a force would be indistinguishable from
        the force acting on the quadrotor by a ever-increasing wind.
        In simulation, the landing detection failed to detect landing
        as proposed in Section~\ref{ssec:logic:landing:detection}, although
        when a landing is simulated, Figure~\ref{fig:results:filtering:windbump}
        exhibits an interesting property where a notable bump occurs at the
        time of the landing. While this is not the expected behavior
        - the estimate should be negative and constant -
        it does show that the wind may be useful as a detector for landing.
        This discrepancy of results is believed to be due to
        the simulation's lack of a model for sensor measurements at the event of ground contact,
        and the observer's lack of control signals in its state estimation.

        The results achieved in simulation implies that landing detection
        using wind estimation might be feasible, although more work is required
        to achieve the quality of estimation needed to securely detect landing.
\pagebreak
    \section{Real-time Performance}
        Although real-time performance is easily achieved on a modern laptop,
        simulations indicate that real-time performance
        could not be achieved on the LinkQuad's gumstix without modifications.

        The performance of the proposed algorithms when evaluated on the
        gumstix is penalized by the lack of a floating point processor (FPU) to
        carry out the calculations. However, the gumstix has a
        Digital Signal Processing unit (DSP) which might be possible to
        use as a FPU to increase execution speed.
        Attempts were in fact also made to convert libraries and algorithms from
        double precision calculations to single precision, although these attempts were deferred.
        This is still, however, believed to be of significance to achieve real-time performance.

        Furthermore, the system was evaluated
        using the reference implementation of BLAS\footnote{\textit{Basic Linear Algebra Subprograms},
        a standard set of mathematical routines.}, which could possibly
        be replaced by a tuned implementation such as ATLAS\footnote{\textit{Automatically Tuned Linear Algebra Software}}.
        Model simplifications may also be necessary to achieve real-time performance
        on the existing hardware, although the nonlinearities introduced 
        by the vehicle's rotations and thrust equations are believed to be relevant.
        Landing in the trivial case can no doubt be performed using a simpler, even linear, model,
        although generalisations extending the utility of the controller, for instance when
        using the controller for other flight cases, would likely require the use
        of a more advanced model, such as proposed in this thesis.
