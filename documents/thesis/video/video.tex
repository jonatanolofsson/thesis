\chapter{Monocular SLAM}
\label{cha:monoslam}
    In the interest of extracting positioning information from a video stream
    of a general, unknown, environment, the common approach is to use SLAM,
    \textit{Simultaneous Localisation And Mapping}.
    In the mathematical SLAM framework, features are identified and tracked throughout time
    in the video stream, and a filtering solution is then traditionally applied
    to estimate the probability densities of the tracked landmarks' positions, as well as
    that of the camera position.

    To determine the depth distance to a feature, stereo vision is often applied
    with which the correlation between two synchronized images is used together with the known
    distance between the cameras to calculate the image depth.
    As the distance to the tracked objects increase however, the stereo
    effect is lost and the algorithms' performance drops.
    There are several other issues to the stereo vision approach - increased
    cost, synchronization issues, computational load to name a few - which has
    led to the development of techniques to utilize the information on movement in only a single
    camera to calculate the depth of the features in the image.
    The application of SLAM to this approach is refferred to as \textit{Monocular SLAM},
    and two approaches are presented in this chapter.

    Both approaches rely on feature detection in video frames.
    An extensive survey of available feature-detecting algorithms
    is given in \citep{idris09reviewoffeaturedetection}.

    \section{Filtering Based Solutions}
        \label{sec:video:filtersolutions}
        One novel approach for camera tracking, used by for instance \citep{DBLP:conf/iccv/Davison03},
        is to use the EKF-SLAM framework and couple the feature
        and camera tracking in a time- and measurement update for each consecutive
        frame (\citep{Eade:2006:SMS:1153170.1153506} does this with FastSLAM 2.0).
        The \textit{Scenelib} library described in \citep{DBLP:conf/iccv/Davison03}
        uses a combination of the particle filter and the EKF for feature initialization
        and feature tracking respectively.

        Common to the filter approaches is that as new features are detected,
        they are initialized and appended as states to the filter.
        As frames arrives, the change of position of each feature is measured and
        the filter is updated accordingly.
        Thus, one must take care to avoid misassociation of the features, as
        this leads to false measurements that will mislead the filter.

        It is trivial to include motion models into the filtering approaches, since
        the general algorithm utilize a classical state-based filter time-update.

    \section{Keyframe-based SLAM}
        A fundamentally different approach is presented in \citep{klein07parallel},
        where the focus is put on collecting video frames of greater importance - keyframes -
        in which a large amount of features are detected.
        The camera's position is recorded at the time the keyframe is grabbed - see Figure~\ref{fig:video:ptamkeyframes} -
        and the new features detected are added to the active set.
        As new video frames arrive, features are reprojected and sought for in
        the new frame, giving a position offset from recorded keyframes which by
        extension gives the updated position of the camera.
        \fig{\imagewidth}{ptamkeyframes}{Keyframes containing features - here represented by dots - are recorded and the camera's position at the time of the frame's capture is stored and visualized as coordinate axes. The thick coordinate system displays the camera's current position estimate.}{fig:video:ptamkeyframes}


    \section{The PTAM Library}
        PTAM - Parallel Tracking And Mapping is a software library for camera tracking
        developed for Augmented Reality (AR) applications in small workspaces~\citep{klein07parallel}.
        It has only recently been applied for quadrotor state estimation~\citep{weiss11monocular},
        although as the library is intended for AR applications, the connection
        with the world's coordinate system is loose, as duly discussed in Chapter~\ref{ssec:observer:sensormodels:camera}.
        Several libraries exist extending the functionality of the PTAM library~\citep{Nguyen_Sandor_Park_2010}.

        As recently published, new algorithms - presented in \citep{Newcombe2011} - surpass
        the library's performance at the expense of computational cost
        and the demand for a high-profile graphics card.
        The performance is nonetheless proven\footnote{Examples of projects using PTAM are listed at \url{http://ewokrampage.wordpress.com/projects-using-ptam/}},
        and improves the usability over preceding libraries, such as \textit{Scenelib}, by including
        a camera calibration tool, estimating parameters describing the lens properties.
        \fig{\imagewidth}{ptamcalibration}{A checker-board pattern is used for calibrating the lens-specific parameters of the camera.}{fig:video:ptamcalibration}

        \subsection{Operation}
            In the tracking procedure, the PTAM library randomly projects previously recorded features into the
            captured video frame, visualized in Figure~\ref{fig:video:ptamtracking}. It could be argued that the feature projection in the PTAM library
            could make more use of external sensors and position estimates in this process.
            This, however, remains as future work.
            \fig{\imagewidth}{ptamtracking}{Features - visualized as dots - are projected into the image and sought for. The calculated offset from the features' original positions then yield a position estimate.}{fig:video:ptamtracking}

            As for mapping; Instead of continuously updating the keyframe's position estimates - as in the filtering solution -
            the PTAM library presented in \citep{klein07parallel} separates
            this task from the tracking entirely to a parallel, less time-critical, processing thread\footnote{Hence its name; \textit{Parallell Tracking And Mapping}}.
            As this update does not have to be performed in real-time each video-frame, this parallellization
            allows for the use of a more advanced adjustment technique \citep{klein07parallel}, namely
            Bundle Adjustment\citep{lour09}. The bundle adjustment problem
            performs a non-linear least-squares optimization over the available keyframe
            positions, utilizing the sparse structure of the optimization problem that
            is solved to make the solution computationally tractable~\citep{Lourakis11bundle}.

        \subsection{Modifications to the Library}
            The PTAM library is originally interfaced with a included graphical interface.
            However, as the source-code is provided for non-commercial use, the
            library was modified to interface over serial port with the CRAP framework
            and also extended with automatic initialization and full non-graphical use.
            This to enable full automatization on the development platform.

            In the standard PTAM initialization procedure, an image is captured
            at a starting position. The camera is then translated sideways until the user
            deems the stereo initialization can be performed with good results.
            A second frame is then captured, and features from the original image
            are tracked between the two to perform a stereo initialization to
            extract the ground plane and a placement for the origin.
            The tracked scene should be planar to retrieve a good estimation
            of the true ground plane, although the tracking will work regardless.

            \fig{\imagewidth}{ptamfeaturetracking}{To create the initial map, PTAM is initialized with a set of two offset images. Features are tracked during the initialization to find the ground plane.}{fig:video:ptamfeaturetracking}

            In the graphical interface, the tracked features are visualized
            during the process with lines in the camera image, as in Figure~\ref{fig:video:ptamfeaturetracking}.
            After the initial frame has been captured, the implementation
            of the automated initialization uses the length of these lines as
            a measure of when to finalize the initialization procedure.
            When the length of the line associated with the first\footnote{First in the list of features from the first image still found in the latest frame.}
            feature exceeds a given theshold, the second frame is
            captured and the stereo initalization algorithm is started.
            Should the initialization procedure fail - e.g. by loosing track of most features - the
            procedure will be restarted until a successful initialization has been performed.
            The procedure is initially triggered by a command from the serial port.

        \subsection{Practical Use}
            The PTAM library is designed to use a wide-angle lens, with
            performance dropping should such not be available~\citep{klein07parallel}.
            In its current state however, PTAM does not support fish-eye lenses, and although
            the LinkQuad-mounted camera features a changeable lens, all
            available wide-angle lenses are fish-eye.
            It is possible to use the PTAM library with a standard lens, although
            tracking is easier lost.

            The tracking and initialization quality is also - as all video tracking - dependent on the
            amount of trackable features in the scene.
            While the library can recover from lost positioning, an unsufficient amount
            of features can cause the initialization process to fail, or the tracking to
            be irrepairably lost.
            The amount of features tracked is also dependent on the processing
            power available, and because the complexity of the mapping
            problem grows fast with the number of keyframes, the libary may
            have difficulties extending the map to unexplored areas.
