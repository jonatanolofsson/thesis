\section{Previous work}
\label{sec:previouswork}
    Here, previous known similar work is referenced and discussed.
    Previous work can be used for both reference implemetations, but also to
    recognize limitations and restrictions posed on the systems studied.

    The problem of positioning an unmanned quadrotor using
    visual feedback is a problem which was only fairly recently solved
    \citep{DBLP:conf/icra/BloschWSS10,weiss11monocular}.
    Few have attempted to use on-board sensors only, but have relied on
    external setups to track the motions of the quadrotors - admittedly with high precision.
    Fewer still have succeeded using strictly real-time algorithms
    running on the limited processing power that standard
    UAV's of the size of the LinkQuad generally are equipped with.
    The LinkQuad is, with its distributed processing power, well suited
    for a full implementation and solution to the problem.

    \subsection{Autonomous landing}
    The problem of landing a quadrocopter can to a large extent be boiled
    down to achieving good pose estimates using available information.
    This problem is studied in e.g. \citep{DM:MS:10,brockers:803111},
    but the most interesting results are obtained in \citep{DBLP:conf/icra/BloschWSS10,weiss11monocular},
    where the ideas from \citep{klein07parallel} of using monocular SLAM
    are implemented on a UAV platform and excellent results are obtained.

    \citep{brockers:803111} implements a landing control reference scheme
    which is summarized in \ref{sec:logic:landingscheme}.

    \subsection{vSLAM}
    A first take on a vSLAM algorithm is presented in \citep{Karlsson05thevslam}, resting on the
    foundation of a Rao-Blackwellized particle filter with Kalman filter banks.
    The algorithm presented in \citep{Karlsson05thevslam} also extends to
    the case of multiple cameras, which could be interesting in a longer perspective.
    An implementation has been made in a ROS project\citep{rosvslam}, which may be used for
    reference.

    \citep{klein07parallel} uses an alternative approach and splits the tracking problem of SLAM from the mapping problem.
    This means that the mapping can run more advanced algorithms at a slower
    pace than required by the tracking.

    The algorithm proposed in \citep{klein07parallel} uses selected keyframes
    from which offsets are calculated continously in the tracking thread, while
    the mapping problem is addressed separately as fast as possible using information only
    from these keyframes. This opposed to the traditional vSLAM filtering
    solution where each frame has to be used for continuous filter updates.

    By for instance not considering uncertainties in either camera pose or feature location,
    the complexity of the algorithm is reduced and the number of studied points
    can be increased to achieve better robustness and
    performance\citep{DBLP:conf/icra/StrasdatMD10} than when a filtering solution is used.
    Again, this is the method on which \citep{weiss11monocular} bases its implementation.
