\chapter{Introduction}
\label{cha:introduction}
    The goal for this thesis was to develop and implement a high-level control system
    for the quadrotor developed by the AIICS\footnote{\url{http://www.ida.liu.se/divisions/aiics/}} team at Linköping University, the LinkQuad.
    The system was to demonstrate autonomous landing, and while this goal
    could not be reached within the time-frame of one Master's thesis, the control
    system developed in this thesis show promising initial results.

    Although the LinkQuad quadrotor was used as development platform for this thesis,
    the results are general and applicable to other quadrotors and, by extension, other UAV configurations.

    The LinkQuad is considered a MAV, a Micro Air Vehicle. % - A UAV small enough to be carried by man \citep{wiki:suav}.
    The size of the system poses limitations on the payload and processing power
    available in-flight. This means that the implementations has to be done
    in an efficient manner on the small computers that are available on the LinkQuad.

    These limitations, however, does not necessarily limit us from utilizing advanced estimation
    and control techniques, and it the LinkQuad design is in fact
    very suitable for the camera-based pose estimation due to its dual
    onboard computers which allows distribution of the workload.
    The video-based estimation can thus effectively be detached from the
    core control and state estimation and modularized as an independent
    virtual sensor, which is exploited in this thesis.

\section{Unmanned Aerial Vehicles}
    Unmanned Aerial Vehicles (UAVs) have been imagined and constructed for millennia, starting in ancient Greece and China \citep{valavanis2007advances}.
    The modern concept of UAVs was introduced in the first world war, which
    illuminates the dominant role that the military has played in the field over the
    last century. A commonly cited alliteration is that UAVs are intended to replace
    human presence in missions that are ''Dull, Dirty and Dangerous''.

    While the military continues to lead the development in the field \citep{united2010u}, recent years
    have seen a great increase in domestic and civilian applications \citep{Wong_Bil_2006}.
    These applications range from pesticide dispersing and crop monitoring to
    traffic control, border watch and rescue scenarios \citep{Doherty_Rudol_2007}.

    The type of UAV that is used in the implementation of this thesis
    falls under the category of Micro Air Vehicles (MAVs)\footnote{Definitions differ for the classification of UAVs,
    although a weight less than $5$~kg has been proposed \citep{arjomandi:classification}.
    Other terms, such as \textit{Small UAVs} (SUAVs), are used by e.g. \citep{valavanis2007advances}.}.
    MAVs are designed to be man-portable in weight and size and is thus
    limited in payload and available processing power.
    This limitation, in combination with the unavailability of indoor positioning (e.g. GPS),
    has led to extensive use of off-board positioning and control in recent research.
    Systems developed for instance by Vicon\footnote{\url{http://www.vicon.com/}} and
    Qualisys\footnote{\url{http://www.qualisys.com/}} yield positioning with
    remarkable precision, but they also limit the application to a confined
    environment with an expensive setup.

    This thesis seeks a different approach, with an efficient self-contained
    on-board implementation. GPS and external cameras are replaced by inertial sensors and an
    on-board camera which uses visual SLAM to determine the position of the LinkQuad relative to
    its surroundings.



\section{The LinkQuad Platform}
    The LinkQuad is a modular quadrotor developed at Linköping University.
    The core configuration is equipped with a standard set of sensors
    (accelerometers, gyroscopes, pressure sensors, GPS and magnetometers),
    but for our purposes, a monocular camera has also been mounted.
    The camera feeds data into a microcomputer specifically devoted to video processing.
    This devoted microcomputer is what allows the primary on-board microcomputer
    to focus on state estimation and control, without being overloaded by
    video processing.

    The primary microcomputer is running a framework named C++ Robot Automation Platform (CRAP),
    which was developed by the thesis' author for this purpose. CRAP is a light-weight
    automation platform with a purpose similar to that of ROS\footnote{\url{http://www.ros.org/}}, a
    modular robotics framework developed by Willow Garage\footnote{\url{http://www.willowgarage.com/}}.
    It is, in contrast to ROS, primarily designed to run on the relatively low-end Linux systems
    that fits the payload and power demands of a MAV. The framework is further
    described in Appendix~\ref{app:crap}.

    The platform is also equipped with two microcontrollers, responsible
    for sensor sampling, motor control, flight logging etc., as depicted in
    Figure~\ref{fig:platform:schematic}. The microcontrollers also contain
    an implementation of a complementary filter to provide angle and
    height estimates from the available sensors.

    Using the \crap framework, the functionality of the thesis implementation is
    distributed in separated modules:
    \begin{description}
        \item[$\bullet$ Observer:] Sensor fusing state estimation. Chapter~\ref{cha:observer}.
        \item[$\bullet$ Control:]  Affine Quadratic Control. Chapter~\ref{cha:controller}.
        \item[$\bullet$ Logic:]    State-machine for scheduling controller parameters and reference trajectory. Chapter~\ref{cha:logic}.
    \end{description}


    \begin{figure}[h]
        \noindent\makebox[\textwidth]{%
            \begin{tikzpicture}[auto,>=latex']
            \tikzstyle{mcu} = [rectangle, draw=none, rounded corners=1mm, fill=Emerald,
                text centered, anchor=north, text=white,font=\small]
            \tikzstyle{sensor} = [rectangle, draw=none, fill=black,
                text centered, anchor=north, text=white,font=\tiny, node distance=0.8cm]
            \tikzstyle{camera} = [rectangle, draw=none, rounded corners=1mm, fill=red,
                text centered, anchor=north, text=white,font=\small]
            \tikzstyle{framework} = [rectangle, draw, fill=YellowGreen,
                text centered, anchor=north, text=white,font=\small,thin, minimum width=2cm,node distance=0.7cm]
            \tikzstyle{gumstix} = [rectangle, draw=none, fill=LimeGreen, rounded corners=2mm,
                text centered, anchor=north, text=white,font=\small]
            \tikzstyle{gumstixtext} = [rectangle, draw=none,
                text centered, align=center, anchor=north, text=white,font=\small]

            \node [gumstixtext] (primarytext) {Primary computer \\ \tiny{State-estimation and control}};
            \node [framework, below of=primarytext] (crap) {CRAP};
            \node [gumstixtext, right of=primarytext, node distance=5cm] (secondarytext) {Secondary computer \\ \tiny{Video processing}};
            \node [framework, below of=secondarytext] (ptam) {PTAM};

            \begin{pgfonlayer}{background}
                \node [gumstix, fit=(primarytext) (crap)] (primary) {};
            \end{pgfonlayer}
            \begin{pgfonlayer}{background}
                \node [gumstix, fit=(secondarytext) (ptam)] (secondary) {};
            \end{pgfonlayer}

            \node [mcu, below of=primary, xshift=-1.5cm, node distance=1.5cm] (smcu) {Sensor MCU};
            \node [mcu, right of=smcu, node distance=3cm] (cmcu) {Control MCU};
            \node [sensor, below of=smcu, xshift=-1.2cm] (accelerometer) {Acc.};
            \node [sensor, right of=accelerometer] (gyro) {Gyro};
            \node [sensor, right of=gyro] (magnetometer) {Mag.};
            \node [sensor, right of=magnetometer] (pressure) {hPa};
            \node [camera, right of=secondary, node distance=3cm] (camera) {Camera};
            \node [sensor, below of=cmcu] (esc) {Speed controller};
            \node [sensor, right of=esc, node distance=1.8cm] (sdcard) {SD-card};

            \path[-]
                        (primary) edge (secondary)
                        (primary.south) edge (smcu.north)
                        (primary.south) edge (cmcu.north)
                        (cmcu.south) edge (esc.north)
                        (smcu.south) edge (accelerometer.north)
                        (smcu.south) edge (gyro.north)
                        (smcu.south) edge (magnetometer.north)
                        (smcu.south) edge (pressure.north)
                        (secondary.east) edge (camera.west)
                        (smcu.east) edge (cmcu.west)
                        (cmcu.south) edge (sdcard.north)
                        ;
            \end{tikzpicture}
        }
        \caption{LinkQuad schematic.}
        \label{fig:platform:schematic}
    \end{figure}

    \begin{subfigures}{The LinkQuad development platform.}{fig:linkquad}
        \sfig{0.45}{LinkQuad}
        \sfig{0.45}{linkquadoutside}
    \end{subfigures}

\section{Related Work}
\label{sec:previouswork}
    %~ Previous work can be used for both reference implemetations, but also to
    %~ recognize limitations and restrictions posed on the systems studied in the works to
    %~ extend the work further. In addition to the sources presented in this
    %~ Section, several more are cited in the following chapters.

    Positioning an unmanned quadrotor using
    visual feedback is a problem which has received extensive attention
    during recent years, and only recently been implemented with convincing
    results \citep{DBLP:conf/icra/BloschWSS10,weiss11monocular}.
    Few have attempted to use on-board sensors only, but have relied on
    external setups to track the motions of a quadrotor - reportedly with high precision.
    Fewer still have succeeded using strictly real-time algorithms
    running on the limited processing power that standard
    MAVs generally are equipped with, although successful implementations do exist, e.g. \citep{Rudol10}.
    %~ The LinkQuad is in that regard, with its distributed processing power, well suited
    %~ as a development platform when studying this problem.

    \subsection{Autonomous Landing and Control}
    The problem of landing a quadrotor can to a large extent be boiled
    down to achieving good pose estimates using available models and sensor measurements.
    The problem is studied in e.g. \citep{mellinger10perching,brockers:803111},
    but perhaps the most interesting results are obtained in \citep{DBLP:conf/icra/BloschWSS10,weiss11monocular},
    where the ideas from \citep{klein07parallel} of using Monocular SLAM
    are implemented on a UAV platform and good results are obtained,
    using the feedback from the camera's pose estimate for control.
    Some problems remain, however, regarding the long-term stability of the controller.

    A landing control scheme, inspirational to the approach in this thesis,
    is suggested by \citep{brockers:803111}, which is summarized in Section~\ref{ssec:logic:landing}.

    Both Linear Quadratic control and PID controllers have been used for control
    in aforementioned projects, and ambiguous results have been attained
    as to which is better \citep{bouabdallah04pid}. Continued effort of
    quadrotor modeling have however shown great potential of the LQ design \citep{bouabdallah07full}.

    \subsection{Visual SLAM}
    Applying the idea of Simultaneous Localization And Mapping (SLAM)
    to the data available in a video feed is known as Visual SLAM (vSLAM).
    A directional paper is presented in \citep{Karlsson05thevslam}, resting on the
    foundation of a Rao-Blackwellized particle filter with Kalman filter banks
    to track landmarks recognized from previous videoframes.
    %~ The algorithm presented in \citep{Karlsson05thevslam} also extends to
    %~ the case of multiple cameras, which could be interesting in a longer perspective.
    %~ An implementation has been made in a ROS project \citep{rosvslam}, which may be used for reference.
    Similar approaches, using Kalman filtering solutions, have been implemented by e.g. \citep{DBLP:conf/iccv/Davison03,Eade:2006:SMS:1153170.1153506},
    and are available as open-source software\footnote{\url{http://www.doc.ic.ac.uk/~ajd/Scene/}}.

    Another approach to vSLAM is suggested in \citep{klein07parallel} and
    implemented as the PTAM - Parallel Tracking And Mapping - library.
    This library splits the tracking problem - estimating the camera position -
    of SLAM from the mapping problem - globally optimizing the positions of
    registered features with regards to each other.
    Without the constraint of real-time mapping, this optimization can run
    more advanced algorithms at a slower pace than required by the tracking.

    Also, by for instance not considering the full uncertainties in either camera pose or feature location,
    the complexity of the algorithm is reduced and the number of studied points
    can be increased to achieve better robustness and
    performance than when a filtering solution is used \citep{DBLP:conf/icra/StrasdatMD10}.

    A modified version of the PTAM library has been implemented on an iPhone \citep{klein09cameraphone}.
    This is of special interest to this thesis, since it demonstrates
    a successful implementation in a resource-constrained environment similar to that available on a MAV.
    The library has been previously applied to the UAV state estimation field, as presented in \citep{weiss11monocular},
    although its primary field of application is Augmented Reality (AR).

    %~ The algorithm proposed in \citep{klein07parallel} uses selected keyframes
    %~ from which offsets are calculated continously in the tracking thread, while
    %~ the map optimization problem is addressed separately as fast as possible using information only
    %~ from these keyframes. This opposed to the traditional VSLAM filtering
    %~ solution where each frame has to be used for continuous filter updates.
    %~ Several existing implementations exist with this technique, e.g. as described by \citep{DBLP:conf/iccv/Davison03}.
    %~ Again, this is the method on which \citep{weiss11monocular} bases its implementation.




\section{Objectives and Limitations}
    The main objective for this thesis was to create a high-level control and state estimation
    system and to demonstrate this by performing autonomous landing with the LinkQuad.
    To achieve this, the main work was put into implementing and testing theoretically solid
    methods and algorithms for positioning and control.
    Having control over the vehicle, landing is then a matter of generating
    a suitable trajectory and detecting the completion of the landing.

    Although time restricted the final demonstration of landing, the necessary tools
    were implemented, albeit lacking the tuning necessary for real flight.

    This thesis does not cover the detection of suitable landing sites,
    nor any advanced flight control in limited space or with collision detection.
    The quadrotor modeling is extensive, but is mostly limited to a study
    of literature on the subject.

\section{Contributions}
    During the thesis work, several tools for future development have been
    designed and developed. The \crap framework collects tools that are usable in future
    projects and theses, both on the LinkQuad and otherwise.
    The modules developed for the \crap framework include:
    \begin{itemize}
        \item General nonlinear filtering, using EKF or UKF,
        \item General nonlinear control, implemented as affine quadratic control,
        \item Extendable scheduling and reasoning through state-machines,
        \item Real-time plotting,
        \item Communication API for internal and external communication.
    \end{itemize}

    Furthermore, a general physical model of a quadrotor has been assembled.
    The model extends and clarifies the many sources of physical modeling
    available, and is presented in a scalable, general manner.

    The state-estimation proposed in this thesis uses the full physical model
    derived in Chapter~\ref{cha:observer}. While the model still needs
    tuning, it does show promising results, and a physically modeled
    quadrotor could potentially improve in-flight performance.

    In Chapter~\ref{cha:observer}, a general method for retaining
    the world-to-PTAM transform is proposed.
    This method could prove useful for extending the utility of the PTAM
    camera positioning library to more than its intended use in Augmented Reality applications.
    The utility for this has already been proven in e.g. \citep{weiss11monocular},
    and providing the theory and implementation for this, as well as a proposed
    initialization routine, could be of great use in future work.
    The PTAM library has also been extended with full autonomous operation by
    proposing an automated initialization procedure as well as providing
    full detachment from the graphical user interface.

    All tools developed during the thesis are released under the GPL license
    and are available at \url{https://github.com/jonatanolofsson/}.

\section{Thesis Outline}
    Following the introductory Chapter~\ref{cha:introduction}, four chapters
    are devoted to presenting the theory and the equations used in the implementation, in order;
    \begin{description}
        \item[$\bullet$ State Estimation:] State estimation theory and physical modeling of a quadrotor,
        \item[$\bullet$ nonlinear Control:]  nonlinear control theory and its implementation,
        \item[$\bullet$ Monocular SLAM:] Video-based SLAM and its applications.
    \end{description}

    The following two chapters of the thesis, Chapters~\ref{cha:results} and \ref{cha:discussion}, present the
    numerical evaluation of the result and the following discussion respectively.
    Concluding remarks and suggestions for further work are then presented in Chapter~\ref{cha:conclusions}.

    A detailed description of the \crap framework is appended to the thesis.
