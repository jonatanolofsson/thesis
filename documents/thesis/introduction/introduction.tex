\chapter{Introduction}
\label{cha:introduction}
    The goal for this thesis is to develop and implement an autonomous landing mode
    for the quadrotor developed by the AIICS team at Linköping University, the LinkQuad.

    The size of the system poses limitations on the payload and processing power
    available in-flight. This means that the implementations has to be done
    in an efficient manner on the small computers that are available on the LinkQuad.

    The limitations, however, do not stop us from utilizing advanced estimation
    and control techniques, and it turns out that the LinkQuad design is in fact
    very suitable for the camera-based pose estimation.
    This estimation can be efficiently detached from the core control
    and modularized as an independent sensor, which will be exploited in the
    thesis.

\section{Unmanned Aerial Vehicles}
    As noted in \citep{valavanis2007advances}, Unmanned Aerial Vehicles (UAV's)
    have been imagined and constructed for millennia, starting in ancient Greece and China.
    The modern concept of UAV's was introduced in the first world war, which
    illuminates the dominant role that the military has played in the field over the
    last century. A commonly cited alliteration is that UAV's are intended to replace
    human presence in missions that are ''Dull, Dirty and Dangerous''.

    While the military\citep{united2010u} continue to lead the development in the field, recent years
    have seen a great increase in domestic and civilian applications\citep{Wong_Bil_2006}.
    These applications range from pesticide dispersing and crop monitoring to
    traffic control, border watch and rescue scenarios\citep{Doherty_Rudol_2007}.

    The type of UAV that is used in the implementation of this thesis
    falls under the category of Small Unmanned Aerial Vehicles (SUAV's).
    SUAV's are designed to be man-portable in weight and size and is thus
    limited in payload and availabe processing power.
    This limitation, in combination with the unavailability of indoor GPS positioning,
    has led to extensive use of off-board positioning and control in recent research.
    Systems developed by for instance Vicon\footnote{\url{http://www.vicon.com/}} and
    Qualisys\footnote{\url{http://www.qualisys.com/}} yield positioning with
    remarkable precision, but they also limit the application to a confined
    environment with an expensive setup.

    This thesis seeks a different approach, with an efficient self-contained
    on-board implementation. GPS and external cameras are replaced by inertial sensors and an
    on-board camera which uses visual SLAM to position the LinkQuad relative to
    its surroundings.

\section{The Platform}
    The LinkQuad is a modular quadrotor developed at Linköping University.
    The core configuration is equipped with standard MEMS sensors
    (accelerometers, gyroscopes and a magnetometer),
    but for our purposes we have also mounted a monocular camera which feeds data
    into a microcomputer specifically devoted to the processing of the camera feed.
    This devoted microcomputer is what allows the primary on-board microcomputer
    to focus on state estimation and control, without beeing overloaded by
    video processing.

    The primary microcomputer is running a framework named C++ Robot Automation Platform (CRAP),
    which was developed by the thesis' author for this purpose. CRAP is a light-weight
    automation platform with a purpose similar to that of ROS\footnote{\url{http://www.ros.org/}}.
    It is, in contrast to ROS, primarily designed to run on the kind relatively low-end Linux systems
    that fits the payload and power demands of a SUAV. The framework is further
    described in Appendix \ref{app:CRAP}.

    Using the framework, the functionality of the implementation is
    distributed in separate modules.
    \begin{description}
        \item[Observer] Sensor fusing state estimation. Chapter \ref{cha:observer}.
        \item[Control]  Outer loop LQ control. Chapter \ref{cha:controller}.
        \item[Logic]    State-machine for scheduling controller model and reference trajectory. Chapter \ref{cha:logic}.
    \end{description}

\section{Previous Work}
    The problem of positioning an unmanned quadrotor using
    visual feedback is a problem which was only fairly recently solved
    \citep{DBLP:conf/icra/BloschWSS10,weiss11monocular}.
    Few have attempted to use on-board sensors only


\section{Objectives}
    The main objective for the thesis is to perform autonomous landing
    with the LinkQuad. To achieve this,

\section{Contributions}
    None

\section{Thesis Outline}
