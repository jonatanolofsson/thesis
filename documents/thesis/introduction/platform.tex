\section{The Platform}
    The LinkQuad is a modular quadrotor developed at Linköping University.
    The core configuration is equipped with standard MEMS sensors
    (accelerometers, gyroscopes and a magnetometer),
    but for our purposes we have also mounted a monocular camera which feeds data
    into a microcomputer specifically devoted to the processing of the camera feed.
    This devoted microcomputer is what allows the primary on-board microcomputer
    to focus on state estimation and control, without beeing overloaded by
    video processing.

    The primary microcomputer is running a framework named C++ Robot Automation Platform (CRAP),
    which was developed by the thesis' author for this purpose. CRAP is a light-weight
    automation platform with a purpose similar to that of ROS\footnote{\url{http://www.ros.org/}}.
    It is, in contrast to ROS, primarily designed to run on the kind relatively low-end Linux systems
    that fits the payload and power demands of a SUAV. The framework is further
    described in Appendix \ref{app:CRAP}.

    Using the framework, the functionality of the implementation is
    distributed in separate modules.
    \begin{description}
        \item[Observer] Sensor fusing state estimation. Chapter \ref{cha:observer}.
        \item[Control]  Affine Quadratic Control. Chapter \ref{cha:controller}.
        \item[Logic]    State-machine for scheduling controller parameters and reference trajectory. Chapter \ref{cha:logic}.
    \end{description}
