\section{Experimental Setup}
\label{sec:results:setup}
    The data used for the evaluation of the model and the filter in
    this chapter was recorded on the LinkQuad quadrotor in the UASTech Vicon Lab
    at Linköping University, Sweden. Ground truth data was recorded using
    the Vicon tracking system at a rate of $10$~Hz, while sensors were sampled
    at $500$~Hz and logged on-board the LinkQuad.
    For the dataset used in this Chapter,
    camera data was collected at a rate of $30$~Hz during 20-second bursts after
    which the data had to be written to memory.
    The camera was tilted approximately $30$ degrees downwards from the
    horizontal body-fixed plane of the quadrotor, giving an overview of the cluttered
    floor as seen in Figure~\ref{fig:results:setup:testarea}.
    The camera settings were tuned to minimize the disturbance from light-sources
    and the infrared light used by the Vicon system.

    The LinkQuad was then manually moved by hand to resemble flight conditions
    while recording sensor data, synchronized with video frames and reference Vicon data.

    \fig{0.8}{testarea}{To provide visual features for the PTAM library to detect, the testscene was cluttered with objects.}{fig:results:setup:testarea}

    After the validation of the model, it was then used for simulated
    control and landing, presented in Section~\ref{sec:results:control},
    to validate the proposed control scheme.
    For the validation, simulated wind, random Gaussian system noise and
    random Gaussian measurement noise was injected into the system.
