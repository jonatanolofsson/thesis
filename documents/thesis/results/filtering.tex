\section{Filtering}
\paragraph{Positioning}
    While the altitude positioning, plotted in Figure \ref{fig:results:filtering:z},
    exhibit disturbances correlated with pressure sensor noise,
    the positioning exhibit very good performance. The position state
    is observed in the state-estimation more or less directly by the camera
    and the stability of the camera positioning is thus of course reflected here.
    \fig{\plotwidth}{X}{}{fig:results:filtering:x}
    \fig{\plotwidth}{Y}{}{fig:results:filtering:y}
    \fig{\plotwidth}{Z}{}{fig:results:filtering:z}

\paragraph{Velocities}
    The velocities, beeing closely coupled with the camera observed position,
    also exhibit good performance in Figures \ref{fig:results:filtering:velx}-\ref{fig:results:filtering:velz}.
    There are shortcomings to the estimation's horizontal precision, although
    this could probably be significantly improved with further filter tuning.
    \fig{\plotwidth}{velX}{}{fig:results:filtering:velx}
    \fig{\plotwidth}{velY}{}{fig:results:filtering:vely}
    \fig{\plotwidth}{velZ}{}{fig:results:filtering:velz}

\paragraph{Orientation, Rotational Velocity and Gyroscope Offset}
    Along with the position, the orientation is estimated from the camera,
    yielding notable precision, as seen in Figures
    \ref{fig:results:filtering:qwb0}-\ref{fig:results:filtering:qwbk}.

    The bias of the gyroscopes is removed during the initialization process.
    Since the time-frame of the tests were far less than the time expeted to
    detect a change in the bias, these should thus be estimated to zero.
    As depicted in Figures \ref{fig:results:filtering:driftroll}-\ref{fig:results:filtering:driftyaw}, they are.

    As noted in Section \ref{ssec:results:modelling:gyroscopes}, the
    filtering of the rotational velocities of the quadrotor body,
    exhibited with its associated covariance in Figures \ref{fig:results:filtering:wroll}-\ref{fig:results:filtering:wyaw},
    correlates very well to the measured results.
    \fig{\plotwidth}{qwb0}{}{fig:results:filtering:qwb0}
    \fig{\plotwidth}{qwbi}{}{fig:results:filtering:qwbi}
    \fig{\plotwidth}{qwbj}{}{fig:results:filtering:qwbj}
    \fig{\plotwidth}{qwbk}{}{fig:results:filtering:qwbk}

    \fig{\plotwidth}{wRoll}{}{fig:results:filtering:wroll}
    \fig{\plotwidth}{wPitch}{}{fig:results:filtering:wpitch}
    \fig{\plotwidth}{wYaw}{}{fig:results:filtering:wyaw}

    \fig{\plotwidth}{driftRoll}{}{fig:results:filtering:driftroll}
    \fig{\plotwidth}{driftPitch}{}{fig:results:filtering:driftpitch}
    \fig{\plotwidth}{driftYaw}{}{fig:results:filtering:driftyaw}

\paragraph{Wind force}
    As the tests were performed inside, the filter was tuned to basically keep the
    wind constant. Thus, it is difficult to come to any conclusions regarding the
    wind impact on the model. They are however, as seen in Figures \ref{fig:results:filtering:windx}-\ref{fig:results:filtering:windy},
    correctly estimated to zero.
    \fig{\plotwidth}{windX}{}{fig:results:filtering:windx}
    \fig{\plotwidth}{windY}{}{fig:results:filtering:windy}
    \fig{\plotwidth}{windZ}{}{fig:results:filtering:windz}

\paragraph{Propeller Velocity}
    As the filter evaluation was performed without the use of the controller,
    the control signal is unavailable. Thus, Eq. \ref{eq:observer:wri2} was used as
    motion model, effectively leaving the estimation of the propeller velocities
    to the measurement update. It is evident, in Figures \ref{fig:results:filtering:wr1}-\ref{fig:results:filtering:wr4}
    that the estimation is active, however it is impossible to validate.
    Ideally, the velocities of the propellers should be measured in flight.
    However, that data is currently unavailable in the development system used for evaluation.
    The estimated velocities are, notably, in a reasonable range,
    increasing the plausability for correctness of Eq. \ref{eq:observer:thrust}, which is
    otherwise hard to verify using the available data.
    \fig{\plotwidth}{wr1}{}{fig:results:filtering:wr1}
    \fig{\plotwidth}{wr2}{}{fig:results:filtering:wr2}
    \fig{\plotwidth}{wr3}{}{fig:results:filtering:wr3}
    \fig{\plotwidth}{wr4}{}{fig:results:filtering:wr4}
