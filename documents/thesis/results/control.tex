\section{Control}
\label{sec:results:control}
    The control algorithm was tested in simulation on the model presented
    of Chapter~\ref{cha:observer} and verified in Section~\ref{sec:results:modelling}.
    The control, very basically tuned, did in simulation exhibit stable and responsive
    properties. The reference track included velocity control in all directions,
    control of yaw rate and finally landing. This corresponds to a wide
    range of the operatations to which a quadrotor is used.

    In Figure~\ref{fig:results:control:referencefollowing}, the reference
    flight is plotted, and as can be seen it starts with a sinusoid velocity.
    After $150$ seconds - simulation time - the landing procedure is initialized and the velocity is
    reduced until after about $180$, when the velocity is below the threshold
    to begin descent. Landing is detected by the shortfall of velocity
    after just over 400 seconds.

    \fig{\plotwidth}{referencefollowing}{Control of the simulated flight was stable, although - beeing untuned - somewhat slow. The actuation of the yaw rate is very weak, which is evident in slow reaction of the yaw rate control.}{fig:results:control:referencefollowing}
