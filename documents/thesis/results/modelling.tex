\section{Modelling}
\label{sec:results:modelling}
    The verification of a complex model is best done in small parts.
    It is however, with the model given in Chapter~\ref{cha:observer},
    difficult to evaluate each equation individually due to the couplings of the model.
    Instead, the verification is performed by evaluating a full test-flight
    with recorded data, using one dataset for calibration and a second for validation.

    For each sensor the predicted and the measured values are compared,
    and the residuals - the difference between the two - are studied and
    fitted to a normal probability density function, \textit{PDF}.

    \subsection{Accelerometers}
        \label{ssec:results:modelling:accelerometers}
        As most of the modelling of Chapter~\ref{cha:observer} concerns
        the forces acting upon the quadrotor, the accelerometers
        provide an interesting measure of the quality of the model.
        It should be noted that parameters were set to reasonable values,
        but no parameter tuning was performed on the motion model,
        leaving the results to be merely directional.
        As depicted in Figure~\ref{fig:results:modelling:accelerometers},
        the model does leave clearly trended residuals, not least in the
        X- and Y-directions where the model does very little.
        From the Figures~\ref{fig:results:modelling:accelerometersresiduals} and \ref{fig:results:modelling:accelerometersnormalfit},
        it can be seen that the model does have beneficial
        effects for the estimation, yielding residuals with zero-close means.
        \fig{\plotwidth}{accelerometers}{Measured and predicted accelerations in the NEDEF system.}{fig:results:modelling:accelerometers}
        \fig{\plotwidth}{accelerometersresiduals}{Residuals between measured and predicted accelerations.}{fig:results:modelling:accelerometersresiduals}
        \fig{\plotwidth}{accelerometersnormalfit}{Accelerometer residuals fitted to a normal PDF. Both theoretical and numerical values have been normalized to a maximum height of one.}{fig:results:modelling:accelerometersnormalfit}

    \subsection{Gyroscopes}
        \label{ssec:results:modelling:gyroscopes}
        It is clear, from Figure~\ref{fig:results:modelling:gyroscopes}, that
        the model well describes the angular velocity of the quadrotor.
        The residuals, described by Figures~\ref{fig:results:modelling:gyroscopesresiduals}
        and \ref{fig:results:modelling:gyroscopesnormalfit}
        display a behaviour which is adequately well described by a random,
        normally distributed, variable, which is expected from the
        standard Kalman filter framework.
        \fig{\plotwidth}{gyroscopes}{Measured and predicted angular rates, in the body-fixed coordinate frame.}{fig:results:modelling:gyroscopes}
        \fig{\plotwidth}{gyroscopesresiduals}{Residuals between measured and predicted angular rates.}{fig:results:modelling:gyroscopesresiduals}
        \fig{\plotwidth}{gyroscopesnormalfit}{Gyroscope residuals fitted to a normal PDF. Both theoretical and numerical values have been normalized to a maximum height of one.}{fig:results:modelling:gyroscopesnormalfit}

    \subsection{Pressure Sensor}
        The pressure sensor is, as clearly seen in Figure~\ref{fig:results:modelling:pressure},
        associated with a great amount of noise. While the residuals,
        Figures~\ref{fig:results:modelling:pressureresiduals} and \ref{fig:results:modelling:pressurenormalfit},
        does not exhibit any obvious trends, noise does spill into
        the positioning with the current tuning, currently adding little contribution
        to the state estimation. This is however likely to be improved by firther filter tuning.
        \fig{\plotwidth}{pressure}{Measured and predicted pressure.}{fig:results:modelling:pressure}
        \fig{\plotwidth}{pressureresiduals}{Residuals between measured and predicted pressure.}{fig:results:modelling:pressureresiduals}
        \fig{\plotwidth}{pressurenormalfit}{Pressure residuals fitted to a normal PDF. Both theoretical and numerical values have been normalized to a maximum height of one.}{fig:results:modelling:pressurenormalfit}

    \subsection{Camera}
        The camera tracking, displayed in Figures~\ref{fig:results:modelling:camera}-\ref{fig:results:modelling:cameranormalfit},
        exhibit very good stability and performance, and significantly add to
        the filter performance. The absolute positioning provided by the
        camera does not only counter the drift in derived observer states,
        but also, not least through its accurate measurements of orientation
        angles exhibited in Figure~\ref{fig:results:modelling:camera}.
        It should be noted that the results, especially in
        Figures~\ref{fig:results:modelling:cameraresiduals}-\ref{fig:results:modelling:cameranormalfit}
        should be scaled approximately by a factor of three to correspond
        to metric quantities.

        When the PTAM library is initialized, it tries to determine the
        ground plane. Thus we are able to verify, in Figure~\ref{fig:results:modelling:camerainitframe}, the initialization process and transformation
        by confirming that the Z-axes are approximately parallell.
        The slight tilting observed in Figure~\ref{fig:results:modelling:camerainitframe}
        is caused by a misplacement of the ground plane in the initialization process.
        Knowing the pose at the initialization allows us to compensate for this in the
        transformation, making the positioning less sensitive for PTAM initialization errors.
        Exact positioning on the moment of initialization is still of importance, however.

        \fig{\plotwidth}{camera}{Measured and predicted angles and positions, in the PTAM coordinate frame. Measures should be multiplied by a scale of approximately $3$ for metric comparison.}{fig:results:modelling:camera}
        \fig{\plotwidth}{cameraresiduals}{Residuals between measured and predicted angles and positions, in the PTAM coordinate frame. Measures should be multiplied by a scale of approximately $3$ for metric comparison.}{fig:results:modelling:cameraresiduals}
        \fig{\plotwidth}{cameranormalfit}{Residuals fitted to a normal PDF. Both theoretical and numerical values have been normalized to a maximum height of one. Measures should be multiplied by a scale of approximately $3$ for metric comparison.}{fig:results:modelling:cameranormalfit}
        \fig{\plotwidth}{camerainitframe}{The PTAM coordinate system is somewhat askew from its theoretical orientation, with the z-axis parallell to the ground. This can also be seen in Figure~\ref{fig:video:ptamtracking}, which is captured from the same dataset.}{fig:results:modelling:camerainitframe}
