\section{Model Verification}
\label{sec:results:modelling}
    The verification of a complex model is best done in small parts.
    It is however, with the model given in Chapter~\ref{cha:observer},
    difficult to evaluate each equation individually due to the couplings of the model.
    Instead, the verification is performed by evaluating a full test-flight
    with recorded data, using one dataset for calibration and a second for validation.

    For each sensor the predicted and the measured values are compared,
    and the residuals - the difference between the two - are studied and
    fitted to a normal probability density function, \textit{PDF}.

    \subsection{Accelerometers}
        \label{ssec:results:modelling:accelerometers}
        As most of the modeling in Chapter~\ref{cha:observer} concerns
        the forces acting upon the quadrotor, the accelerometers
        provide an interesting measure of the quality of the model.
        It should be noted that parameters were set to reasonable values,
        but no parameter tuning was performed on the motion model,
        leaving the results to be merely directional.
        As depicted in Figure~\ref{fig:results:modelling:accelerometers},
        the model does leave clearly trended residuals, not least in the
        X- and Y-directions where the model does very little.
        From the Figures~\ref{fig:results:modelling:accelerometersresiduals} and \ref{fig:results:modelling:accelerometersnormalfit},
        it can be seen that the model does have beneficial
        effects for the estimation, yielding predictions with lesser residuals
        than for instance a constant velocity model would.
        The residuals are also centered around zero due to the compensation
        for sensor bias, which is relevant to reduce drift in the system.
        \fig{\plotwidth}{accelerometers}{Measured and predicted accelerations in the NEDEF system. The motion model does not describe the horizontal motion although does - even in its pre-mature state - explain some of the altitudinal acceleration.}{fig:results:modelling:accelerometers}
        \fig{\plotwidth}{accelerometersresiduals}{Residuals between measured and predicted accelerations. A zero mean is relevant to reduce drift in the system.}{fig:results:modelling:accelerometersresiduals}
        \fig{\plotwidth}{accelerometersnormalfit}{Accelerometer residuals fitted to a normal PDF. Both theoretical and numerical values have been normalized to a maximum height of one.}{fig:results:modelling:accelerometersnormalfit}

    \subsection{Gyroscopes}
        \label{ssec:results:modelling:gyroscopes}
        It is clear, from Figure~\ref{fig:results:modelling:gyroscopes}, that
        the model describes the angular velocity of the quadrotor well.
        The residuals, presented by Figures~\ref{fig:results:modelling:gyroscopesresiduals}
        and \ref{fig:results:modelling:gyroscopesnormalfit}
        display a behavior which is adequately well described by a random,
        normally distributed, variable, which is expected from the
        standard Kalman filter framework.
        \fig{\plotwidth}{gyroscopes}{Measured and predicted angular rates, in the body-fixed coordinate frame.}{fig:results:modelling:gyroscopes}
        \fig{\plotwidth}{gyroscopesresiduals}{Residuals between measured and predicted angular rates.}{fig:results:modelling:gyroscopesresiduals}
        \fig{\plotwidth}{gyroscopesnormalfit}{Gyroscope residuals fitted to a normal PDF. Both theoretical and numerical values have been normalized to a maximum height of one.}{fig:results:modelling:gyroscopesnormalfit}

    \subsection{Pressure Sensor}
        Pressure sensors are associated with a great amount of noise,
        as clearly seen in Figure~\ref{fig:results:modelling:pressure}.
        While the residuals, Figures~\ref{fig:results:modelling:pressureresiduals}
        and \ref{fig:results:modelling:pressurenormalfit}, does not exhibit
        any obvious trends, noise does spill into the positioning with
        the current tuning, currently adding little contribution
        to the state estimation. This may be improved by further filter tuning,
        although indoor use of pressure sensors is, like magnetometers, known to
        be problematic due to air conditioning and other sources of pressure changes.
        To remove the noise induced into the filter by the pressure sensor,
        it was removed from the final filtering evaluation.
        \fig{\plotwidth}{pressure}{Measured and predicted pressure.}{fig:results:modelling:pressure}
        \fig{\plotwidth}{pressureresiduals}{Residuals between measured and predicted pressure.}{fig:results:modelling:pressureresiduals}
        \fig{\plotwidth}{pressurenormalfit}{Pressure residuals fitted to a normal PDF. Both theoretical and numerical values have been normalized to a maximum height of one.}{fig:results:modelling:pressurenormalfit}

    \subsection{Camera Positioning}
        The camera tracking, shown in Figures~\ref{fig:results:modelling:camera}-\ref{fig:results:modelling:cameranormalfit},
        exhibit very good stability and performance, and significantly add to
        the filter performance. The absolute positioning provided by the
        camera counters the drift in derived observer states,
        and provides accurate measurements of orientation angles and position, as
        exhibited in Figure~\ref{fig:results:modelling:camera}.
        Because of potential errors in the world-PTAM transformation, the orientation - which
        is independent of the scale - is often the most reliable of the two quantities.
        It should be noted that the position results, especially in
        Figures~\ref{fig:results:modelling:cameraresiduals}-\ref{fig:results:modelling:cameranormalfit}
        should be scaled by a factor - for this data approximately $3$ - to correspond
        to metric quantities.

        When the PTAM library is initialized, it tries to determine the
        ground plane. Thus we are able to verify, in Figure~\ref{fig:results:modelling:camerainitframe}, the initialization process and transformation
        by confirming that the Z-axes ($Z^{NEDEF}$ and $z^{PTAM}$) are approximately parallel.
        The slight tilting observed in Figure~\ref{fig:results:modelling:camerainitframe}
        is caused by a misplacement of the ground plane in the initialization process,
        which is also visible in Figure~\ref{fig:video:ptamtracking}.
        Since the pose at the time of initialization is known, this tilting
        is compensated for in the calculated transformation, making the positioning
        less sensitive for PTAM initialization errors.
        Exact positioning on the moment of initialization is still of importance, however.

        In Figure~\ref{fig:results:modelling:camera}, a slight bias can be noted.
        This is due to the discrepancies between the PTAM library's ground plane position,
        and the true ground plane.

        \fig{\plotwidth}{camera}{Measured and predicted angles and positions, in the PTAM coordinate frame. Positional measures should be multiplied by a scale of approximately $3$ for metric comparison.}{fig:results:modelling:camera}
        \fig{\plotwidth}{cameraresiduals}{Residuals between measured and predicted angles and positions, in the PTAM coordinate frame. Positional measures should be multiplied by a scale of approximately $3$ for metric comparison.}{fig:results:modelling:cameraresiduals}
        \fig{\plotwidth}{cameranormalfit}{Residuals fitted to a normal PDF. Both theoretical and numerical values have been normalized to a maximum height of one. Positional measures should be multiplied by a scale of approximately $3$ for metric comparison.}{fig:results:modelling:cameranormalfit}
        \fig{\plotwidth}{camerainitframe}{The PTAM coordinate system is somewhat askew from its theoretical orientation, with the z-axis parallel to the ground. This can also be seen in Figure~\ref{fig:video:ptamtracking}, which is captured from the same dataset.}{fig:results:modelling:camerainitframe}
