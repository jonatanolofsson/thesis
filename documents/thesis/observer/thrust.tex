\subsubsection{Rotor thrust}
\label{ssec:observer:thrust}
    Each of the four propellers on the quadrotor induce a torque
    and a thrust vector on the system, proportional to the square of the
    propeller velocity. The rotational velocity of the propeller is directly
    influenced by the controller. It may thus be modelled as a first order system - using the time constant $\tau_{rotor}$ with
    the reference velocity as input, as in Eq. \eqref{eq:observer:wri}.
    For testing purposes, or where the control signal is not available, Eq. \eqref{eq:observer:wri2} may be used instead.
    \begin{equation}
        \label{eq:observer:wri}
        \dot{\omega_{ri}} = \frac{1}{\tau_{rotor}} \left( \omega_{ri} - r_{i} \right)
    \end{equation}
    \begin{equation}
        \label{eq:observer:wri2}
        \dot{\omega_{ri}} = 0
    \end{equation}

    Due to the differences in relative airspeed around
    the rotor blade tip as the blades move either along or against
    the wind-relative velocity, the lifting force on the blade will vary
    around a rotation lap.
    This unbalance in lifting force will cause the blades to lean and the
    direction of the thrust vector to vary with regards to the motions of the quadrotor.

    This phenomenon is called \textit{flapping}, and is discussed
    in e.g. \citep{Pounds_modellingand}. The flapping of the rotors
    and the centrifugal force acting upon the rotating blades
    will result in that the tilted blade trajectories will
    form a cone with the plane to which the rotor axis is normal.
    These motions of the propellers add dynamics to the
    description of the quadrotors motion which must be considered in a deeper analysis.

    It is desirable, for the purpose of this thesis and
    considering computational load, to find a closed-form
    solution to the flapping equations.
    %~ This implies several approximations and restrictions which will be discussed in Section~\ref{sec:discussion:flapping}.
    The resulting flapping angles and their impact on
    the thrust vectors can be described as in equations
    \eqref{eq:observer:thrust}-\eqref{eq:observer:flapping}
    \citep{Pounds_modellingand,prouty1995helicopter,leishman2002principles}.

    The momenta induced by the propeller rotation and thrust
    are described in equations \eqref{eq:observer:torque}-\eqref{eq:observer:thrustmomentum}.
    All equations in this section are given in the body-fixed coordinate system.

    \begin{subequations}
        \begin{equation}
            \label{eq:observer:thrust}
            F_{ri} = C_{T} \rho A_{r} R^{2} \omega_{ri}^{2}\left(
                \begin{array}{c}
                    -\sin{a_{1_{s}i}} \\
                    -\cos{a_{1_{s}i}}\sin{b_{1_{s}i}} \\
                    -\cos{a_{1_{s}i}}\cos{b_{1_{s}i}}
                \end{array}\right)
        \end{equation}

        \begin{equation}
            \label{eq:observer:torque}
            M_{Qi} = -C_{Q} \rho A R^{3} \omega_{ri}|\omega_{ri}|e_{3}^{\text{NED}}
        \end{equation}

        \begin{equation}
            \label{eq:observer:thrustmomentum}
            M_{ri} = F_{ri} \times D_{ri}
        \end{equation}
    \end{subequations}
    The equations for the flapping angles $\left(a_{1_{s}i}, b_{1_{s}i}\right)$ are
    derived in \citep{Pounds_modellingand,prouty1995helicopter,leishman2002principles},
    but are in \eqref{eq:observer:flapping} extended to include the velocity relative to the wind.
    $V_{ri(n)}$ denotes the n'th element of the vector $V_{ri}$.
    \begin{subequations}
        \label{eq:observer:flapping}
        \begin{equation}
            V_{\text{rel}} = V - V_{\text{wind}}
        \end{equation}
        \begin{equation}
            V_{ri} = V_{\text{rel}} + \Omega \times D_{ri} % Velocity relative to the wind //Jonatan
        \end{equation}
        \begin{equation}
            \mu_{ri} = \frac{||V_{ri(1,2)}||}{\omega_{i}R}
        \end{equation}
        \begin{equation}
            \psi_{ri} = \arctan{\frac{V_{ri(2)}}{V_{ri(1)}}}
        \end{equation}
        \begin{equation}
            \label{eq:observer:flapping:ab}
            \begin{array}{rr}\left(
                \begin{array}{c}
                    a_{1_{s}}i \\
                    b_{1_{s}}i
                \end{array} \right)
                = \left(
                \begin{array}{cc}
                    \cos{\psi_{ri}} & -\sin{\psi_{ri}} \\
                    \sin{\psi_{ri}} & \cos{\psi_{ri}}
                \end{array}
                \right) & \left(
                    \begin{array}{c}
                        \frac{1}{1 - \frac{\mu_{ri}^{2}}{2}}\mu_{ri}\left( 4 \theta_{twist} - 2\lambda_{i}\right) \\
                        \frac{1}{1 + \frac{\mu_{ri}^{2}}{2}}\frac{4}{3}\left( \frac{C_{T}}{\sigma}\frac{2}{3}\frac{\mu_{ri}\gamma}{a} + \mu_{ri}\right)
                    \end{array}
                \right) \\
                & +
                \left(
                    \begin{array}{c}
                        \frac{-\frac{16}{\gamma}\left(\frac{\omega_{\theta}}{\omega_{ri}}\right) + \left(\frac{\omega_{\psi}}{\omega_{ri}}\right)}{1 - \frac{\mu_{ri}^{2}}{2}} \\
                        \frac{-\frac{16}{\gamma}\left(\frac{\omega_{\psi}}{\omega_{ri}}\right) + \left(\frac{\omega_{\theta}}{\omega_{ri}}\right)}{1 + \frac{\mu_{ri}^{2}}{2}}
                    \end{array}
                \right)
            \end{array}
        \end{equation}

        %~ \textbf{Note: In the equations \eqref{eq:observer:flapping:ab}, taken from \citep{Pounds_modellingand}, I assume that by ''$a$'', they mean lift curve slope, and by ''$a_{0}$'', they mean the linearization point of a and NOT the coning angle of Prouty pp.468, or the mean coning of Prouty pp.153}

        %~ \citep{Pounds_modellingand}
        %~ \citep{prouty1995helicopter} pp. 165
        %~ \textbf{Not quite finished here.. FIXME: $\theta_{0}$ to table!}
        \begin{equation}
            \lambda_{i} = \mu\alpha_{si} + \frac{v_{1i}}{\omega_{i} R}
        \end{equation}
        \begin{equation}
            v_{1i} = \sqrt{
                -\frac{V_{rel}^{2}}{2} + \sqrt{
                    \left( \frac{V_{rel}^{2}}{2} \right)^{2}
                    + \left( \frac{mg}{2 \rho A_{r}} \right)^{2}
                }
            }
        \end{equation}
        \begin{equation}
            C_{T} = \frac{\sigma a}{4}\left\lbrace
                  \left( \frac{2}{3} + \mu_{ri}^{2} \right) \theta_{0}
                - \left( \frac{1}{2} + \frac{\mu^{2}}{2} \right) \theta_{\text{twist}}
                + \lambda
            \right\rbrace
        \end{equation}
        \begin{equation} % angle between shaft plane and path (rel. to wind) pp.160
            \alpha_{si} = \frac{\pi}{2} - \arccos{ -\frac{V_{rel} \cdot e_{z}}{||V_{rel}||} }
        \end{equation}
        % This is from bouabdallah07design. Same as above _but_ with negative signs... Why?
        %~ \begin{equation}
            %~ C_{T} = \sigma a \left[
                %~ \left(\frac{1}{6} + \frac{1}{4}\mu^{2}\right)\theta_{0}
                %~ - (1 + \mu^{2})\frac{\theta_{twist}}{8}
                %~ - \frac{1}{4}\lambda \right]
        %~ \end{equation}
        \begin{equation}
            C_{Q} = \sigma a \left[
                \frac{1}{8a}\left( 1 + \mu_{ri}^{2} \right) \bar{C_{d}}
                + \lambda\left(
                    \frac{1}{6}\theta_{0}
                    - \frac{1}{8}\theta_{\text{twist}}
                    + \frac{1}{4}\lambda
                    \right)
                \right]
        \end{equation}
    \end{subequations}

    \begin{table}
        \begin{tabularx}{\tablewidth}{|c|c|X|c|}\hline
            \textbf{Symbol} & \textbf{Expression} & \textbf{Description}  & \textbf{Unit} \\\hline
            $a$ & $\frac{\operatorname{d}\!C_{L}}{\operatorname{d}\!\alpha} \approx 2\pi$ & Slope of the lift curve. & $\frac{1}{\text{rad}}$ \\\hline
            $\alpha_{si}$ & - & Propeller angle of attack. & $\text{rad}$ \\\hline
            $A_{r}$ & - & Rotor disk area.   & $\text{m}^{2}$\\\hline
            $c$ & - & Blade chord - the (mean) length between the trailing and leading edge of the propeller.   & $\text{m}$ \\\hline
            $C_{L}$ & - & Coefficient of lift. & $1$ \\\hline
            $C_{T}$ & * & Coefficient of thrust. This is primarily the scaling factor for how the thrust is related to the square of $\omega_{i}$, as per Eq.~\ref{eq:observer:thrust}.  & $1$\\\hline
            $C_{T0}$ & - & Linearization point for thrust coefficient.  & $1$\\\hline
            $C_{Q}$ & * & Torque coefficient. This constant primarily is the scaling factor relating the square of $\omega_{i}$ to the torque from each rotor. & $1$\\\hline
            $\gamma$ & $\frac{\rho a c R^{4}}{I_{b}}$ & $\gamma$ is the Lock Number \citep{leishman2002principles}, described as the ratio between the aerodynamic forces and the inertal forces of the blade.   & $1$ \\ \hline
            $I_{b}$ & - & Rotational inertia of the blade  & $\text{kgm}^{2}$\\\hline
            $\lambda_{i}$ & * & $\lambda_{i}$ denotes the air inflow to the propeller. & $1$ \\\hline
            $R$ & - & Rotor radius.   & $\text{m}$ \\\hline
            $\rho$ & - & Air density.   & $\frac{\text{kg}}{\text{m}^{3}}$ \\\hline
            $\sigma$ & $\frac{\text{blade area}}{\text{disk area}}$ & Disk solidity. & $1$ \\\hline
            $\theta_{0}$ & - & The angle of the propeller at its base, relative to the horizontal disk plane. & $\text{rad}$ \\\hline
            $\theta_{\text{twist}}$ & - & The angle with which the propeller is twisted.  & $\text{rad}$ \\\hline
            $\omega_{\phi},\omega_{\theta},\omega_{\psi}$ & - & The rotational, body-fixed, velocity of the quadrotor. & $\frac{\text{rad}}{\text{s}}$ \\\hline
            $\omega_{ri}$ & - & The rotational velocity of propeller $i$. & $\frac{\text{rad}}{\text{s}}$ \\\hline
            $\mu_{ri}$ & - & The normalized, air-relative, blade tip velocity. & $1$ \\\hline
            %$q$ & - & Pitch rate \\\hline % Can this be replaced by \omega_{\theta}? Just did... =) Body-fixed, so should be good
            %$p$ & - & Roll rate \\\hline % Can this be replaced by \omega_{\psi}? Just did... =) Body-fixed, so should be good
        \end{tabularx}
        \label{tbl:observer:flapping:symbols}
        \caption{Table of symbols used in the flapping equations}
    \end{table}
