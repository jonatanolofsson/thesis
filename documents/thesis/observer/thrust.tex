\subsubsection{Rotor thrust}
\label{ssec:observer:thrust}
    Each of the four propellers on the quadrotor induce a torque
    and a thrust vector on the system.
    Due to the differences in relative airspeed around
    the rotor blade tip as the blades move either along or against
    the wind-relative velocity, the direction of the thrust will vary
    with regards to the motions of the quadrotor.

    This phenomenon is called \textit{flapping}, and as discussed
    in e.g. \citep{Pounds_modellingand}, the flapping of the rotors
    and the centrifugal force acting upon the rotating blades
    will result in that the tilted blade trajectories will
    form a cone with the plane to which the rotor axis is normal.
    These motions of the propellers add important dynamics to the
    description of the quadrotors motion.

    It is desirable, for the purpose of this this thesis and
    considering computational load, to find a closed-form
    solution to the flapping equations. This implies several
    approximations and restrictions which will be discussed in
    \ref{cha:discussion:flapping}.
    The resulting flapping angles and their impact on
    the thrust vectors can be described as in equations
    \eqref{eq:observer:thrust}-\eqref{eq:observer:flapping}
    \citep{Pounds_modellingand,prouty1995helicopter,leishman2002principles}.

    The momentums induced by the propeller rotation and thrust
    are described in equations \eqref{eq:observer:torque}-\eqref{eq:observer:thrustmomentum}.

    \begin{subequations}
        \begin{equation}
            \label{eq:observer:torque}
            Q_{i} = -C_{Q} \rho A R^{3} \omega_{ri}||\omega_{ri}||e_{3}^{NED}
        \end{equation}

        \begin{equation}
            \label{eq:observer:thrustmomentum}
            M_{ri} = F_{ri} \times D_{ri}
        \end{equation}

        \begin{equation}
            \label{eq:observer:thrust}
            F_{ri}^{BF} = C_{T} \rho A_{r} R^{2} \omega_{ri}^{2}\left(
                \begin{array}{c}
                    -\sin{a_{1_{s}i}} \\
                    \cos{a_{1_{s}i}}\sin{b_{1_{s}i}} \\
                    -\cos{a_{1_{s}i}}\cos{b_{1_{s}i}}
                \end{array}\right)
        \end{equation}
    \end{subequations}
    The equations for the flapping angles $\left(a_{1_{s}i}, b_{1_{s}i}\right)$ are
    derived in \citep{Pounds_modellingand,prouty1995helicopter,leishman2002principles},
    but are in \eqref{eq:observer:flapping} extended to include the velocity relative to the wind.
    $V_{ri(n)}$ denotes the n'th element of the vector $V_{ri}$.
    \begin{subequations}
        \label{eq:observer:flapping}
        \begin{equation}
            V_{ri} = V_{rel} + \Omega \times D_{ri} % Velocity relative to the wind //Jonatan
        \end{equation}
        \begin{equation}
            \mu_{ri} = \frac{||V_{ri(1,2)}||}{w_{i}R}
        \end{equation}
        \begin{equation}
            \psi_{ri} = \arctan{\frac{V_{ri(2)}}{V_{ri(1)}}}
        \end{equation}
        \begin{equation}
            \label{eq:observer:flapping:ab}
            \begin{array}{rr}\left(
                \begin{array}{c}
                    a_{1_{s}}i \\
                    b_{1_{s}}i
                \end{array} \right)
                = \left(
                \begin{array}{cc}
                    \cos{\psi_{ri}} & -\sin{\psi_{ri}} \\
                    \sin{\psi_{ri}} & \cos{\psi_{ri}}
                \end{array}
                \right) & \left(
                    \begin{array}{c}
                        \frac{1}{1 - \frac{\mu_{ri}^{2}}{2}}\mu_{ri}\left( 4 \theta_{twist} - 2\lambda_{i}\right) \\
                        \frac{1}{1 + \frac{\mu_{ri}^{2}}{2}}\frac{4}{3}\left( \frac{C_{T}}{\sigma}\frac{2}{3}\frac{\mu_{ri}\gamma}{a} + \mu_{ri}\right)
                    \end{array}
                \right) \\
                & +
                \left(
                    \begin{array}{c}
                        \frac{-\frac{16}{\gamma}\left(\frac{\omega_{\theta}}{\omega_{ri}}\right) + \left(\frac{\omega_{\psi}}{\omega_{ri}}\right)}{1 - \frac{\mu_{ri}^{2}}{2}} \\
                        \frac{-\frac{16}{\gamma}\left(\frac{\omega_{\psi}}{\omega_{ri}}\right) + \left(\frac{\omega_{\theta}}{\omega_{ri}}\right)}{1 + \frac{\mu_{ri}^{2}}{2}}
                    \end{array}
                \right)
            \end{array}
        \end{equation}

        \textbf{Note: In the equations \eqref{eq:observer:flapping:ab}, taken from \citep{Pounds_modellingand}, I assume that by ''$a$'', they mean lift curve slope, and by ''$a_{0}$'', they mean the linearization point of a and NOT the coning angle of Prouty pp.468, or the mean coning of Prouty pp.153}

        \citep{Pounds_modellingand}
        \citep{prouty1995helicopter} pp. 165
        \textbf{Not quite finished here..}
        \begin{equation}
            C_{T} = \frac{\sigma a}{4}\left[\theta_{tip} - \right]
        \end{equation}
    \end{subequations}

    \begin{table}
        \begin{tabularx}{\tablewidth}{|c|c|X|c|}\hline
            \textbf{Symbol} & \textbf{Expression} & \textbf{Description}  & \textbf{Unit} \\\hline
            $a$ & $\frac{\text{d}C_{L}}{\text{d}\alpha}$ & Slope of the lift curve. & $\frac{1}{rad}$ \\\hline
            $a$ & $2\pi$ & Lift curve slope.   & $\frac{1}{rad}$ \\\hline
            $\alpha$ & - & Propeller angle of attack. & $rad$ \\\hline
            $A_{r}$ & - & Rotor disk area.   & $m^{2}$\\\hline
            $c$ & - & Blade chord - the (mean) length between the trailing and leading edge of the propeller.   & $m$ \\\hline
            $C_{L}$ & - & Coefficient of lift. & $1$ \\\hline
            $C_{T0}$ & - & Linearization point for thrust coefficient.  & $1$\\\hline
            $C_{Q}$ & $C_{Q} = C_{P} \approx \frac{C_{T}^{3/2}}{\sqrt{2}}$ \footnote{Leishman, pp.67}& Torque coefficient, approximated using hovering conditions.  & $1$\\\hline
            $\gamma$ & $\frac{\rho a c R^{4}}{I_{b}}$ & $\gamma$ is the Lock Number \citep{leishman2002principles}, described as the ratio between the aerodynamic forces and the inertal forces of the blade.   & $1$ \\ \hline
            $I_{b}$ & - & Rotational inertia of the blade  & $kgm^{2}$\\\hline
            $\lambda_{i}$ & $\sqrt{C_{T}/2}$ & $\lambda_{i}$ denotes the inflow to the propeller, which is approximated by the expression to the left.   & $1$ \\\hline
            $R$ & - & Rotor radius.   & $m$ \\\hline
            $\rho$ & - & Air density.   & $\frac{kg}{m^{3}}$ \\\hline
            $\sigma$ & $\frac{\text{blade area}}{\text{disk area}}$ & Disk solidity. & $1$ \\\hline
            $\theta_{twist}$ & - & The angle with which the propeller is twisted.  & $rad$ \\\hline
            %$q$ & - & Pitch rate \\\hline % Can this be replaced by \omega_{\theta}? Just did... =) Body-fixed, so should be good
            %$p$ & - & Roll rate \\\hline % Can this be replaced by \omega_{\psi}? Just did... =) Body-fixed, so should be good
        \end{tabularx}
        \label{tbl:observer:flapping:symbols}
        \caption{Table of symbols used in the flapping equations}
    \end{table}
