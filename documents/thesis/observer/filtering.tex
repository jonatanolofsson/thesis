\section{The Filtering Problem}
\label{sec:observer:filtering}
    The problem of estimating the state of a system - in this case
    its position, orientation, velocity etc. - is in general filtering
    expressed as the problem of finding the state estimate,
    $\hat{x}$, that in a well defined best way (e.g. with Minimum Mean Square Error, MMSE)
    describes the behaviour of the system.

    The evolution of a system plant is traditionally described by a set of differential equations
    that link the change in the variables to the current state and known inputs, $u$.
    This propagation through time is described by Eq.~\eqref{eq:observer:xdot} ($f_{c}$ denoting the continous case).
    The system is also assumed to be subject to an additive white Gaussian noise $v(t)$ with
    known covariance $Q$.
    This introduces an uncertainty associated with the system's state, which accounts
    for imperfections in the model compared to the physical real-world-system.
    \begin{equation}
        \label{eq:observer:xdot}
        \dot{x}(t) = f_{c}(x(t),u(t),t) + v_{c}(t)
    \end{equation}
    With numeric or analytical solutions, we can obtain the discrete form of
    \eqref{eq:observer:xdot}, where only the sampling times are considered.
    The control signal, $u(t)$, is for instance assumed to be constant in the time interval,
    and we obtain the next predicted state directly, yielding the prediction of $\hat{x}$
    at the time $t$ \textit{given} the information at time
    $t-1$\footnote{Note that we have not yet performed any measurements that provide information about the state at time t.}.
    This motivates the notation used in this thesis - $\hat{x}_{t|t-1}$.
    \begin{equation}
        \label{eq:observer:xnext}
        x_{t|t-1} = f(x_{t-1|t-1},u_{t},t) + v(t)
    \end{equation}

    In the ideal case, a simulation of a prediction $\hat{x}$ would
    with the prediction model in \eqref{eq:observer:xnext} fully describe
    the evolution of the system.
    To be able to provide a good estimate in the realistic case, however,
    we must also feed back measurements given from sensors measuring
    properties of the system.

    These measurements, $y_{t}$, are fed back and fused with the
    prediction using the \textit{innovation}, $\nu$.
    \begin{equation}
        \nu_{t} = y_{t} - \hat{y}_{t}
    \end{equation}
    That is, the difference between the measured value and what would be
    expected in the ideal (predicted) case.
    To account for disturbances affecting the sensors, the measurements
    are also associated with an additive white Gaussian noise $w(t)$, with
    known covariance, $R$.
    \begin{equation}
        \hat{y}_{t} = h(\hat{x}_{t}, u_{t}, t) + w(t)
    \end{equation}

    The innovation is then fused with the prediction to yield a new
    estimation of $x$ given the
    information available as of the time $t$~\citep{gustafsson2010statistical}.
    \begin{equation}
        \hat{x}_{t|t} = \hat{x}_{t|t-1} + K_{t}\nu_{t}
    \end{equation}

    The choice of $K_{t}$ is a balancing between of trusting the model,
    or trusting the measurement. In the Kalman filter framework,
    this balancing is made by tracking and weighing the uncertainties
    introduced by the prediction and the measurement noise.

    \begin{algorithm}[Kalman Filter]
        \label{alg:kf}
        For a linear system, given predictions and
        measurements with known covariances $Q$ and $R$, the optimal
        solution to the filtering problem is given by the forward recursion\footnote{As first suggested by Rudolf E. Kálmán in \cite{kalman1960}.}

        \begin{subequations}
            \textbf{Prediction update}
            \begin{equation}
                \hat{x}_{t|t-1} = A \hat{x}_{t-t|t-1} + B u_{t}
            \end{equation}
            \begin{equation}
                P_{t|t-1} = A P_{t-1|t-1} A^{T} + Q
            \end{equation}
        \end{subequations}
        \begin{subequations}
            \textbf{Measurement update}
            \begin{equation}
                \label{eq:observer:filtering:kalmanK}
                K = P_{t|t-1} H^{T} \left( H P_{t|t-1} H^{T} + R \right)^{-1}
            \end{equation}
            \begin{equation}
                \hat{x}_{t|t} = \hat{x}_{t|t-1} + K \left( y - H \hat{x}_{t|t-1} \right)
            \end{equation}
            \begin{equation}
                P_{t|t} = P_{t|t-1} + K H P_{t|t-1}
            \end{equation}
        \end{subequations}
    \end{algorithm}

    Because of the assumptions on the noise and of the linear property
    of the innovation feedback, the Gaussian property of the noise is preserved
    in the filtering process. The system states can thus ideally
    be considered drawn from a normal distribution.
    \begin{equation}
        x \sim \mathcal{N}\left(\hat{x}, P_{xx}\right)
    \end{equation}

    Conditioned on the state and measurements before time $k$ (kept in the set $\mathcal{Y}^{k}$), the
    covariance of the sample distribution is defined as
    \begin{equation}
        P_{xx}(t|k) = E \left[ \left\lbrace x(t) - \hat{x}_{t|k} \right\rbrace
                               \left\lbrace x(t) - \hat{x}_{t|k} \right\rbrace^{T}
                               | \mathcal{Y}^{k} \right] .
    \end{equation}
    As new measurements are taken, the covariance of the state evolves
    with the state estimate as in Eq. \eqref{eq:observer:filtering:pxx}~\citep{Julier95anewapproach}.
    \begin{equation}
        \label{eq:observer:filtering:pxx}
        P_{xx}(t|t) = P_{xx}(t|t-1) - K_{t}P_{\nu\nu}(t|t-1)K_{t}^{T}
    \end{equation}
    \begin{equation}
        \label{eq:observer:filtering:pnunu}
        P_{\nu\nu}(t|t-1) = P_{yy}(t|t-1) + R(t) .
    \end{equation}
    With known covariances, $K$ can be chosen optimmaly for the linear case as derived in e.g. \citep{gustafsson2010statistical};
    \begin{equation}
        \label{eq:observer:filtering:kalmanK2}
        K_{t} = P_{xy}(t|t-1)P_{\nu\nu}^{-1}(t|t-1).
    \end{equation}
    Note that \eqref{eq:observer:filtering:kalmanK2} is another, albeit equivalent,
    way of calculating \eqref{eq:observer:filtering:kalmanK}, which will be exploited
    in Section \ref{sec:observer:ukf}.

    Although the Kalman filter is optimal in the linear case, no guarantees are given for the non-linear case.
    Several methods exist to give a sub-optimal estimate of the non-linear cases,
    two of which will be studied here.

    One problem with the non-linear case is how to propagate the uncertainty,
    as described by the covariance, through the prediction and measurement models.
    With the assumed Gaussian prior, it is desirable to retain the Gaussian
    property in the posterior estimate, even though this clearly is in violation with the
    non-linear propagation, which generally does not preserve this property.

    As the linear case is simple however;
    \begin{equation}
        P_{xx}(t|t-1) = AP(t|t)A^{T} + Q_{t};
    \end{equation}

    the novel approach is to linearize the system to yield $A$ in every timestep.
    This method is called the Extended Kalman Filter, and is considered the
    de facto standard non-linear filter\cite{Julier04nonlinear}.

    \begin{algorithm}[Extended Kalman Filter]
        \label{alg:ekf}
        The Kalman filter in Algorithm \ref{alg:ekf} is in the Extended Kalman filter extended to the
        non-linear case by straight-forward linearization where nescessary.

        \begin{subequations}
            \textbf{Prediction update}
            \begin{equation}
                \hat{x}_{t|t-1} = f\left( \hat{x}_{t-t|t-1}, u_{t} \right)
            \end{equation}
            \begin{equation}
                P_{t|t-1} = A P_{t-1|t-1} A^{T} + Q
            \end{equation}
        \end{subequations}
        \begin{subequations}\textbf{Measurement update}
            \begin{equation}
                \label{eq:filtering:kalmanK}
                K = P_{t|t-1} H^{T} \left( H P_{t|t-1} H^{T} + R \right)^{-1}
            \end{equation}
            \begin{equation}
                \hat{x}_{t|t} = \hat{x}_{t|t-1} + K \left( y - h(\hat{x}_{t|t-1}) \right)
            \end{equation}
            \begin{equation}
                P_{t|t} = P_{t|t-1} + K H P_{t|t-1},
            \end{equation}
        \end{subequations}
        where
        \begin{equation}
            \begin{array}{cc}
                A = \left.\frac{\partial f(x,u)}{\partial x}\right|_{x = \hat{x}_{t|t}}, & H = \left.\frac{\operatorname{d}\!h(x)}{\operatorname{d}\!x}\right|_{x = \hat{x}_{t|t-1}}.
            \end{array}
        \end{equation}
    \end{algorithm}

    This linearization, some argue\cite{Julier95anewapproach}, fails to capture the finer details
    of highly non-linear systems and may furthermore be tedious to
    calculate, analytically or otherwise. An alternative approach, known as the Unscented Kalman Filter, is therefore
    discussed in Section \ref{sec:observer:ukf}.

