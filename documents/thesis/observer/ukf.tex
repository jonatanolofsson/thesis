\section{The Unscented Kalman Filter}
\label{sec:observer:ukf}
    The basic version of the Unscented Kalman Filter was proposed in \citep{Julier95anewapproach}
    based on the following intuition \citep{Julier95anewapproach}
    \begin{quote}\textit{
        With a fixed number of parameters it should be easier to approximate a Gaussian
        distribution than it is to approximate an arbitrary nonlinear function.
        }
    \end{quote}
    The approach is thus to propagate the uncertainty of the system
    through the non-linear system and fit the results as a Gaussian distribution.
    The propagation is made by simulating the system in the prediction
    model for carefully chosen offsets from the current state called
    \textit{sigma points}, each associated with a weight of importance.
    The selection scheme for these points can vary (and yield other
    types of filters), but a common choice
    is the \textit{Scaled Unscented Transform} (SUT) \citep{vandermerwe:upf}.
    The SUT uses a minimal set of sigma points needed to describe the
    first two moments of the propagated distribution - two for each
    dimension ($n$) of the state vector and one for the mean.
    \begin{align}\nonumber
        \mathbf{\mathcal{X}_{0}} &= \hat{x} & \\\nonumber
        \mathbf{\mathcal{X}_{i}} &= \hat{x} + \left( \sqrt{(n + \lambda) P_{xx}} \right)_{i}
            & i = 1,\cdots,n \\
        \mathbf{\mathcal{X}_{i}} &= \hat{x} - \left( \sqrt{(n + \lambda) P_{xx}} \right)_{i}
            & i = n+1,\cdots,2n
    \end{align}
    \begin{align}\nonumber
        \begin{array}{lr}
        W_{0}^{m} = \frac{\lambda}{n + \lambda} \qquad&
            W_{0}^{c} = \frac{\lambda}{n + \lambda} + (1-\alpha^{2} + \beta){}
        \end{array}\\
        \begin{array}{c}
            W^{m}_{i} =  W^{c}_{i} = \frac{1}{2(n + \lambda)} \qquad i = 1,\cdots,2n{}
        \end{array}
    \end{align}
    \begin{equation}
        \lambda = \alpha^{2}(n + \kappa) - n
    \end{equation}
    The three parameters introduced here, $\alpha$, $\beta$ and $\kappa$
    are summarized in Table~\ref{tbl:observer:ukf::parameters}.
    The term $\left( \sqrt{(n + \lambda) P_{xx}} \right)_{i}$ is used to
    denote the $i$'th column of the matrix square root $\sqrt{(n + \lambda) P_{xx}}$.


    \begin{table}
        \begin{tabularx}{\tablewidth}{c|p{2cm}X}
            \textbf{Variable} & \textbf{Value} & \textbf{Description} \\ \hline
            $\alpha$ & $0 \leq \alpha \leq 1$ (e.g. $0.01$) & Scales the size of the sigma point distribution.
                                                A small $\alpha$ can be used to avoid large non-local non-linearities. \\
            $\beta$  & $2$ &  As discussed in \citep{Julier02thescaled}, $\beta$ affects the weighting of the center point,
                            which will directly influence the magnitude of errors introduced by the fourth and higher
                            order moments. In the strictly Gaussian case, $\beta = 2$ can be shown to be optimal. \\
            $\kappa$ & $0$ &  $\kappa$ is the number of times that the centerpoint is included in the set of sigma points,
                            which will add weight to the centerpoint and scale the distribution of sigma points. \\
        \end{tabularx}
        \label{tbl:observer:ukf::parameters}
        \caption{Description of the parameters used in the SUT.}
    \end{table}

    When the sigma points $\mathbf{\mathcal{X}_{i}}$ have been calculated,
    they are propagated through the non-linear prediction function and
    the resulting mean and covariance can be calculated.
    \begin{equation}
        \mathbf{\mathcal{X}^{+}_{i}} = f(\mathbf{\mathcal{X}_{i}}, u, t) \qquad i = 0,\cdots,2n
    \end{equation}
    \begin{equation}
        \hat{x} = \sum_{i=0}^{2n}W^{m}_{i}\mathbf{\mathcal{X}^{+}_{i}}
    \end{equation}
    \begin{equation}
        P_{xx} = \sum_{i=0}^{2n}W^{c}_{i}
            \left\lbrace \mathbf{\mathcal{X}^{+}_{i}} - \hat{y} \right\rbrace
            \left\lbrace \mathbf{\mathcal{X}^{+}_{i}} - \hat{y} \right\rbrace^{T}
    \end{equation}

    For the measurement update, similar results are obtained, and the equations
    \eqref{eq:observer:ukf:pyy}-\eqref{eq:observer:ukf:pxy} can be connected to
    equations \eqref{eq:observer:filtering:pnunu}-\eqref{eq:observer:filtering:kalmanK2}.
    \begin{equation}
        \mathbf{\mathcal{Y}_{i}} = h(\mathbf{\mathcal{X}_{i}}, u, t) \qquad i = 0,\cdots,2n
    \end{equation}
    \begin{equation}
        \hat{y} = \sum_{i=0}^{2n}W^{m}_{i}\mathbf{\mathcal{Y}_{i}}
    \end{equation}
    \begin{equation}
        \label{eq:observer:ukf:pyy}
        P_{yy} = \sum_{i=0}^{2n}W^{c}_{i}
            \left\lbrace \mathbf{\mathcal{Y}_{i}} - \hat{y} \right\rbrace
            \left\lbrace \mathbf{\mathcal{Y}_{i}} - \hat{y} \right\rbrace^{T}
    \end{equation}
    \begin{equation}
        \label{eq:observer:ukf:pxy}
        P_{xy} = \sum_{i=0}^{2n}W^{c}_{i}
            \left\lbrace \mathbf{\mathcal{X}_{i}} - \hat{x} \right\rbrace
            \left\lbrace \mathbf{\mathcal{Y}_{i}} - \hat{y} \right\rbrace^{T}
    \end{equation}

    As can be seen from the equations in this section, the UKF handles the
    propagation of the probability densities through the model without
    the need for explicit calculation of the Jacobians or Hessians for the system.
    The filtering is based solely on function evaluations of small offsets from the
    expected mean state\footnote{It is not, however, merely a central difference linearization of the functions, as \citep{Julier95anewapproach} notes.},
    be it for the measurement functions, discussed in
    Section~\ref{sec:observer:sensormodels}, or the time update
    prediction function - the motion model.
