\chapter{State Estimation}
\label{cha:observer}
    A central part of automatic control is to know the state of the device you are
    controlling. The system studied in this thesis - the LinkQuad - is in constant
    motion, so determining the up-to-date position is of vital importance.
    This chapter deals with the estimation of the states relevant for positioning
    and controlling the LinkQuad.
    Filter theory and notation is established in Section~\ref{sec:observer:filtering}.

    In this thesis, an Unscented Kalman Filter (UKF) and an Extended Kalman Filter (EKF) both are evaluated.
    These filters both extends the linear Kalman filter theory to the non-linear case.
    The UKF circumvents the linearization used in the EKF in an appealing black-box way,
    albeit it is more sensitive to obscure cases in the physical model,
    as detailed in the discussion in Chapter~\ref{cha:discussion}.
    The theory of the EKF and UKF is treated in
    Sections~\ref{sec:observer:filtering} and \ref{sec:observer:ukf} respectively.

    The motion model of the system is derived and discussed in Section~\ref{sec:observer:motionmodel}.

    As well as beeing modelled, the motions of the system are captured by the on-board sensors.
    A measurement $y$ is related to the motion model by the sensor model $h$;
    \begin{equation}
        y(t) = h(x(t),u(t),t)
    \end{equation}
    The models for the sensors used in this work are discussed in Section~\ref{sec:observer:sensormodels}.

    \section{Filtering}
\label{sec:results:filtering}
The filter implementation was evaluated with recorded data.
Due to model stability issues, as discussed in Chapter \ref{cha:discussion},
the EKF-filter was selected for evaluation.

\subsection{Positioning}
    While the altitude positioning, shown in Figure~\ref{fig:results:filtering:position},
    exhibit disturbances correlated with pressure sensor noise,
    the positioning generally exhibit very good performance. The position state
    is observed in the state-estimation more or less directly by the camera
    and the stability of the camera positioning is thus of course reflected here.

    In the plot of the Z-axial positioning in Figure~\ref{fig:results:filtering:position},
    the estimates from the pre-existing complementary filter -
    moved to share the initial conditions of the Kalman filter - was added for
    comparison. The complementary filter exhibits the problems associated
    with the pressure sensor, while the camera-based positioning is stable
    throughout the test.

    \begin{subfigures}{Positioning in the X- and Y-direction mostly well corresponds to ground truth, thanks to the camera positioning.
    In the Z-axis, the estimation of the pre-existing complementary filter is added for comparison, its starting point adjusted to produce comparable plots.}{fig:results:filtering:position}
        \sfig{\splotwidth}{X}
        \sfig{\splotwidth}{Y}
        \sfig{\splotwidth}{Z}
    \end{subfigures}

\subsection{Velocities}
    The velocities, being closely coupled with the camera observed position,
    also exhibit good performance (Figure~\ref{fig:results:filtering:velocities}).
    There are shortcomings to the estimation's horizontal precision, although
    this could probably be significantly improved with further filter tuning
    and the availability of control signals.
    \begin{subfigures}{Velocity estimates are generally adequate, but with more tuning, the results are likely to improve.}{fig:results:filtering:velocities}
        \sfig{\splotwidth}{velX}
        \sfig{\splotwidth}{velY}
        \sfig{\splotwidth}{velZ}
    \end{subfigures}

\subsection{Orientation, Rotational Velocity and Gyroscope Bias}
    Along with the position, the orientation is estimated from the camera,
    yielding notable precision, as seen in Figure~\ref{fig:results:filtering:qwb}.

    The bias of the gyroscopes is removed during the initialization process.
    Since the time-frame of the tests were far less than the time expected to
    detect a change in the bias, these should thus be estimated to zero,
    as verified in Figure~\ref{fig:results:filtering:drift}.

    As noted in Section~\ref{ssec:results:modelling:gyroscopes}, the
    filtering of the rotational velocities of the quadrotor body,
    exhibited with their associated covariance in Figure~\ref{fig:results:filtering:omega},
    correlates very well to the measurements.
    In Figure~\ref{fig:results:filtering:headings}, the Kalman filter performance is compared
    to the previously existing complementary filter. It can be seen that
    the prediction update of the Kalman filter improves the phase of the filter,
    although they are of comparable performance towards the end of the dataset.

    \begin{subfigures}{The orientation of the quadrotor was estimated with good accuracy.}{fig:results:filtering:qwb}
        \sfig{\sfplotwidth}{qwb0}
        \sfig{\sfplotwidth}{qwbi}
        \sfig{\sfplotwidth}{qwbj}
        \sfig{\sfplotwidth}{qwbk}
    \end{subfigures}

    \begin{subfigures}{The predicted angular velocities corresponds very well to the gyro measurements.}{fig:results:filtering:omega}
        \sfig{\splotwidth}{wRoll}
        \sfig{\splotwidth}{wPitch}
        \sfig{\splotwidth}{wYaw}
    \end{subfigures}

    \begin{subfigures}{The gyroscopes' drift was removed prior to entering the filter, and does not change during the short recording of data.}{fig:results:filtering:drift}
        \sfig{\splotwidth}{driftRoll}
        \sfig{\splotwidth}{driftPitch}
        \sfig{\splotwidth}{driftYaw}
    \end{subfigures}
    \fig{\plotwidth}{headings}{A comparison with the tuned complementary filter show that both filters accurately describe the heading, and although the phase of the Kalman filter is better in the beginning, they are of comparable performance towards the end of the dataset.}{fig:results:filtering:headings}

\subsection{Wind Force}
    As the tests were performed inside, the filter was tuned to basically keep the
    wind constant. Thus, it is difficult to come to any conclusions regarding the
    wind impact on the model. Figure~\ref{fig:results:filtering:wind}
    shows them to be correctly estimated to zero in the collected dataset,
    although in the case with simulated data with wind, shown in Figure~\ref{fig:results:filtering:simwind},
    results are poor, most likely due to that the simulation of the sensors' data
    does not fully cover the non-continuous behavior that is expected in the
    event of touching the ground, for example proper clamping of accelerometer simulations.

    \begin{subfigures}{Wind estimates from recorded test-data.}{fig:results:filtering:wind}
        \sfig{\splotwidth}{windX}
        \sfig{\splotwidth}{windY}
        \sfig{\splotwidth}{windZ}
    \end{subfigures}

    \begin{subfigures}{Wind from simulated test-flight.}{fig:results:filtering:simwind}
        \sfig{\splotwidth}{simwindX}
        \sfig{\splotwidth}{simwindY}
        \sfig{\splotwidth}{simwindZ}
    \end{subfigures}

    \fig{\plotwidth}{windbump}{Even in simulation, the landing was detectable in the vertical wind estimate, although more work is required to properly model and detect the event.}{fig:results:filtering:windbump}

\subsection{Propeller Velocity}
    As the filter evaluation was performed without the use of the controller,
    the control signal is unavailable. Thus, Eq.~\ref{eq:observer:wri2} was used as
    motion model in the filter validation, effectively leaving the estimation of the propeller velocities
    to the measurement update. It is evident, in Figure~\ref{fig:results:filtering:wr}
    that the estimation is active, however it is impossible to validate properly with the available data.
    Ideally, the velocities of the propellers should be measured in flight.
    However, that data is currently unavailable in the development system used for evaluation.
    The estimated velocities are, notably, in a reasonable range,
    increasing the plausibility for correctness of Eq.~\ref{eq:observer:thrust}, which is
    otherwise hard to verify using the available data.

    \begin{subfigures}{Propeller angular rate estimates could not be properly verified with the data available, although do exhibit a reasonable value range given the model parameter settings.}{fig:results:filtering:wr}
        \sfig{\sfplotwidth}{wr1}
        \sfig{\sfplotwidth}{wr2}
        \sfig{\sfplotwidth}{wr3}
        \sfig{\sfplotwidth}{wr4}
    \end{subfigures}

    \section{The Unscented Kalman Filter}
\label{sec:observer:ukf}
    The basic version of the Unscented Kalman Filter was proposed in \citep{Julier95anewapproach}
    based on the following intuition \citep{Julier95anewapproach}
    \begin{quote}\textit{
        With a fixed number of parameters it should be easier to approximate a Gaussian
        distribution than it is to approximate an arbitrary nonlinear function.
        }
    \end{quote}
    The approach is thus to propagate the uncertainty of the system
    through the non-linear system and fit the results as a Gaussian distribution.
    The propagation is made by simulating the system in the prediction
    model for carefully chosen offsets from the current state called
    \textit{sigma points}, each associated with a weight of importance.
    The selection scheme for these points can vary (and yield other
    types of filters), but a common choice
    is the \textit{Scaled Unscented Transform} (SUT) \citep{vandermerwe:upf}.
    The SUT uses a minimal set of sigma points needed to describe the
    first two moments of the propagated distribution - two for each
    dimension ($n$) of the state vector and one for the mean.
    \begin{align}\nonumber
        \mathbf{\mathcal{X}_{0}} &= \hat{x} & \\\nonumber
        \mathbf{\mathcal{X}_{i}} &= \hat{x} + \left( \sqrt{(n + \lambda) P_{xx}} \right)_{i}
            & i = 1,\cdots,n \\
        \mathbf{\mathcal{X}_{i}} &= \hat{x} - \left( \sqrt{(n + \lambda) P_{xx}} \right)_{i}
            & i = n+1,\cdots,2n
    \end{align}
    \begin{align}\nonumber
        \begin{array}{lr}
        W_{0}^{m} = \frac{\lambda}{n + \lambda} \qquad&
            W_{0}^{c} = \frac{\lambda}{n + \lambda} + (1-\alpha^{2} + \beta){}
        \end{array}\\
        \begin{array}{c}
            W^{m}_{i} =  W^{c}_{i} = \frac{1}{2(n + \lambda)} \qquad i = 1,\cdots,2n{}
        \end{array}
    \end{align}
    \begin{equation}
        \lambda = \alpha^{2}(n + \kappa) - n
    \end{equation}
    The three parameters introduced here, $\alpha$, $\beta$ and $\kappa$
    are summarized in Table~\ref{tbl:observer:ukf::parameters}.
    The term $\left( \sqrt{(n + \lambda) P_{xx}} \right)_{i}$ is used to
    denote the $i$'th column of the matrix square root $\sqrt{(n + \lambda) P_{xx}}$.


    \begin{table}
        \begin{tabularx}{\tablewidth}{c|p{2cm}X}
            \textbf{Variable} & \textbf{Value} & \textbf{Description} \\ \hline
            $\alpha$ & $0 \leq \alpha \leq 1$ (e.g. $0.01$) & Scales the size of the sigma point distribution.
                                                A small $\alpha$ can be used to avoid large non-local non-linearities. \\
            $\beta$  & $2$ &  As discussed in \citep{Julier02thescaled}, $\beta$ affects the weighting of the center point,
                            which will directly influence the magnitude of errors introduced by the fourth and higher
                            order moments. In the strictly Gaussian case, $\beta = 2$ can be shown to be optimal. \\
            $\kappa$ & $0$ &  $\kappa$ is the number of times that the centerpoint is included in the set of sigma points,
                            which will add weight to the centerpoint and scale the distribution of sigma points. \\
        \end{tabularx}
        \label{tbl:observer:ukf::parameters}
        \caption{Description of the parameters used in the SUT.}
    \end{table}

    When the sigma points $\mathbf{\mathcal{X}_{i}}$ have been calculated,
    they are propagated through the non-linear prediction function and
    the resulting mean and covariance can be calculated.
    \begin{equation}
        \mathbf{\mathcal{X}^{+}_{i}} = f(\mathbf{\mathcal{X}_{i}}, u, t) \qquad i = 0,\cdots,2n
    \end{equation}
    \begin{equation}
        \hat{x} = \sum_{i=0}^{2n}W^{m}_{i}\mathbf{\mathcal{X}^{+}_{i}}
    \end{equation}
    \begin{equation}
        P_{xx} = \sum_{i=0}^{2n}W^{c}_{i}
            \left\lbrace \mathbf{\mathcal{X}^{+}_{i}} - \hat{y} \right\rbrace
            \left\lbrace \mathbf{\mathcal{X}^{+}_{i}} - \hat{y} \right\rbrace^{T}
    \end{equation}

    For the measurement update, similar results are obtained, and the equations
    \eqref{eq:observer:ukf:pyy}-\eqref{eq:observer:ukf:pxy} can be connected to
    equations \eqref{eq:observer:filtering:pnunu}-\eqref{eq:observer:filtering:kalmanK2}.
    \begin{equation}
        \mathbf{\mathcal{Y}_{i}} = h(\mathbf{\mathcal{X}_{i}}, u, t) \qquad i = 0,\cdots,2n
    \end{equation}
    \begin{equation}
        \hat{y} = \sum_{i=0}^{2n}W^{m}_{i}\mathbf{\mathcal{Y}_{i}}
    \end{equation}
    \begin{equation}
        \label{eq:observer:ukf:pyy}
        P_{yy} = \sum_{i=0}^{2n}W^{c}_{i}
            \left\lbrace \mathbf{\mathcal{Y}_{i}} - \hat{y} \right\rbrace
            \left\lbrace \mathbf{\mathcal{Y}_{i}} - \hat{y} \right\rbrace^{T}
    \end{equation}
    \begin{equation}
        \label{eq:observer:ukf:pxy}
        P_{xy} = \sum_{i=0}^{2n}W^{c}_{i}
            \left\lbrace \mathbf{\mathcal{X}_{i}} - \hat{x} \right\rbrace
            \left\lbrace \mathbf{\mathcal{Y}_{i}} - \hat{y} \right\rbrace^{T}
    \end{equation}

    As can be seen from the equations in this section, the UKF handles the
    propagation of the probability densities through the model without
    the need for explicit calculation of the Jacobians or Hessians for the system.
    The filtering is based solely on function evaluations of small offsets from the
    expected mean state\footnote{It is not, however, merely a central difference linearization of the functions, as \citep{Julier95anewapproach} notes.},
    be it for the measurement functions, discussed in
    Section~\ref{sec:observer:sensormodels}, or the time update
    prediction function - the motion model.

    \section{Motion Model}
\label{sec:observer:motionmodel}
    As the system studied in the filtering problem progresses through time,
    the state estimate can be significantly improved by if a prediction
    is made on what measurements can be expected, and evaluating the plausibility
    of each measurement after how well they correspond to the prediction.
    With assumed Gaussian white noise distributions, this evaluation
    can be done in the probabilistic Kalman framework as presented in Sections
    \ref{sec:observer:filtering}-\ref{sec:observer:ukf}, where the
    probablity estimate of the sensors' measurements are based on the motion
    model's prediction. In this section, a motion model is derived and evaluated.

    \subsection{Reference frames}
        In the model of the quadrotor, there are several frames of reference.
        \begin{description}
            \item[North-East-Down Earth Fixed (NEDEF)] This frame is fixed at
            a given origin in the earth and is considered an inertial frame.
            All states are expressed in this frame of reference unless explicitly stated otherwise.

            \item[North-East-Down (NED)]
            The NED-frame is fixed at the center of gravity of the quadrotor.
            The NED system's $\hat{z}$-axis is aligned with the gravitational axis and
            the $\hat{x}$-axis along the northern axis.
            The $\hat{y}$-axis is chosen to point east to form a right-hand system.

            \item[Body-fixed] The body-fixed coordinate system is fixed in the
            quadrotor with $\hat{x}$-axis in the forward direction and the $z$-axis in the downward
            direction - as depicted in Figure \ref{fig:observer:referenceframes}.

            \item[Propeller fixed] Each of the propellers are associated
            with their own frame of reference, $P_{i}$, which tracks the
            virtual tilting of the thrust vector due to flapping,
            discussed in Section \ref{ssec:observer:thrust}.

            \item[Camera frame] This is the frame which describes the
            location of the camera.

            \item[IMU frame] This is the body-fixed frame in which the IMU measurements are said
            to be done. The origin is thus fixed close to the interial sensors.
        \end{description}

        The conversion between the reference frames are characterized by a
        transformation including translation and a three-dimensional rotation.
        Both the origin of the body-centered reference frames
        - the quadrotor's position - and the rotation of the body-fixed
        system are stored as system states.

        The centers of each of the propeller fixed coordinate systems
        are parametrized on the height $h$ and distance $d$ from the center of
        gravity as follows
        \begin{align}
            D_{0} &= (d, 0, h)^{BF} \\
            D_{1} &= (0, -d, h)^{BF} \\
            D_{2} &= (-d, 0, h)^{BF} \\
            D_{3} &= (0, d, h)^{BF}
        \end{align}

        \paragraph{Notation:}
        In the following sections, vectors and points in e.g. the NED
        coordinate systems are denoted $x^{NED}$.
        Rotation described by unit quaternions are denoted $R(q)$ for
        the quaternion $q$, corresponding to the matrix rotation \citep{kuipers2002quaternions} given by
        \begin{equation}
            \left(
            \begin{array}{cccc}
                q_{1}^{2} + q_{i}^{2} - q_{j}^{2} - q_{k}^{2}   & 2q_{i}q_{j}+2q_{1}q_{k}                       & 2q_{i}q_{k} - 2q_{1}q_{j} \\
                2q_{i}q_{j} - 2q_{1}q_{k}                       & q_{1}^{2} - q_{i}^{2} + q_{j}^{2} - q_{k}^{2} & 2q_{j}q_{k} + 2q_{1}q_{i} \\
                2q_{i}q_{k} + 2q_{1}q_{j}                       & 2q_{j}q_{k} - 2q_{1}q_{i}                     & q_{1}^{2} - q_{i}^{2} - q_{j}^{2} + q_{k}^{2}
            \end{array}
            \right)
        \end{equation}
    \subsection{Kinematics}
        The motions of the quadrotor are described by the following relations \citep{Pounds_modellingand}: % more citations needed
        \begin{subequations}
            \label{eq:observer:kinematicsc}
            \begin{equation}
                \label{eq:observer:position}
                \dot{\xi} = R(q^{c})V
            \end{equation}
            \begin{equation}
                \label{eq:observer:quaternionsc}
                \left(\begin{array}{c}
                    \dot{q}_{1}^{c} \\
                    \dot{q}_{i}^{c} \\
                    \dot{q}_{j}^{c} \\
                    \dot{q}_{k}^{c}
                \end{array}\right) = -\frac{1}{2}\left(\begin{array}{cccc}
                0 & -\omega_{x} & -\omega_{y} & -\omega_{z} \\
                \omega_{x} & 0 & -\omega_{z} & \omega_{y} \\
                \omega_{y} & \omega_{z} & 0 & -\omega_{x} \\
                \omega_{z} & -\omega_{y} & \omega_{x} & 0
                \end{array}\right)\left(\begin{array}{c}
                q_{1}^{c} \\
                q_{i}^{c} \\
                q_{j}^{c} \\
                q_{k}^{c}
                \end{array}\right)
            \end{equation}
        \end{subequations}

        In practice, a normalization step also has to be added to account for
        the unit length constraint on rotation quaternions.


    \subsection{Dynamics}
        The motions of the quadrotor can be fully mathematically explained by the
        forces and moments acting on the vehicle. Using the rigid-body assumtion,
        Eulers' extension of Newton's laws of motion for the
        quadrotor's center of gravity, $\mathcal{G}$, yields
        \begin{subequations}
            \begin{equation}
                \dot{V} = a_{\mathcal{G}} = \frac{1}{m}\sum F
            \end{equation}
            \begin{equation}
                \dot{\omega} = I_{\mathcal{G}}^{-1}\sum M_{\mathcal{G}}
            \end{equation}
        \end{subequations}

        The main forces acting upon the quadrotor are the effects of three different components
        \begin{itemize}
            \item[$\sum_{i=0}^{3}F_{ri}$] Rotor thrust,
            \item[$F_{g}$] Gravity,
            \item[$F_{wind}$] Wind.
        \end{itemize}

        Of these, the gravity is trivially described as
        \begin{equation}
            F_{g} = mg\cdot e_{3}^{NED}
        \end{equation}

        The following sections will describe the rotor thrust and wind forces
        respectively.

        \subsubsection{Rotor thrust}
\label{ssec:observer:thrust}
    Each of the four propellers on the quadrotor induce a torque
    and a thrust vector on the system, proportional to the square of the
    propeller velocity. The rotational velocity of the propeller is directly
    influenced by the controller. It may thus be modelled as a first order system - using the time constant $\tau_{rotor}$ with
    the reference velocity as input, as in Eq. \eqref{eq:observer:wri}.
    For testing purposes, or where the control signal is not available, Eq. \eqref{eq:observer:wri2} may be used instead.
    \begin{equation}
        \label{eq:observer:wri}
        \dot{\omega_{ri}} = \frac{1}{\tau_{rotor}} \left( \omega_{ri} - r_{i} \right)
    \end{equation}
    \begin{equation}
        \label{eq:observer:wri2}
        \dot{\omega_{ri}} = 0
    \end{equation}

    Due to the differences in relative airspeed around
    the rotor blade tip as the blades move either along or against
    the wind-relative velocity, the lifting force on the blade will vary
    around a rotation lap.
    This unbalance in lifting force will cause the blades to lean and the
    direction of the thrust vector to vary with regards to the motions of the quadrotor.

    This phenomenon is called \textit{flapping}, and is discussed
    in e.g. \citep{Pounds_modellingand}. The flapping of the rotors
    and the centrifugal force acting upon the rotating blades
    will result in that the tilted blade trajectories will
    form a cone with the plane to which the rotor axis is normal.
    These motions of the propellers add dynamics to the
    description of the quadrotors motion which must be considered in a deeper analysis.

    It is desirable, for the purpose of this thesis and
    considering computational load, to find a closed-form
    solution to the flapping equations.
    %~ This implies several approximations and restrictions which will be discussed in Section~\ref{sec:discussion:flapping}.
    The resulting flapping angles and their impact on
    the thrust vectors can be described as in equations
    \eqref{eq:observer:thrust}-\eqref{eq:observer:flapping}
    \citep{Pounds_modellingand,prouty1995helicopter,leishman2002principles}.

    The momenta induced by the propeller rotation and thrust
    are described in equations \eqref{eq:observer:torque}-\eqref{eq:observer:thrustmomentum}.
    All equations in this section are given in the body-fixed coordinate system.

    \begin{subequations}
        \begin{equation}
            \label{eq:observer:thrust}
            F_{ri} = C_{T} \rho A_{r} R^{2} \omega_{ri}^{2}\left(
                \begin{array}{c}
                    -\sin{a_{1_{s}i}} \\
                    -\cos{a_{1_{s}i}}\sin{b_{1_{s}i}} \\
                    -\cos{a_{1_{s}i}}\cos{b_{1_{s}i}}
                \end{array}\right)
        \end{equation}

        \begin{equation}
            \label{eq:observer:torque}
            M_{Qi} = -C_{Q} \rho A R^{3} \omega_{ri}|\omega_{ri}|e_{3}^{\text{NED}}
        \end{equation}

        \begin{equation}
            \label{eq:observer:thrustmomentum}
            M_{ri} = F_{ri} \times D_{ri}
        \end{equation}
    \end{subequations}
    The equations for the flapping angles $\left(a_{1_{s}i}, b_{1_{s}i}\right)$ are
    derived in \citep{Pounds_modellingand,prouty1995helicopter,leishman2002principles},
    but are in \eqref{eq:observer:flapping} extended to include the velocity relative to the wind.
    $V_{ri(n)}$ denotes the n'th element of the vector $V_{ri}$.
    \begin{subequations}
        \label{eq:observer:flapping}
        \begin{equation}
            V_{\text{rel}} = V - V_{\text{wind}}
        \end{equation}
        \begin{equation}
            V_{ri} = V_{\text{rel}} + \Omega \times D_{ri} % Velocity relative to the wind //Jonatan
        \end{equation}
        \begin{equation}
            \mu_{ri} = \frac{||V_{ri(1,2)}||}{\omega_{i}R}
        \end{equation}
        \begin{equation}
            \psi_{ri} = \arctan{\frac{V_{ri(2)}}{V_{ri(1)}}}
        \end{equation}
        \begin{equation}
            \label{eq:observer:flapping:ab}
            \begin{array}{rr}\left(
                \begin{array}{c}
                    a_{1_{s}}i \\
                    b_{1_{s}}i
                \end{array} \right)
                = \left(
                \begin{array}{cc}
                    \cos{\psi_{ri}} & -\sin{\psi_{ri}} \\
                    \sin{\psi_{ri}} & \cos{\psi_{ri}}
                \end{array}
                \right) & \left(
                    \begin{array}{c}
                        \frac{1}{1 - \frac{\mu_{ri}^{2}}{2}}\mu_{ri}\left( 4 \theta_{twist} - 2\lambda_{i}\right) \\
                        \frac{1}{1 + \frac{\mu_{ri}^{2}}{2}}\frac{4}{3}\left( \frac{C_{T}}{\sigma}\frac{2}{3}\frac{\mu_{ri}\gamma}{a} + \mu_{ri}\right)
                    \end{array}
                \right) \\
                & +
                \left(
                    \begin{array}{c}
                        \frac{-\frac{16}{\gamma}\left(\frac{\omega_{\theta}}{\omega_{ri}}\right) + \left(\frac{\omega_{\psi}}{\omega_{ri}}\right)}{1 - \frac{\mu_{ri}^{2}}{2}} \\
                        \frac{-\frac{16}{\gamma}\left(\frac{\omega_{\psi}}{\omega_{ri}}\right) + \left(\frac{\omega_{\theta}}{\omega_{ri}}\right)}{1 + \frac{\mu_{ri}^{2}}{2}}
                    \end{array}
                \right)
            \end{array}
        \end{equation}

        %~ \textbf{Note: In the equations \eqref{eq:observer:flapping:ab}, taken from \citep{Pounds_modellingand}, I assume that by ''$a$'', they mean lift curve slope, and by ''$a_{0}$'', they mean the linearization point of a and NOT the coning angle of Prouty pp.468, or the mean coning of Prouty pp.153}

        %~ \citep{Pounds_modellingand}
        %~ \citep{prouty1995helicopter} pp. 165
        %~ \textbf{Not quite finished here.. FIXME: $\theta_{0}$ to table!}
        \begin{equation}
            \lambda_{i} = \mu\alpha_{si} + \frac{v_{1i}}{\omega_{i} R}
        \end{equation}
        \begin{equation}
            v_{1i} = \sqrt{
                -\frac{V_{rel}^{2}}{2} + \sqrt{
                    \left( \frac{V_{rel}^{2}}{2} \right)^{2}
                    + \left( \frac{mg}{2 \rho A_{r}} \right)^{2}
                }
            }
        \end{equation}
        \begin{equation}
            C_{T} = \frac{\sigma a}{4}\left\lbrace
                  \left( \frac{2}{3} + \mu_{ri}^{2} \right) \theta_{0}
                - \left( \frac{1}{2} + \frac{\mu^{2}}{2} \right) \theta_{\text{twist}}
                + \lambda
            \right\rbrace
        \end{equation}
        \begin{equation} % angle between shaft plane and path (rel. to wind) pp.160
            \alpha_{si} = \frac{\pi}{2} - \arccos{ -\frac{V_{rel} \cdot e_{z}}{||V_{rel}||} }
        \end{equation}
        % This is from bouabdallah07design. Same as above _but_ with negative signs... Why?
        %~ \begin{equation}
            %~ C_{T} = \sigma a \left[
                %~ \left(\frac{1}{6} + \frac{1}{4}\mu^{2}\right)\theta_{0}
                %~ - (1 + \mu^{2})\frac{\theta_{twist}}{8}
                %~ - \frac{1}{4}\lambda \right]
        %~ \end{equation}
        \begin{equation}
            C_{Q} = \sigma a \left[
                \frac{1}{8a}\left( 1 + \mu_{ri}^{2} \right) \bar{C_{d}}
                + \lambda\left(
                    \frac{1}{6}\theta_{0}
                    - \frac{1}{8}\theta_{\text{twist}}
                    + \frac{1}{4}\lambda
                    \right)
                \right]
        \end{equation}
    \end{subequations}

    \begin{table}
        \begin{tabularx}{\tablewidth}{|c|c|X|c|}\hline
            \textbf{Symbol} & \textbf{Expression} & \textbf{Description}  & \textbf{Unit} \\\hline
            $a$ & $\frac{\operatorname{d}\!C_{L}}{\operatorname{d}\!\alpha} \approx 2\pi$ & Slope of the lift curve. & $\frac{1}{\text{rad}}$ \\\hline
            $\alpha_{si}$ & - & Propeller angle of attack. & $\text{rad}$ \\\hline
            $A_{r}$ & - & Rotor disk area.   & $\text{m}^{2}$\\\hline
            $c$ & - & Blade chord - the (mean) length between the trailing and leading edge of the propeller.   & $\text{m}$ \\\hline
            $C_{L}$ & - & Coefficient of lift. & $1$ \\\hline
            $C_{T}$ & * & Coefficient of thrust. This is primarily the scaling factor for how the thrust is related to the square of $\omega_{i}$, as per Eq.~\ref{eq:observer:thrust}.  & $1$\\\hline
            $C_{T0}$ & - & Linearization point for thrust coefficient.  & $1$\\\hline
            $C_{Q}$ & * & Torque coefficient. This constant primarily is the scaling factor relating the square of $\omega_{i}$ to the torque from each rotor. & $1$\\\hline
            $\gamma$ & $\frac{\rho a c R^{4}}{I_{b}}$ & $\gamma$ is the Lock Number \citep{leishman2002principles}, described as the ratio between the aerodynamic forces and the inertal forces of the blade.   & $1$ \\ \hline
            $I_{b}$ & - & Rotational inertia of the blade  & $\text{kgm}^{2}$\\\hline
            $\lambda_{i}$ & * & $\lambda_{i}$ denotes the air inflow to the propeller. & $1$ \\\hline
            $R$ & - & Rotor radius.   & $\text{m}$ \\\hline
            $\rho$ & - & Air density.   & $\frac{\text{kg}}{\text{m}^{3}}$ \\\hline
            $\sigma$ & $\frac{\text{blade area}}{\text{disk area}}$ & Disk solidity. & $1$ \\\hline
            $\theta_{0}$ & - & The angle of the propeller at its base, relative to the horizontal disk plane. & $\text{rad}$ \\\hline
            $\theta_{\text{twist}}$ & - & The angle with which the propeller is twisted.  & $\text{rad}$ \\\hline
            $\omega_{\phi},\omega_{\theta},\omega_{\psi}$ & - & The rotational, body-fixed, velocity of the quadrotor. & $\frac{\text{rad}}{\text{s}}$ \\\hline
            $\omega_{ri}$ & - & The rotational velocity of propeller $i$. & $\frac{\text{rad}}{\text{s}}$ \\\hline
            $\mu_{ri}$ & - & The normalized, air-relative, blade tip velocity. & $1$ \\\hline
            %$q$ & - & Pitch rate \\\hline % Can this be replaced by \omega_{\theta}? Just did... =) Body-fixed, so should be good
            %$p$ & - & Roll rate \\\hline % Can this be replaced by \omega_{\psi}? Just did... =) Body-fixed, so should be good
        \end{tabularx}
        \label{tbl:observer:flapping:symbols}
        \caption{Table of symbols used in the flapping equations}
    \end{table}

        \subsubsection{Wind}
    For describing the wind's impact on the quadrotor motion,
    a simple wind model is applied where the wind is modeled with
    a static velocity that imposes forces and moments on the quadrotor.
    The wind velocity vector is estimated by the observer and may thus still vary
    in its estimation through the measurement update.
    The wind velocities in the filter are given in the NEDEF reference frame.

    The wind drag force is calculated using equation \eqref{eq:observer:wind:dragforce},
    whereas the moments are given by equations \eqref{eq:observer:wind:moments}.
    In this thesis, the moments acting on the quadrotor body (as opposed to the rotors)
    are neglected or described by moments imposed by the wind acting on the rotors.

    \begin{subequations}
    \label{eq:observer:wind:dragforce}
        \begin{align}
            F_{\text{wind}} &= F_{wind,body} + \sum_{i=0}^{3} F_{wind,ri} \\
%
            F_{\text{wind,body}} &= -\frac{1}{2} C_{D} \rho A V_{\text{rel}} ||V_{\text{rel}}|| \\
%
            F^{BF}_{\text{wind},ri} &= -\frac{1}{2} \rho C_{D,r} \sigma A_{r} (V_{ri} \cdot e_{P_{ri}3}^{BF}) ||V_{ri} \cdot e_{P_{ri}3}^{BF}|| e_{P_{ri}3}^{BF}
        \end{align}
    \end{subequations}

    \begin{subequations}
    \label{eq:observer:wind:moments}
        \begin{align}
            M_{\text{wind}} = M_{\text{wind,body}} + \sum_{i=0}^{3}M_{\text{wind},ri} \\
%
            M_{\text{wind,body}} \approx 0 \\ % Could perhaps be better approximated by constant*\sum_{i=0}^{3}M_{wind,ri}, for small constant, modeling the wind's effect on the four arms
%
            M_{\text{wind},ri} = D_{ri}^{BF} \times F^{BF}_{\text{wind},ri}
        \end{align}
    \end{subequations}

    The wind model applied in this thesis is a decaying model that tends
    towards zero if no measurements tell otherwise.
    This decaying model is presented in Eq. \eqref{eq:observer:wind:wind} ($\epsilon$ being a small number).
    \begin{equation}
        \label{eq:observer:wind:wind}
        \dot{V}_{\text{wind}} = -\epsilon \cdot V_{\text{wind}}
    \end{equation}


    \begin{table}
        \begin{tabularx}{\tablewidth}{|c|c|X|}\hline
            \textbf{Symbol} & \textbf{Expression} & \textbf{Description} \\\hline
            $A$ & - & 3x3 matrix describing the area of the quadrotor, excluding the rotors. \\\hline
            $C_{D}$ & - & 3x3 matrix describing the drag coefficients of the quadrotor. \\\hline
            $C_{Dr}$ & - & Propeller's coefficient of drag. \\\hline
        \end{tabularx}
        \label{tbl:observer:wind:symbols}
        \caption{Table of symbols used in the wind equations}
    \end{table}


    \section{Sensor Models}
\label{sec:observer:sensormodels}
    This section relates the estimated state of the quadrotor to the expected sensor measurements, $\hat{y}$.

    In UAV state estimation, it is common to include a GPS sensor, providing world-fixed measurements,
    to prevent drift in the filtering process. In this thesis, the GPS is replaced with a camera,
    which is discussed in detail in Section~\ref{ssec:observer:sensormodels:camera}, following
    a description of the accelerometers, gyroscopes, magnetometers and the pressure sensor.

    In the measurement equations in this section, a zero-mean Gaussian term $e$ with known covariance is added
    to account for measurement noise. The Gaussian assumption may in some
    cases be severely inappropriate, but the Kalman filter framework requires its use.

    \subsection{Accelerometers}
        The accelerometers, as the name suggests, provides measurements of the
        accelerations of the sensor. In general, this does not directly correspond
        to the accelerations of the mathematical center of gravity used
        as center of the measured vehicle. This motivates a correction
        for angular acceleration and velocity with regards to the sensor's
        relative position from the center of gravity, $r_{\text{acc}/\mathcal{G}}$.
        \begin{equation}
            \hat{y}_{\text{acc}} = a_{\mathcal{G}} + \dot{\omega} \times r_{\text{acc}/\mathcal{G}} + \omega \times \left( \omega \times r_{\text{acc}/\mathcal{G}} \right) + e_{\text{acc}}
        \end{equation}

    \subsection{Gyroscopes}
        Gyroscopes, or rate gyroscopes specifically, measure the angular velocity
        of the sensor. Unlike acceleration, the angular rate is theoretically
        invariant of the relative position of the sensor and the center of gravity.
        However, gyroscope measurements are associated with a bias which may
        change over time. This bias term may be introduced as a state variable in the observer,
        modeled constant as in Eq.~\eqref{eq:filter:sensormodels:accelerometertime},
        leaving its adjustment to the observer's measurement update.
        \begin{equation}
            y_{\text{gyro}} = \omega + b + e_{\text{gyro}}
        \end{equation}
        \begin{equation}
            \label{eq:filter:sensormodels:accelerometertime}
            \dot{b} = 0 + e_{\text{bias}}
        \end{equation}

    \subsection{Magnetometers}
        Capable of sensing magnetic fields, the magnetometers can be used to
        sense the direction of the Earth's magnetic field and, from knowing
        the field at the current location, estimate the orientation of a vehicle.
        \begin{equation}
            y = R(q) m^{e} + e_{m}
        \end{equation}
        The Earth's magnetic field can initialized at startup, or approximated using the World Magnetic Model\cite{wmm2010},
        which for Linköping, Sweden, is given in Eq.~\eqref{eq:observer:sensormodels:magneticfield}.
        \begin{equation}
            \label{eq:observer:sensormodels:magneticfield}
            m^{e} = \left[\left(\begin{array}{ccc}
                15.7 & 1.11 & 48.4
            \end{array}\right)^{T}\right]^{NEDEF} \mu T
        \end{equation}

        Magnetometer measurements are, however, very sensitive to disturbances, and in indoor
        flight, measurements are often useless due to electrical wiring, lighting etc.
        Thus, the magnetometers were not used for the state estimation in this thesis.

    \subsection{Pressure Sensor}
        The barometric pressure, $p$, can be related to altitude using Eq.~\eqref{eq:sensormodels:pressure} \cite{physicshandbook}.
        The pressure sensor, as shown in the validation, is inherently
        noisy and especially so in an indoor environment where air conditioning
        causes disturbances in the relatively small - and thus pressure sensitive -
        environments that are available indoors.
        The pressure sensor is also affected by propeller turbulence.
        \begin{equation}
            \label{eq:sensormodels:pressure}
            p = p_{0} \left( 1 - \frac{L \cdot h}{T_{0}} \right)^{\frac{g \cdot M}{R\cdot L}} + e_{p}
        \end{equation}
        \begin{table}
            \begin{tabularx}{\tablewidth}{|c|X|c|c|}\hline
                \textbf{Parameter} & \textbf{Description} & \textbf{Value} & \textbf{Unit} \\\hline
                $L$       & Temperature lapse rate.            & $0.0065$ & $\frac{\text{K}}{\text{m}}$ \\\hline
                $M$       & Molar mass of dry air.             & $0.0289644$ & $\frac{\text{kg}}{\text{mol}}$ \\\hline
                $p_{0}$   & Atmospheric pressure at sea level. & $101325$ & $\text{Pa}$ \\\hline
                $R$       & Universal gas constant.            & $8.31447$ & $\frac{\text{J}}{\text{mol} \cdot \text{K}}$ \\\hline
                $T_{0}$   & Standard temperature at sea level. & $288.15$ & $\text{K}$ \\\hline
            \end{tabularx}
            \label{tos:pressuresensor}
            \caption{Table of Symbols used in the pressure equation, Eq.~\eqref{eq:sensormodels:pressure}}.
        \end{table}

    \subsection{Camera}
\label{ssec:observer:sensormodels:camera}
    To estimate the position of the camera using the captured images,
    the PTAM-library is used.
    Because the main application of the PTAM library is reprojection of
    augmented reality into the image, consistency between a metric world-fixed
    coordinate frame (such as the NEDEF-system used on the LinkQuad), and the
    internally used coordinate system is not of importance - and thus not implemented in the PTAM positioning.

    The measurements from the camera consists of the transform from the PTAM ``world''-coordinates
    to the camera lens, in terms of
    \begin{itemize}
        \item translation, $X^{\text{PTAM}}$,
        \item and orientation, $q^{PTAM,c}$.
    \end{itemize}
    However, since the quite arbtirary\citep{klein07parallel} coordinate system
    of PTAM is neither of the same scale nor aligned with the quadrotor coordinate system,
    we need to estimate the affine transformation between the two in order to get useful results.

    The transformation is characterized by
    \begin{itemize}
        \item a translation T to the origin, $\varnothing_{\text{PTAM}}$,
        \item a rotation R by the quaternion $q^{Pw}$,
        \item and a scaling S by a factor $s$.
    \end{itemize}
    %~ \begin{equation}
        %~ x_{\text{world}} = \underbrace{R(q^{wb})}_{quadrotor orientation} * \underbrace{T(\varnothing_{\text{cam}}) R(q^{bc}) S(s)}_{quadrotor to PTAM transform} x_{\text{cam}}
    %~ \end{equation}
    These are collected to a single transformation in Eq.~\ref{eq:observer:sensormodels:camera:transformation},
    forming the full transformation from the global NEDEF system to the
    PTAM coordinate frame.
    \begin{equation}
        \label{eq:observer:sensormodels:camera:transformation}
        x^{\text{PTAM}} = \underbrace{S(s) R(q^{Pw}) T(-\varnothing_{\text{PTAM}})}_{\triangleq \mathcal{J}^{Pw}, \text{transformation from camera to PTAM}}
         x^{\text{NEDEF}}
    \end{equation}

    E.g. \citep{hayashi2010} studies this problem in the offline case,
    whereas the method used in this thesis extends the idea to the on-line
    case where no ground truth is available. This is performed by using the
    first measurement to construct an initial guess which then is filtered
    through time using the observer.

    While PTAM exhibit very stable positioning, it has a tendency to move
    its origin due to association errors. To provide stable position
    measurements, we need to detect these movements and adjust the
    camera transformation accordingly. Initialization and tracking is dealt with in
    Section~\ref{sssec:observer:sensormodels:camera:initialization} and \ref{sssec:observer:sensormodels:camera:refinement}
    respectively, whereas the teleportation problem is discussed in
    Section~\ref{sssec:observer:sensormodels:camera:teleportation}.

    \subsubsection{Initialization}
        \label{sssec:observer:sensormodels:camera:initialization}
        When the first camera measurement arrives, there is a need to construct a
        first guess of the transform. Since the PTAM initialization places
        the origin at what it considers the ground level, the most informed
        guess we can do without any information about the environment is
        to assume that this is a horizontal plane at zero height.

        First, however, the orientation of the PTAM coordinate system is calculated
        from the estimated quadrotor orientation and the measurement in the
        PTAM coordinate frame;
        \begin{equation}
            \label{eq:observer:sensormodels:camera:qpw}
            q^{Pw} = q^{PTAM,c} q^{cb} q^{bw}.
        \end{equation}

        $q^{bc}$, the inverse of $q^{cb}$ of Equation~\ref{eq:observer:sensormodels:camera:qpw},
        describes the rotation from camera coordinates to body-fixed coordinates, taking
        into account the differing definitions between the PTAM library
        camera coordinate system and that used in this thesis.
        With known camera pitch and yaw - $\Theta_{c}$ and $\Psi_{c}$ respectively -
        this  corresponds to four consecutive rotations, given in
        Eq.~\eqref{eq:observer:sensormodels:qbc} as rotations around gives axes.

        %~ Quaternion<double> qbc(
            %~ AngleAxis<double>(config["camera"]["tilt"].as<double>(), Vector3d::UnitY())
            %~ * AngleAxis<double>(config["camera"]["yaw"].as<double>(), Vector3d::UnitZ())
            %~ * AngleAxis<double>(M_PI_2, Vector3d::UnitZ())
            %~ * AngleAxis<double>(M_PI_2, Vector3d::UnitX()));
        \begin{equation}
        \label{eq:observer:sensormodels:qbc}
            q^{bc} = \text{rot}(\Theta_{c}, \hat{y}) * \text{rot}(\Psi_{c}, \hat{z}) * \text{rot}(\pi/2, \hat{z}) * \text{rot}(\pi/2, \hat{x})
        \end{equation}

        To determine the distance to this plane according to the current estimation,
        Equation \eqref{eq:observer:sensormodels:camera:origin} is solved for $\lambda$
        in accordance with Eq.~\ref{eq:observer:sensormodels:camera:lambda}.
        \begin{equation}
            \label{eq:observer:sensormodels:camera:origin}
            \left\lbrace
            \begin{array}{ll}
                \varnothing_{\text{PTAM}} &= \xi + R(q^{wb}) r_{\text{camera}/\mathcal{G}} + \lambda R(q^{wP}) \frac{X^{PTAM}}{|X^{PTAM}|} \\
                \varnothing_{\text{PTAM}} \cdot \hat{z} &= 0
            \end{array}\right.
        \end{equation}

        \begin{equation}
            \label{eq:observer:sensormodels:camera:lambda}
            \lambda = -\frac{\left(\xi + R(q^{wb}) r_{\text{camera}/\mathcal{G}} \right) \cdot \hat{z}}{\left( R(q^{wP}) \frac{X^{PTAM}}{|X^{PTAM}|} \right) \cdot \hat{z}}
        \end{equation}


        By comparing the approximated distance with the distance measured
        in the PTAM coordinate system, we obtain a starting guess for the scaling factor, $s$;
        \begin{equation}
            s = \frac{|X^{PTAM}|}{|\lambda|}.
        \end{equation}

        Together, these parameters form an initial estimate of the transformation
        connecting the PTAM reference frame. This estimate could be inserted into
        the global observer filter - introducing eight new states containing
        $q^{Pw}$, $s$ and $\varnothing_{\text{cam}}$ - although little benefit
        of this has been observed in simulated testing.
        Because the PTAM coordinate system is defined fixed in the global NEDEF coordinate system,
        the transform parameters are unchanged in the observer's time update.

    \subsubsection{Continuous Refinement}
        \label{sssec:observer:sensormodels:camera:refinement}
        The measurement update is made separate from the update of the inertial sensors, using
        the measurement equations in \eqref{eq:observer:measurement:camera}, expanding the
        equation derived in Eqs. \eqref{eq:observer:sensormodels:camera:transformation} and \eqref{eq:observer:sensormodels:camera:qpw}.

        \begin{subequations}
            \label{eq:observer:measurement:camera}
            \begin{align}
                \hat{X}^{\text{PTAM}} &= \mathcal{J}^{Pw} (\xi + R(q^{wb})r_{\text{camera}/\mathcal{G}}) + e_{\text{PTAM,X}} \\
                \hat{q}^{\text{PTAM,c}} &= q^{Pw} q^{wb} q^{bc} + e_{\text{PTAM,q}}
            \end{align}
        \end{subequations}

    \subsubsection{Teleportation}
        \label{sssec:observer:sensormodels:camera:teleportation}
        The PTAM tracking may sometimes exhibit a ``teleporting'' behaviour.
        That is, although tracking is overall stable, the origin may
        sometimes be misassociated and placed at a new position as the
        tracking gets lost.
        To detect this, the measurements may be monitored for sudden changes in position.
        If a teleportation is detected, a reinitialization would be needed,
        either performing a new initial estimation, or utilizing the previous state
        to recognize the new pose of the origin.
        The teleporting behaviour is detected using simple thresholding,
        as in Eq.~\eqref{eq:observer:sensormodels:camera:teleport},
        although no action is currently implemented to recover.
        In Eq.~\eqref{eq:observer:sensormodels:camera:teleport}, the
        estimated motion is scaled by the factor $s$ from the tranformation
        from PTAM coordinates, and thresholded by a configurable parameter $\epsilon$.
        \begin{equation}
            \label{eq:observer:sensormodels:camera:teleport}
            \left|X^{\text{PTAM}}_{t} - X^{\text{PTAM}}_{t-1}\right| \cdot s > \epsilon
        \end{equation}


