\chapter{State Estimation}
\label{cha:observer}
    A central part of automatic control is to know the state of the device you are
    controlling. The system studied in this thesis - the LinkQuad - is in constant
    motion, so determining the up-to-date position if of vital importance to the performance
    of the control.
    This chapter deals with the estimation of the states relevant for positioning
    and controlling the LinkQuad.
    Filter theory and notation is established in Section \ref{sec:observer:filtering}.

    In this thesis, an Unscented Kalman Filter (UKF) is used, which extends the
    linear Kalman filter theory to a non-linear model in an appealing black-box way.
    The theory of the UKF is treated in Section \ref{sec:observer:ukf}.

    The motion model of the system is derived and discussed in Section \ref{sec:observer:motionmodel}.

    The motions of the system is also captured by the on-board sensors.
    A measurement $z$ is related to the motion model by the sensor model $h$;
    \begin{equation}
        z(t) = h(x(t),u(t),t)
    \end{equation}
    The models for the sensors used on the LinkQuad are discussed in
    Section \ref{sec:observer:sensormodels}.

    \section{The Filtering Problem}
\label{sec:observer:filtering}
    The problem of estimating the state of a system - in this case
    it's position, orientation, velocity etc. - is in the Kalman filter
    framework expressed as the problem of finding the state estimate
    $\hat{x}$ that in a well defined best way (e.g. with Minimum Mean Square Error, MMSE)
    describes the behaviour of the system.

    The evolution of a system plant is traditionally described by a set of differential equations
    that link the change in the variables to the current state and known inputs, $u$.
    The system is also assumed to be subject to an additive white Gaussian noise $v(t)$ with
    known covariance $Q$.
    This introduces an uncertainty associated with the system, which accounts
    for imperfections in the model compared to the physical real-worl-system.
    \begin{equation}
        \label{eq:observer:xdot}
        \dot{x}(t) = f_{c}(x(t),u(t),t) + v_{c}(t)
    \end{equation}
    With numeric or analytical solutions, we can obtain the discrete form of
    \eqref{eq:observer:xdot}, where only the sampling times are considered.
    The control signal, $u(t)$, is for instance assumed to be constant in the time interval,
    and we obtain obtain the next predicted state directly, yielding the prediction of $\hat{x}$
    at the time $t$ \textit{given} the information at time $t-1$.
    \footnote{Note that we have not yet performed any measurements that provide information about the state at time t.}
    This motivates the notation used in this thesis - $\hat{x}_{t|t-1}$.
    \begin{equation}
        \label{eq:observer:xnext}
        x_{t|t-1} = f(x_{t-1|t-1},u_{t},t) + v(t)
    \end{equation}

    In the ideal case, a simulation of a prediction $\hat{x}$ would
    with the prediction model in \eqref{eq:observer:xnext} fully describe
    the evolution of the system.
    To be able to provide a good estimate in the realistic case, however,
    we must also feed back measurements given from sensors measuring
    properties of the system.

    These measurements, $y_{t}$, are fed back and fused with the
    prediction using the \textit{innovation}, $\nu$.
    \begin{equation}
        \nu_{t} = y_{t} - \hat{y}_{t}
    \end{equation}
    That is, the difference between the measured value and what would be
    expected in the ideal (simulated) case.
    To account for disturbances affecting the sensors, the measurements
    are associated with an additive white Gaussian noise $w(t)$, with
    known covariance $R$.
    \begin{equation}
        \hat{y}_{t} = h(\hat{x}_{t}, u_{t}, t) + w(t)
    \end{equation}

    The innovation is then fused with the prediction to yield a new
    estimation \citep{gustafsson2010statistical} of $x$ given the
    information available as of the time $t$.
    \begin{equation}
        \hat{x}_{t|t} = \hat{x}_{t|t-1} + K_{t}\nu_{t}
    \end{equation}

    The choice of $K_{t}$ is a balancing between of trusting the model,
    or trusting the measurement. In the Kalman filter framework,
    this balancing is made by tracking and weighing the uncertainties
    introduced by the prediction and the measurement noise.

    Because of the assumptions on the noise and the linear property
    of the innovation feedback, the Gaussian property of the noise is preserved
    in the filtering process. The system states can thus ideally
    be considered drawn from a normal distribution.
    \begin{equation}
        x \sim \mathcal{N}\left(\hat{x}, P_{xx}\right)
    \end{equation}

    Conditioned on the state and measurements before time $k$, the
    covariance of the sample distribution is defined as
    \begin{equation}
        P_{xx}(t|k) = E \left[ \left\lbrace x(t) - \hat{x}_{t|k} \right\rbrace
                               \left\lbrace x(t) - \hat{x}_{t|k} \right\rbrace^{T}
                               | \mathcal{Z}^{j} \right] .
    \end{equation}
    As new measurements are taken, the covariance of the state evolves
    \citep{Julier95anewapproach} with the state estimate as
    \begin{equation}
        P_{xx}(t|t) = P_{xx}(t|t-1) - K_{t}P_{\nu\nu}(t|t-1)K_{t}^{T}
    \end{equation}
    \begin{equation}
        \label{eq:observer:filtering:pnunu}
        P_{\nu\nu}(t|t-1) = P_{yy}(t|t-1) + R(t) .
    \end{equation}
    $K$ is derived in e.g. \citep{gustafsson2010statistical} as
    \begin{equation}
        \label{eq:observer:filtering:pxy}
        K_{t} = P_{xy}(t|t-1)P_{\nu\nu}^{-1}(t|t-1),
    \end{equation}
    given the cross-correlation of the predicted state and output, $P_{xy}$.

    There are several approaches to how to propagate the covariance
    through the prediction-model to acquire $P_{xx}(t|t-1)$ with retained Gaussian properties.
    As the linear case is simple;
    \begin{equation}
        P_{xx}(t|t-1) = AP(t|t)A^{T} + Q_{t};
    \end{equation}
    a novel approach is to linearize the system for $A$.
    This linearization, however, fails to capture the finer details
    of highly non-linear systems and may furthermore be tedious to
    calculate, analytically or otherwise. A different approach is
    discussed in Section \ref{sec:observer:ukf}.

    \section{The Unscented Kalman Filter}
\label{sec:observer:ukf}
    The basic version of the Unscented Kalman Filter was proposed in \citep{Julier95anewapproach}
    based on the following intuition \citep{Julier95anewapproach}:
    \begin{quote}\textit{
        With a fixed number of parameters it should be easier to approximate a Gaussian
        distribution than it is to approximate an arbitrary nonlinear function.
        }
    \end{quote}
    The approach is thus to propagate the uncertainty of the system
    through the nonlinear system and fit the results as a Gaussian distribution.
    The propagation is made by simulating the system in the prediction
    model for carefully chosen offsets from the current state called
    \textit{sigma points}, each associated with a weight of importance.
    The selection scheme for these points can vary (and yield other
    types of filters), but a common choice
    is the \textit{Scaled Unscented Transform} (SUT) \citep{vandermerwe:upf}.
    The SUT uses a minimal set of sigma points needed to describe the
    first two moments of the propagated distribution - two for each
    of the $n$ dimensions of the state vector and one for the mean.
    The sigma points and their associated weights are chosen according to
    Eqs.~\ref{eq:observer:ukf:sigmapoints}-\ref{eq:observer:ukf:lambda}.
    \begin{align}\nonumber
        \mathbf{\mathcal{X}_{0}} &= \hat{x} & \\\nonumber
        \mathbf{\mathcal{X}_{i}} &= \hat{x} + \left( \sqrt{(n + \lambda) P_{xx}} \right)_{i}
            & i = 1,\cdots,n \\
        \mathbf{\mathcal{X}_{i}} &= \hat{x} - \left( \sqrt{(n + \lambda) P_{xx}} \right)_{i}
            & i = n+1,\cdots,2n
    \end{align}
    \begin{align}\nonumber
        \begin{array}{lr}
        W_{0}^{m} = \frac{\lambda}{n + \lambda} \qquad&
            W_{0}^{c} = \frac{\lambda}{n + \lambda} + (1-\alpha^{2} + \beta){}
        \end{array}\\
        \begin{array}{c}
            W^{m}_{i} =  W^{c}_{i} = \frac{1}{2(n + \lambda)} \qquad i = 1,\cdots,2n{}
        \end{array}
        \label{eq:observer:ukf:sigmapoints}
    \end{align}
    \begin{equation}
        \lambda = \alpha^{2}(n + \kappa) - n
    \end{equation}
    The three introduced parameters, $\alpha$, $\beta$ and $\kappa$
    are summarized and described in Table~\ref{tbl:observer:ukf::parameters}.
    The term $\left( \sqrt{(n + \lambda) P_{xx}} \right)_{i}$ is used to
    denote the $i$'th column of the matrix square root $\sqrt{(n + \lambda) P_{xx}}$.


    \begin{table}
        \begin{tabularx}{\tablewidth}{c|p{2cm}X}
            \textbf{Variable} & \textbf{Example value} & \textbf{Description} \\ \hline
            $\alpha$ & $0 \leq \alpha \leq 1$ (e.g. $0.01$) & Scales the size of the sigma point distribution.
                                                A small $\alpha$ can be used to avoid large non-local nonlinearities. \\
            $\beta$  & $2$ &  As discussed in \citep{Julier02thescaled}, $\beta$ affects the weighting of the center point,
                            which will directly influence the magnitude of errors introduced by the fourth and higher
                            order moments. In the strictly Gaussian case, $\beta = 2$ can be shown to be optimal. \\
            $\kappa$ & $0$ &  $\kappa$ is the number of times that the center point is included in the set of sigma points,
                            which will add weight to the center point and scale the distribution of sigma points. \\
        \end{tabularx}
        \label{tbl:observer:ukf::parameters}
        \caption{Description of the parameters used in the SUT.}
    \end{table}

    When the sigma points $\mathbf{\mathcal{X}_{i}}$ have been calculated,
    they are propagated through the nonlinear prediction function and
    the resulting mean and covariance can be calculated.
    \begin{align}
        \mathbf{\mathcal{X}^{+}_{i}} &= f(\mathbf{\mathcal{X}_{i}}, u, t) \qquad i = 0,\cdots,2n \\
        \hat{x} &= \sum_{i=0}^{2n}W^{m}_{i}\mathbf{\mathcal{X}^{+}_{i}} \\
        P_{xx} &= \sum_{i=0}^{2n}W^{c}_{i}
            \left\lbrace \mathbf{\mathcal{X}^{+}_{i}} - \hat{y} \right\rbrace
            \left\lbrace \mathbf{\mathcal{X}^{+}_{i}} - \hat{y} \right\rbrace^{T}
    \end{align}

    For the measurement update, similar results are obtained, and Eqs.
    \eqref{eq:observer:ukf:pyy}-\eqref{eq:observer:ukf:pxy} can be connected to
    Eqs. \eqref{eq:observer:filtering:pnunu}-\eqref{eq:observer:filtering:kalmanK2}
    in the general filter formulation.
    \begin{align}
        \mathbf{\mathcal{Y}_{i}} &= h(\mathbf{\mathcal{X}_{i}}, u, t) \qquad i = 0,\cdots,2n \\
        \hat{y} &= \sum_{i=0}^{2n}W^{m}_{i}\mathbf{\mathcal{Y}_{i}} \\
        P_{yy} &= \sum_{i=0}^{2n}W^{c}_{i}
            \left\lbrace \mathbf{\mathcal{Y}_{i}} - \hat{y} \right\rbrace
            \left\lbrace \mathbf{\mathcal{Y}_{i}} - \hat{y} \right\rbrace^{T} \label{eq:observer:ukf:pyy} \\
        P_{xy} &= \sum_{i=0}^{2n}W^{c}_{i}
            \left\lbrace \mathbf{\mathcal{X}_{i}} - \hat{x} \right\rbrace
            \left\lbrace \mathbf{\mathcal{Y}_{i}} - \hat{y} \right\rbrace^{T} \label{eq:observer:ukf:pxy}
    \end{align}

    As can be seen in the equations of this section, the UKF handles the
    propagation of the probability densities through the model without
    the need for explicit calculation of the Jacobians or Hessians for the system.
    The filtering is based solely on function evaluations of small offsets from the
    expected mean state\footnote{It is not, however, merely a central difference linearization of the functions, as stressed in \citep{Julier95anewapproach}.},
    be it for the measurement functions, discussed in
    Section~\ref{sec:observer:sensormodels}, or the time update
    prediction function - the motion model.

