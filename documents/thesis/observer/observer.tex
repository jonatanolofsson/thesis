\chapter{State Estimation}
\label{cha:observer}
    A central part of automatic control is to know the state of the device you are
    controlling. The system studied in this thesis - the LinkQuad - is in constant
    motion, so determining the up-to-date position if of vital importance to the performance
    of the control.
    This chapter deals with the estimation of the states relevant for positioning
    and controlling the LinkQuad.
    Filter theory and notation is established in Section \ref{sec:observer:filtering}.

    In this thesis, an Unscented Kalman Filter (UKF) is used, which extends the
    linear Kalman filter theory to a non-linear model in an appealing black-box way.
    The theory of the UKF is treated in Section \ref{sec:observer:ukf}.

    The motion model of the system is derived and discussed in Section \ref{sec:observer:motionmodel}.

    The motions of the system is also captured by the on-board sensors.
    A measurement $z$ is related to the motion model by the sensor model $h$;
    \begin{equation}
        z(t) = h(x(t),u(t),t)
    \end{equation}
    The models for the sensors used on the LinkQuad are discussed in
    Section \ref{sec:observer:sensormodels}.

    \section{Filtering}
\label{sec:results:filtering}
The filter implementation was evaluated with recorded data.
Due to model stability issues, as discussed in Chapter \ref{cha:discussion},
the EKF-filter was selected for evaluation.

\subsection{Positioning}
    While the altitude positioning, shown in Figure~\ref{fig:results:filtering:position},
    exhibit disturbances correlated with pressure sensor noise,
    the positioning generally exhibit very good performance. The position state
    is observed in the state-estimation more or less directly by the camera
    and the stability of the camera positioning is thus of course reflected here.

    In the plot of the Z-axial positioning in Figure~\ref{fig:results:filtering:position},
    the estimates from the pre-existing complementary filter -
    moved to share the initial conditions of the Kalman filter - was added for
    comparison. The complementary filter exhibits the problems associated
    with the pressure sensor, while the camera-based positioning is stable
    throughout the test.

    \begin{subfigures}{Positioning in the X- and Y-direction mostly well corresponds to ground truth, thanks to the camera positioning.
    In the Z-axis, the estimation of the pre-existing complementary filter is added for comparison, its starting point adjusted to produce comparable plots.}{fig:results:filtering:position}
        \sfig{\splotwidth}{X}
        \sfig{\splotwidth}{Y}
        \sfig{\splotwidth}{Z}
    \end{subfigures}

\subsection{Velocities}
    The velocities, being closely coupled with the camera observed position,
    also exhibit good performance (Figure~\ref{fig:results:filtering:velocities}).
    There are shortcomings to the estimation's horizontal precision, although
    this could probably be significantly improved with further filter tuning
    and the availability of control signals.
    \begin{subfigures}{Velocity estimates are generally adequate, but with more tuning, the results are likely to improve.}{fig:results:filtering:velocities}
        \sfig{\splotwidth}{velX}
        \sfig{\splotwidth}{velY}
        \sfig{\splotwidth}{velZ}
    \end{subfigures}

\subsection{Orientation, Rotational Velocity and Gyroscope Bias}
    Along with the position, the orientation is estimated from the camera,
    yielding notable precision, as seen in Figure~\ref{fig:results:filtering:qwb}.

    The bias of the gyroscopes is removed during the initialization process.
    Since the time-frame of the tests were far less than the time expected to
    detect a change in the bias, these should thus be estimated to zero,
    as verified in Figure~\ref{fig:results:filtering:drift}.

    As noted in Section~\ref{ssec:results:modelling:gyroscopes}, the
    filtering of the rotational velocities of the quadrotor body,
    exhibited with their associated covariance in Figure~\ref{fig:results:filtering:omega},
    correlates very well to the measurements.
    In Figure~\ref{fig:results:filtering:headings}, the Kalman filter performance is compared
    to the previously existing complementary filter. It can be seen that
    the prediction update of the Kalman filter improves the phase of the filter,
    although they are of comparable performance towards the end of the dataset.

    \begin{subfigures}{The orientation of the quadrotor was estimated with good accuracy.}{fig:results:filtering:qwb}
        \sfig{\sfplotwidth}{qwb0}
        \sfig{\sfplotwidth}{qwbi}
        \sfig{\sfplotwidth}{qwbj}
        \sfig{\sfplotwidth}{qwbk}
    \end{subfigures}

    \begin{subfigures}{The predicted angular velocities corresponds very well to the gyro measurements.}{fig:results:filtering:omega}
        \sfig{\splotwidth}{wRoll}
        \sfig{\splotwidth}{wPitch}
        \sfig{\splotwidth}{wYaw}
    \end{subfigures}

    \begin{subfigures}{The gyroscopes' drift was removed prior to entering the filter, and does not change during the short recording of data.}{fig:results:filtering:drift}
        \sfig{\splotwidth}{driftRoll}
        \sfig{\splotwidth}{driftPitch}
        \sfig{\splotwidth}{driftYaw}
    \end{subfigures}
    \fig{\plotwidth}{headings}{A comparison with the tuned complementary filter show that both filters accurately describe the heading, and although the phase of the Kalman filter is better in the beginning, they are of comparable performance towards the end of the dataset.}{fig:results:filtering:headings}

\subsection{Wind Force}
    As the tests were performed inside, the filter was tuned to basically keep the
    wind constant. Thus, it is difficult to come to any conclusions regarding the
    wind impact on the model. Figure~\ref{fig:results:filtering:wind}
    shows them to be correctly estimated to zero in the collected dataset,
    although in the case with simulated data with wind, shown in Figure~\ref{fig:results:filtering:simwind},
    results are poor, most likely due to that the simulation of the sensors' data
    does not fully cover the non-continuous behavior that is expected in the
    event of touching the ground, for example proper clamping of accelerometer simulations.

    \begin{subfigures}{Wind estimates from recorded test-data.}{fig:results:filtering:wind}
        \sfig{\splotwidth}{windX}
        \sfig{\splotwidth}{windY}
        \sfig{\splotwidth}{windZ}
    \end{subfigures}

    \begin{subfigures}{Wind from simulated test-flight.}{fig:results:filtering:simwind}
        \sfig{\splotwidth}{simwindX}
        \sfig{\splotwidth}{simwindY}
        \sfig{\splotwidth}{simwindZ}
    \end{subfigures}

    \fig{\plotwidth}{windbump}{Even in simulation, the landing was detectable in the vertical wind estimate, although more work is required to properly model and detect the event.}{fig:results:filtering:windbump}

\subsection{Propeller Velocity}
    As the filter evaluation was performed without the use of the controller,
    the control signal is unavailable. Thus, Eq.~\ref{eq:observer:wri2} was used as
    motion model in the filter validation, effectively leaving the estimation of the propeller velocities
    to the measurement update. It is evident, in Figure~\ref{fig:results:filtering:wr}
    that the estimation is active, however it is impossible to validate properly with the available data.
    Ideally, the velocities of the propellers should be measured in flight.
    However, that data is currently unavailable in the development system used for evaluation.
    The estimated velocities are, notably, in a reasonable range,
    increasing the plausibility for correctness of Eq.~\ref{eq:observer:thrust}, which is
    otherwise hard to verify using the available data.

    \begin{subfigures}{Propeller angular rate estimates could not be properly verified with the data available, although do exhibit a reasonable value range given the model parameter settings.}{fig:results:filtering:wr}
        \sfig{\sfplotwidth}{wr1}
        \sfig{\sfplotwidth}{wr2}
        \sfig{\sfplotwidth}{wr3}
        \sfig{\sfplotwidth}{wr4}
    \end{subfigures}

    \section{The Unscented Kalman Filter}
\label{sec:observer:ukf}
    The basic version of the Unscented Kalman Filter was proposed in \citep{Julier95anewapproach}
    based on the following intuition \citep{Julier95anewapproach}
    \begin{quote}\textit{
        With a fixed number of parameters it should be easier to approximate a Gaussian
        distribution than it is to approximate an arbitrary nonlinear function.
        }
    \end{quote}
    The approach is thus to propagate the uncertainty of the system
    through the non-linear system and fit the results as a Gaussian distribution.
    The propagation is made by simulating the system in the prediction
    model for carefully chosen offsets from the current state called
    \textit{sigma points}, each associated with a weight of importance.
    The selection scheme for these points can vary (and yield other
    types of filters), but a common choice
    is the \textit{Scaled Unscented Transform} (SUT) \citep{vandermerwe:upf}.
    The SUT uses a minimal set of sigma points needed to describe the
    first two moments of the propagated distribution - two for each
    dimension ($n$) of the state vector and one for the mean.
    \begin{align}\nonumber
        \mathbf{\mathcal{X}_{0}} &= \hat{x} & \\\nonumber
        \mathbf{\mathcal{X}_{i}} &= \hat{x} + \left( \sqrt{(n + \lambda) P_{xx}} \right)_{i}
            & i = 1,\cdots,n \\
        \mathbf{\mathcal{X}_{i}} &= \hat{x} - \left( \sqrt{(n + \lambda) P_{xx}} \right)_{i}
            & i = n+1,\cdots,2n
    \end{align}
    \begin{align}\nonumber
        \begin{array}{lr}
        W_{0}^{m} = \frac{\lambda}{n + \lambda} \qquad&
            W_{0}^{c} = \frac{\lambda}{n + \lambda} + (1-\alpha^{2} + \beta){}
        \end{array}\\
        \begin{array}{c}
            W^{m}_{i} =  W^{c}_{i} = \frac{1}{2(n + \lambda)} \qquad i = 1,\cdots,2n{}
        \end{array}
    \end{align}
    \begin{equation}
        \lambda = \alpha^{2}(n + \kappa) - n
    \end{equation}
    The three parameters introduced here, $\alpha$, $\beta$ and $\kappa$
    are summarized in Table~\ref{tbl:observer:ukf::parameters}.
    The term $\left( \sqrt{(n + \lambda) P_{xx}} \right)_{i}$ is used to
    denote the $i$'th column of the matrix square root $\sqrt{(n + \lambda) P_{xx}}$.


    \begin{table}
        \begin{tabularx}{\tablewidth}{c|p{2cm}X}
            \textbf{Variable} & \textbf{Value} & \textbf{Description} \\ \hline
            $\alpha$ & $0 \leq \alpha \leq 1$ (e.g. $0.01$) & Scales the size of the sigma point distribution.
                                                A small $\alpha$ can be used to avoid large non-local non-linearities. \\
            $\beta$  & $2$ &  As discussed in \citep{Julier02thescaled}, $\beta$ affects the weighting of the center point,
                            which will directly influence the magnitude of errors introduced by the fourth and higher
                            order moments. In the strictly Gaussian case, $\beta = 2$ can be shown to be optimal. \\
            $\kappa$ & $0$ &  $\kappa$ is the number of times that the centerpoint is included in the set of sigma points,
                            which will add weight to the centerpoint and scale the distribution of sigma points. \\
        \end{tabularx}
        \label{tbl:observer:ukf::parameters}
        \caption{Description of the parameters used in the SUT.}
    \end{table}

    When the sigma points $\mathbf{\mathcal{X}_{i}}$ have been calculated,
    they are propagated through the non-linear prediction function and
    the resulting mean and covariance can be calculated.
    \begin{equation}
        \mathbf{\mathcal{X}^{+}_{i}} = f(\mathbf{\mathcal{X}_{i}}, u, t) \qquad i = 0,\cdots,2n
    \end{equation}
    \begin{equation}
        \hat{x} = \sum_{i=0}^{2n}W^{m}_{i}\mathbf{\mathcal{X}^{+}_{i}}
    \end{equation}
    \begin{equation}
        P_{xx} = \sum_{i=0}^{2n}W^{c}_{i}
            \left\lbrace \mathbf{\mathcal{X}^{+}_{i}} - \hat{y} \right\rbrace
            \left\lbrace \mathbf{\mathcal{X}^{+}_{i}} - \hat{y} \right\rbrace^{T}
    \end{equation}

    For the measurement update, similar results are obtained, and the equations
    \eqref{eq:observer:ukf:pyy}-\eqref{eq:observer:ukf:pxy} can be connected to
    equations \eqref{eq:observer:filtering:pnunu}-\eqref{eq:observer:filtering:kalmanK2}.
    \begin{equation}
        \mathbf{\mathcal{Y}_{i}} = h(\mathbf{\mathcal{X}_{i}}, u, t) \qquad i = 0,\cdots,2n
    \end{equation}
    \begin{equation}
        \hat{y} = \sum_{i=0}^{2n}W^{m}_{i}\mathbf{\mathcal{Y}_{i}}
    \end{equation}
    \begin{equation}
        \label{eq:observer:ukf:pyy}
        P_{yy} = \sum_{i=0}^{2n}W^{c}_{i}
            \left\lbrace \mathbf{\mathcal{Y}_{i}} - \hat{y} \right\rbrace
            \left\lbrace \mathbf{\mathcal{Y}_{i}} - \hat{y} \right\rbrace^{T}
    \end{equation}
    \begin{equation}
        \label{eq:observer:ukf:pxy}
        P_{xy} = \sum_{i=0}^{2n}W^{c}_{i}
            \left\lbrace \mathbf{\mathcal{X}_{i}} - \hat{x} \right\rbrace
            \left\lbrace \mathbf{\mathcal{Y}_{i}} - \hat{y} \right\rbrace^{T}
    \end{equation}

    As can be seen from the equations in this section, the UKF handles the
    propagation of the probability densities through the model without
    the need for explicit calculation of the Jacobians or Hessians for the system.
    The filtering is based solely on function evaluations of small offsets from the
    expected mean state\footnote{It is not, however, merely a central difference linearization of the functions, as \citep{Julier95anewapproach} notes.},
    be it for the measurement functions, discussed in
    Section~\ref{sec:observer:sensormodels}, or the time update
    prediction function - the motion model.

