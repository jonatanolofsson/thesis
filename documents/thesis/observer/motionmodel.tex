\section{Motion Model}
\label{sec:observer:motionmodel}
    As the system studied in the filtering problem progresses through time,
    the state estimate can be significantly improved if a prediction
    is made on what measurements can be expected, and evaluating the plausibility
    of each measurement after how well they correspond to the prediction.
    With assumed Gaussian white noise distributions, this evaluation
    can be done in the probabilistic Kalman framework as presented in
    Sections~\ref{sec:observer:filtering}-\ref{sec:observer:ukf}, where the
    probability estimate of the sensors' measurements are based on the motion
    model's prediction. In this section, a motion model is derived and evaluated.
    The model is presented in its continuous form, since the discretization
    is made numerically on-line.

    \subsection{Coordinate Frames}
        In the model of the quadrotor, there are several frames of reference.
        \begin{description}
            \item[North-East-Down (NED):]
            The NED-frame is fixed at the center of gravity of the quadrotor.
            The NED system's $\hat{z}$-axis is aligned with the gravitational axis and
            the $\hat{x}$-axis along the northern axis.
            The $\hat{y}$-axis is chosen to point east to form a right-hand system.

            \item[North-East-Down Earth Fixed (NEDEF):] This frame is fixed at
            a given origin in the earth (such as the take-off point) and is
            considered an inertial frame, but is in all
            other aspects equivalent to the NED frame.
            All states are expressed in this frame of reference unless explicitly stated otherwise.
            This frame is often referred to as the ``world'' coordinate system,
            or ``$w$'' for short in equations in disambiguation from the NED system.

            \item[Body-fixed (BF):] The body-fixed coordinate system is fixed in the
            quadrotor with $\hat{x}$-axis in the forward direction and the $z$-axis in the downward
            direction - as depicted in Figure~\ref{fig:observer:motionmodel:quadrotorcoordframes}.

            \item[Propeller fixed:] Each of the propellers are associated
            with their own frame of reference, $P_{i}$, which tracks the
            virtual tilting of the thrust vector due to flapping,
            discussed in Section~\ref{ssec:observer:thrust}.

            \item[Camera frame:] This is the frame which describes the
            location of the camera.

            \item[PTAM frame:] This is the frame of reference used by the PTAM video SLAM library.
            This frame is initially unknown, and its recovery is discussed in Section~\ref{ssec:observer:sensormodels:camera}.

            \item[IMU frame:] This is the body-fixed frame in which the IMU measurements are said
            to be done. The origin is thus fixed close to the inertial sensors.
        \end{description}
        The coordinate frames are visualized in Figures~\ref{fig:observer:motionmodel:worldcoordframes}-\ref{fig:observer:motionmodel:quadrotorcoordframes}.
        \fig{0.8}{worldcoordframes}{The NEDEF, body-fixed and the PTAM coordinate frames are of global importance.}{fig:observer:motionmodel:worldcoordframes}
        \fig{0.8}{quadrotorcoordframes}{Locally, on the quadrotor, there are several coordinate frames used in the thesis. Here, the IMU frame coincides with the body-fixed frame.}{fig:observer:motionmodel:quadrotorcoordframes}

        The conversion between the reference frames are characterized by a
        transformation including translation and a three-dimensional rotation.
        Both the origin of the body-centered reference frames
        - the quadrotor's position - and the rotation of the body-fixed
        system are stored as system states.

        The centers of each of the propeller fixed coordinate systems
        are parametrized on the height $h$ and distance $d$ from the center of
        gravity as follows
        \begin{align}
            D_{0} &= (d, 0, h)^{BF} \\
            D_{1} &= (0, -d, h)^{BF} \\
            D_{2} &= (-d, 0, h)^{BF} \\
            D_{3} &= (0, d, h)^{BF}
        \end{align}

        In the following sections, vectors and points in e.g. the NED
        coordinate systems are denoted $x^{NED}$.
        Rotation described by unit quaternions are denoted $R(q)$ for
        the quaternion $q$, corresponding to the matrix rotation
        \citep{kuipers2002quaternions}\footnote{\citep{kuipers2002quaternions} uses a negated convention of signs compared to what is used here.}
        given by
%~ FIXME
%~ From http://www.euclideanspace.com/maths/geometry/rotations/conversions/quaternionToMatrix/index.htm
%~ and the Eigen src/Geometry/Quaternion.h. Note the equivalence given the unit length constraint on the rotation quaternion
%~ 1 - 2*qy2 - 2*qz2    2*qx*qy - 2*qz*qw   2*qx*qz + 2*qy*qw
%~ 2*qx*qy + 2*qz*qw    1 - 2*qx2 - 2*qz2   2*qy*qz - 2*qx*qw
%~ 2*qx*qz - 2*qy*qw    2*qy*qz + 2*qx*qw   1 - 2*qx2 - 2*qy2
        \begin{equation}
            \left(
            \begin{array}{cccc}
                q_{1}^{2} + q_{i}^{2} - q_{j}^{2} - q_{k}^{2}   & 2q_{i}q_{j}-2q_{1}q_{k}                       & 2q_{i}q_{k} + 2q_{1}q_{j} \\
                2q_{i}q_{j} + 2q_{1}q_{k}                       & q_{1}^{2} - q_{i}^{2} + q_{j}^{2} - q_{k}^{2} & 2q_{j}q_{k} - 2q_{1}q_{i} \\
                2q_{i}q_{k} - 2q_{1}q_{j}                       & 2q_{j}q_{k} + 2q_{1}q_{i}                     & q_{1}^{2} - q_{i}^{2} - q_{j}^{2} + q_{k}^{2}
            \end{array}
            \right) .
        \end{equation}

        Rotation quaternions describing the rotation \textit{to frame $a$ from frame $b$}
        is commonly denoted $q^{ab}$, whereas the \textit{b} may be dropped if the rotation is global, i.e. relative the NEDEF system.
        \textit{a} and \textit{b} are, where unambiguous, replaced by the first character of the
        name of the reference frame.
        Full transformations between coordinate systems - including rotation, translation and scaling -
        are similarly denoted $\mathcal{J}^{ab}$.

        The equations in this chapter is given in their time-continuous form
        to simplify the description of the forces.
        The equations are discretized in the implementation using numerical Euler integration (Eq. \eqref{eq:observer:euler})
        or Runge-Kutta integration (Eq. \eqref{eq:observer:rungekutta}), using an integration step of $T$.
        \begin{equation}
        \label{eq:observer:euler}
            X_{t+1} = X_{t} + T \cdot f(X,t)
        \end{equation}
        \begin{subequations}
        \label{eq:observer:rungekutta}
            \begin{align}
                X_{t+1} &= X_{t} + \tfrac{1}{6} \left(k_1 + 2k_2 + 2k_3 + k_4 \right) \\
                k_1 &= Tf(X, t) \\
                k_2 &= Tf(X_{t} + \tfrac{1}{2} k_1, t + \tfrac{1}{2}T) \\
                k_3 &= Tf(X_{t} + \tfrac{1}{2} k_2, t + \tfrac{1}{2}T) \\
                k_4 &= Tf(X_{t} + k_3, t + T)
            \end{align}
        \end{subequations}

    \subsection{Kinematics}
        The motions of the quadrotor are described by the following relations \citep{Pounds_modellingand}: % more citations needed?
        \begin{subequations}
            \label{eq:observer:kinematicsc}
            \begin{equation}
                \label{eq:observer:position}
                \dot{\xi} = V
            \end{equation}
            \begin{equation}
                \label{eq:observer:quaternionsc}
                \left(\begin{array}{c}
                    \dot{q}^{wb}_{0} \\
                    \dot{q}^{wb}_{i} \\
                    \dot{q}^{wb}_{j} \\
                    \dot{q}^{wb}_{k}
                \end{array}\right) = -\frac{1}{2}\left(\begin{array}{cccc}
                0 & -\omega_{x} & -\omega_{y} & -\omega_{z} \\
                \omega_{x} & 0 & -\omega_{z} & \omega_{y} \\
                \omega_{y} & \omega_{z} & 0 & -\omega_{x} \\
                \omega_{z} & -\omega_{y} & \omega_{x} & 0
                \end{array}\right)\left(\begin{array}{c}
                q^{wb}_{0} \\
                q^{wb}_{i} \\
                q^{wb}_{j} \\
                q^{wb}_{k}
                \end{array}\right)
            \end{equation}
        \end{subequations}

        In practice, a normalization step also has to be added to account for
        the unit length constraint on rotation quaternions.


    \subsection{Dynamics}
        The motions of the quadrotor can be fully explained by the
        forces and moments acting on the vehicle. Using the rigid-body assumption,
        Euler's extension of Newton's laws of motion applied to the
        quadrotor's center of gravity, $\mathcal{G}$, yields Eqs.~\ref{eq:motionmodel:newton}.
        These equations describe the acceleration and angular acceleration of the vehicle
        as related to the forces acting on the vehicle.
        \begin{subequations}
        \label{eq:motionmodel:newton}
            \begin{align}
                \dot{V} &= a^{w}_{\mathcal{G}} = R(q^{wb})\frac{1}{m}\sum F \\
                \dot{\Omega} &= R(q^{wb})I_{\mathcal{G}}^{-1}\sum M_{\mathcal{G}}
            \end{align}
        \end{subequations}

        The main forces acting upon the quadrotor are the effects of three different components
        \begin{itemize}
            \item $\sum_{i=1}^{4}F_{ri}$ - Rotor thrust,
            \item $F_{g}$ - Gravity,
            \item $F_{\text{wind}}$ - Wind.
        \end{itemize}

        Of these, the gravity is trivially described with the
        gravitational acceleration and the total mass of the quadrotor as
        \begin{equation}
            F_{g} = mg\cdot \hat{z}^{NED}
        \end{equation}

        The following sections will describe the rotor thrust and wind forces respectively.
        Additionally, other minor forces and moments are discussed in
        Section~\ref{ssec:observer:additionalforces}.

        \subsubsection{Rotor thrust}
\label{ssec:observer:thrust}
    Each of the four propellers on the quadrotor induce a torque
    and a thrust vector on the system, proportional to the square of the
    propeller velocity. The rotational velocity of the propeller is directly
    influenced by the controller. It may thus be modelled as a first order system - using the time constant $\tau_{rotor}$ with
    the reference velocity as input, as in Eq. \eqref{eq:observer:wri}.
    For testing purposes, or where the control signal is not available, Eq. \eqref{eq:observer:wri2} may be used instead.
    \begin{equation}
        \label{eq:observer:wri}
        \dot{\omega_{ri}} = \frac{1}{\tau_{rotor}} \left( \omega_{ri} - r_{i} \right)
    \end{equation}
    \begin{equation}
        \label{eq:observer:wri2}
        \dot{\omega_{ri}} = 0
    \end{equation}

    Due to the differences in relative airspeed around
    the rotor blade tip as the blades move either along or against
    the wind-relative velocity, the lifting force on the blade will vary
    around a rotation lap.
    This unbalance in lifting force will cause the blades to lean and the
    direction of the thrust vector to vary with regards to the motions of the quadrotor.

    This phenomenon is called \textit{flapping}, and is discussed
    in e.g. \citep{Pounds_modellingand}. The flapping of the rotors
    and the centrifugal force acting upon the rotating blades
    will result in that the tilted blade trajectories will
    form a cone with the plane to which the rotor axis is normal.
    These motions of the propellers add dynamics to the
    description of the quadrotors motion which must be considered in a deeper analysis.

    It is desirable, for the purpose of this thesis and
    considering computational load, to find a closed-form
    solution to the flapping equations.
    %~ This implies several approximations and restrictions which will be discussed in Section~\ref{sec:discussion:flapping}.
    The resulting flapping angles and their impact on
    the thrust vectors can be described as in equations
    \eqref{eq:observer:thrust}-\eqref{eq:observer:flapping}
    \citep{Pounds_modellingand,prouty1995helicopter,leishman2002principles}.

    The momenta induced by the propeller rotation and thrust
    are described in equations \eqref{eq:observer:torque}-\eqref{eq:observer:thrustmomentum}.
    All equations in this section are given in the body-fixed coordinate system.

    \begin{subequations}
        \begin{equation}
            \label{eq:observer:thrust}
            F_{ri} = C_{T} \rho A_{r} R^{2} \omega_{ri}^{2}\left(
                \begin{array}{c}
                    -\sin{a_{1_{s}i}} \\
                    -\cos{a_{1_{s}i}}\sin{b_{1_{s}i}} \\
                    -\cos{a_{1_{s}i}}\cos{b_{1_{s}i}}
                \end{array}\right)
        \end{equation}

        \begin{equation}
            \label{eq:observer:torque}
            M_{Qi} = -C_{Q} \rho A R^{3} \omega_{ri}|\omega_{ri}|e_{3}^{\text{NED}}
        \end{equation}

        \begin{equation}
            \label{eq:observer:thrustmomentum}
            M_{ri} = F_{ri} \times D_{ri}
        \end{equation}
    \end{subequations}
    The equations for the flapping angles $\left(a_{1_{s}i}, b_{1_{s}i}\right)$ are
    derived in \citep{Pounds_modellingand,prouty1995helicopter,leishman2002principles},
    but are in \eqref{eq:observer:flapping} extended to include the velocity relative to the wind.
    $V_{ri(n)}$ denotes the n'th element of the vector $V_{ri}$.
    \begin{subequations}
        \label{eq:observer:flapping}
        \begin{equation}
            V_{\text{rel}} = V - V_{\text{wind}}
        \end{equation}
        \begin{equation}
            V_{ri} = V_{\text{rel}} + \Omega \times D_{ri} % Velocity relative to the wind //Jonatan
        \end{equation}
        \begin{equation}
            \mu_{ri} = \frac{||V_{ri(1,2)}||}{\omega_{i}R}
        \end{equation}
        \begin{equation}
            \psi_{ri} = \arctan{\frac{V_{ri(2)}}{V_{ri(1)}}}
        \end{equation}
        \begin{equation}
            \label{eq:observer:flapping:ab}
            \begin{array}{rr}\left(
                \begin{array}{c}
                    a_{1_{s}}i \\
                    b_{1_{s}}i
                \end{array} \right)
                = \left(
                \begin{array}{cc}
                    \cos{\psi_{ri}} & -\sin{\psi_{ri}} \\
                    \sin{\psi_{ri}} & \cos{\psi_{ri}}
                \end{array}
                \right) & \left(
                    \begin{array}{c}
                        \frac{1}{1 - \frac{\mu_{ri}^{2}}{2}}\mu_{ri}\left( 4 \theta_{twist} - 2\lambda_{i}\right) \\
                        \frac{1}{1 + \frac{\mu_{ri}^{2}}{2}}\frac{4}{3}\left( \frac{C_{T}}{\sigma}\frac{2}{3}\frac{\mu_{ri}\gamma}{a} + \mu_{ri}\right)
                    \end{array}
                \right) \\
                & +
                \left(
                    \begin{array}{c}
                        \frac{-\frac{16}{\gamma}\left(\frac{\omega_{\theta}}{\omega_{ri}}\right) + \left(\frac{\omega_{\psi}}{\omega_{ri}}\right)}{1 - \frac{\mu_{ri}^{2}}{2}} \\
                        \frac{-\frac{16}{\gamma}\left(\frac{\omega_{\psi}}{\omega_{ri}}\right) + \left(\frac{\omega_{\theta}}{\omega_{ri}}\right)}{1 + \frac{\mu_{ri}^{2}}{2}}
                    \end{array}
                \right)
            \end{array}
        \end{equation}

        %~ \textbf{Note: In the equations \eqref{eq:observer:flapping:ab}, taken from \citep{Pounds_modellingand}, I assume that by ''$a$'', they mean lift curve slope, and by ''$a_{0}$'', they mean the linearization point of a and NOT the coning angle of Prouty pp.468, or the mean coning of Prouty pp.153}

        %~ \citep{Pounds_modellingand}
        %~ \citep{prouty1995helicopter} pp. 165
        %~ \textbf{Not quite finished here.. FIXME: $\theta_{0}$ to table!}
        \begin{equation}
            \lambda_{i} = \mu\alpha_{si} + \frac{v_{1i}}{\omega_{i} R}
        \end{equation}
        \begin{equation}
            v_{1i} = \sqrt{
                -\frac{V_{rel}^{2}}{2} + \sqrt{
                    \left( \frac{V_{rel}^{2}}{2} \right)^{2}
                    + \left( \frac{mg}{2 \rho A_{r}} \right)^{2}
                }
            }
        \end{equation}
        \begin{equation}
            C_{T} = \frac{\sigma a}{4}\left\lbrace
                  \left( \frac{2}{3} + \mu_{ri}^{2} \right) \theta_{0}
                - \left( \frac{1}{2} + \frac{\mu^{2}}{2} \right) \theta_{\text{twist}}
                + \lambda
            \right\rbrace
        \end{equation}
        \begin{equation} % angle between shaft plane and path (rel. to wind) pp.160
            \alpha_{si} = \frac{\pi}{2} - \arccos{ -\frac{V_{rel} \cdot e_{z}}{||V_{rel}||} }
        \end{equation}
        % This is from bouabdallah07design. Same as above _but_ with negative signs... Why?
        %~ \begin{equation}
            %~ C_{T} = \sigma a \left[
                %~ \left(\frac{1}{6} + \frac{1}{4}\mu^{2}\right)\theta_{0}
                %~ - (1 + \mu^{2})\frac{\theta_{twist}}{8}
                %~ - \frac{1}{4}\lambda \right]
        %~ \end{equation}
        \begin{equation}
            C_{Q} = \sigma a \left[
                \frac{1}{8a}\left( 1 + \mu_{ri}^{2} \right) \bar{C_{d}}
                + \lambda\left(
                    \frac{1}{6}\theta_{0}
                    - \frac{1}{8}\theta_{\text{twist}}
                    + \frac{1}{4}\lambda
                    \right)
                \right]
        \end{equation}
    \end{subequations}

    \begin{table}
        \begin{tabularx}{\tablewidth}{|c|c|X|c|}\hline
            \textbf{Symbol} & \textbf{Expression} & \textbf{Description}  & \textbf{Unit} \\\hline
            $a$ & $\frac{\operatorname{d}\!C_{L}}{\operatorname{d}\!\alpha} \approx 2\pi$ & Slope of the lift curve. & $\frac{1}{\text{rad}}$ \\\hline
            $\alpha_{si}$ & - & Propeller angle of attack. & $\text{rad}$ \\\hline
            $A_{r}$ & - & Rotor disk area.   & $\text{m}^{2}$\\\hline
            $c$ & - & Blade chord - the (mean) length between the trailing and leading edge of the propeller.   & $\text{m}$ \\\hline
            $C_{L}$ & - & Coefficient of lift. & $1$ \\\hline
            $C_{T}$ & * & Coefficient of thrust. This is primarily the scaling factor for how the thrust is related to the square of $\omega_{i}$, as per Eq.~\ref{eq:observer:thrust}.  & $1$\\\hline
            $C_{T0}$ & - & Linearization point for thrust coefficient.  & $1$\\\hline
            $C_{Q}$ & * & Torque coefficient. This constant primarily is the scaling factor relating the square of $\omega_{i}$ to the torque from each rotor. & $1$\\\hline
            $\gamma$ & $\frac{\rho a c R^{4}}{I_{b}}$ & $\gamma$ is the Lock Number \citep{leishman2002principles}, described as the ratio between the aerodynamic forces and the inertal forces of the blade.   & $1$ \\ \hline
            $I_{b}$ & - & Rotational inertia of the blade  & $\text{kgm}^{2}$\\\hline
            $\lambda_{i}$ & * & $\lambda_{i}$ denotes the air inflow to the propeller. & $1$ \\\hline
            $R$ & - & Rotor radius.   & $\text{m}$ \\\hline
            $\rho$ & - & Air density.   & $\frac{\text{kg}}{\text{m}^{3}}$ \\\hline
            $\sigma$ & $\frac{\text{blade area}}{\text{disk area}}$ & Disk solidity. & $1$ \\\hline
            $\theta_{0}$ & - & The angle of the propeller at its base, relative to the horizontal disk plane. & $\text{rad}$ \\\hline
            $\theta_{\text{twist}}$ & - & The angle with which the propeller is twisted.  & $\text{rad}$ \\\hline
            $\omega_{\phi},\omega_{\theta},\omega_{\psi}$ & - & The rotational, body-fixed, velocity of the quadrotor. & $\frac{\text{rad}}{\text{s}}$ \\\hline
            $\omega_{ri}$ & - & The rotational velocity of propeller $i$. & $\frac{\text{rad}}{\text{s}}$ \\\hline
            $\mu_{ri}$ & - & The normalized, air-relative, blade tip velocity. & $1$ \\\hline
            %$q$ & - & Pitch rate \\\hline % Can this be replaced by \omega_{\theta}? Just did... =) Body-fixed, so should be good
            %$p$ & - & Roll rate \\\hline % Can this be replaced by \omega_{\psi}? Just did... =) Body-fixed, so should be good
        \end{tabularx}
        \label{tbl:observer:flapping:symbols}
        \caption{Table of symbols used in the flapping equations}
    \end{table}

        \subsubsection{Wind}
    For describing the wind's impact on the quadrotor motion,
    a simple wind model is applied where the wind is modeled with
    a static velocity that imposes forces and moments on the quadrotor.
    The wind velocity vector is estimated by the observer and may thus still vary
    in its estimation through the measurement update.
    The wind velocities in the filter are given in the NEDEF reference frame.

    The wind drag force is calculated using equation \eqref{eq:observer:wind:dragforce},
    whereas the moments are given by equations \eqref{eq:observer:wind:moments}.
    In this thesis, the moments acting on the quadrotor body (as opposed to the rotors)
    are neglected or described by moments imposed by the wind acting on the rotors.

    \begin{subequations}
    \label{eq:observer:wind:dragforce}
        \begin{align}
            F_{\text{wind}} &= F_{wind,body} + \sum_{i=0}^{3} F_{wind,ri} \\
%
            F_{\text{wind,body}} &= -\frac{1}{2} C_{D} \rho A V_{\text{rel}} ||V_{\text{rel}}|| \\
%
            F^{BF}_{\text{wind},ri} &= -\frac{1}{2} \rho C_{D,r} \sigma A_{r} (V_{ri} \cdot e_{P_{ri}3}^{BF}) ||V_{ri} \cdot e_{P_{ri}3}^{BF}|| e_{P_{ri}3}^{BF}
        \end{align}
    \end{subequations}

    \begin{subequations}
    \label{eq:observer:wind:moments}
        \begin{align}
            M_{\text{wind}} = M_{\text{wind,body}} + \sum_{i=0}^{3}M_{\text{wind},ri} \\
%
            M_{\text{wind,body}} \approx 0 \\ % Could perhaps be better approximated by constant*\sum_{i=0}^{3}M_{wind,ri}, for small constant, modeling the wind's effect on the four arms
%
            M_{\text{wind},ri} = D_{ri}^{BF} \times F^{BF}_{\text{wind},ri}
        \end{align}
    \end{subequations}

    The wind model applied in this thesis is a decaying model that tends
    towards zero if no measurements tell otherwise.
    This decaying model is presented in Eq. \eqref{eq:observer:wind:wind} ($\epsilon$ being a small number).
    \begin{equation}
        \label{eq:observer:wind:wind}
        \dot{V}_{\text{wind}} = -\epsilon \cdot V_{\text{wind}}
    \end{equation}


    \begin{table}
        \begin{tabularx}{\tablewidth}{|c|c|X|}\hline
            \textbf{Symbol} & \textbf{Expression} & \textbf{Description} \\\hline
            $A$ & - & 3x3 matrix describing the area of the quadrotor, excluding the rotors. \\\hline
            $C_{D}$ & - & 3x3 matrix describing the drag coefficients of the quadrotor. \\\hline
            $C_{Dr}$ & - & Propeller's coefficient of drag. \\\hline
        \end{tabularx}
        \label{tbl:observer:wind:symbols}
        \caption{Table of symbols used in the wind equations}
    \end{table}

        \subsubsection{Additional Forces and Moments}
\label{ssec:observer:additionalforces}
    Several additional forces act on the quadrotor to give its dynamics in flight.
    Some of these are summarized briefly in this section, and are discussed further in \citep{bouabdallah07full}.
    Unless where explicitly noted, annotation is similar to Section~\ref{ssec:observer:thrust}.
    \paragraph{Hub Force}
        \begin{equation}
            C_{H} = \sigma a \left[
                \frac{1}{4a}\mu\bar{C_{d}}
                + \frac{1}{4}\lambda\mu\left( \theta_{0} + \frac{\theta_{twist}}{2} \right)
                \right]
        \end{equation}
        \begin{equation}
            F_{\text{hub},i} = -C_{H} \rho A R^{2} \omega_{i}^{2}\hat{x}
        \end{equation}
        \begin{equation}
            M_{\text{hub},i} = D_{i} \times F_{hub,i}
        \end{equation}

    \paragraph{Rolling Moment}
        \begin{equation}
            C_{\text{RM}} = - \sigma a \mu \left[
                \frac{1}{6}\theta_{0}
                + \frac{1}{8}\theta_{twist}
                - \frac{1}{8}\lambda_{i}
                \right]
        \end{equation}
        \begin{equation}
            M_{\text{RM},i} = C_{\text{RM}} \rho A R^{3} \omega_{i}^{2}
        \end{equation}

    \paragraph{Ground Effect}
        As the vehicle gets close to ground, the wind foils of the propellers
        provide a cushion of air under the vehicle, giving extra lift.
        \begin{equation}
            T_{\text{IGE}} = \frac{1}{1-\frac{R^{2}}{16z^{2}}} T
        \end{equation}

    \paragraph{Gyro Effects and Counter-Torque}
        $I_{\text{rotor}}$ is the propeller inertia.
        \begin{equation}\left(
            \begin{array}{c}
                \dot{\omega}_{\theta}\dot{\omega}_{\psi}(I_{yy}-I_{zz}) + I_{\text{rotor}}\dot{\omega}_{\theta}\sum_{i=0}^{4}\omega_{ri} \\
                \dot{\omega}_{\theta}\dot{\omega}_{\psi}(I_{zz}-I_{xx}) + I_{\text{rotor}}\dot{\omega}_{\theta}\sum_{i=0}^{4}\omega_{ri} \\
                \dot{\omega}_{\theta}\dot{\omega}_{\phi}(I_{xx}-I_{yy}) + I_{\text{rotor}}\sum_{i=0}^{4}\dot{\omega}_{ri}
            \end{array}\right)
        \end{equation}

