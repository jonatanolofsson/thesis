\section{Sensor Models}
\label{sec:observer:sensormodels}
    This section relates the estimated state of the quadrotor to the expected sensor measurements, $\hat{y}$.

    In UAV state estimation, it is common to include a GPS sensor, providing world-fixed measurements,
    to prevent drift in the filtering process. In this thesis, the GPS is replaced with a camera,
    which is discussed in detail in Section~\ref{ssec:observer:sensormodels:camera}, following
    a description of the accelerometers, gyroscopes, magnetometers and the pressure sensor.

    In the measurement equations in this section, a zero-mean Gaussian term $e$ with known covariance is added
    to account for measurement noise. The Gaussian assumption may in some
    cases be severely inappropriate, but the Kalman filter framework requires its use.

    \subsection{Accelerometers}
        The accelerometers, as the name suggests, provides measurements of the
        accelerations of the sensor. In general, this does not directly correspond
        to the accelerations of the mathematical center of gravity used
        as center of the measured vehicle. This motivates a correction
        for angular acceleration and velocity with regards to the sensor's
        relative position from the center of gravity, $r_{\text{acc}/\mathcal{G}}$.
        \begin{equation}
            \hat{y}_{\text{acc}} = a_{\mathcal{G}} + \dot{\omega} \times r_{\text{acc}/\mathcal{G}} + \omega \times \left( \omega \times r_{\text{acc}/\mathcal{G}} \right) + e_{\text{acc}}
        \end{equation}

    \subsection{Gyroscopes}
        Gyroscopes, or rate gyroscopes specifically, measure the angular velocity
        of the sensor. Unlike acceleration, the angular rate is theoretically
        invariant of the relative position of the sensor and the center of gravity.
        However, gyroscope measurements are associated with a bias which may
        change over time. This bias term may be introduced as a state variable in the observer,
        modeled constant as in Eq.~\eqref{eq:filter:sensormodels:accelerometertime},
        leaving its adjustment to the observer's measurement update.
        \begin{equation}
            y_{\text{gyro}} = \omega + b + e_{\text{gyro}}
        \end{equation}
        \begin{equation}
            \label{eq:filter:sensormodels:accelerometertime}
            \dot{b} = 0 + e_{\text{bias}}
        \end{equation}

    \subsection{Magnetometers}
        Capable of sensing magnetic fields, the magnetometers can be used to
        sense the direction of the Earth's magnetic field and, from knowing
        the field at the current location, estimate the orientation of a vehicle.
        \begin{equation}
            y = R(q) m^{e} + e_{m}
        \end{equation}
        The Earth's magnetic field can initialized at startup, or approximated using the World Magnetic Model\cite{wmm2010},
        which for Linköping, Sweden, is given in Eq. \eqref{eq:observer:sensormodels:magneticfield}.
        \begin{equation}
            \label{eq:observer:sensormodels:magneticfield}
            m^{e} = \left[\left(\begin{array}{ccc}
                15.7 & 1.11 & 48.4
            \end{array}\right)^{T}\right]^{NEDEF} \mu T
        \end{equation}

        Magnetometer measurements are, however, very sensitive to disturbances, and in indoor
        flight, measurements are often useless due to electrical wiring, lighting etc.
        Thus, the magnetometers were not used for the state estimation in this thesis.

    \subsection{Pressure Sensor}
        The barometric pressure, $p$, can be related to altitude using Eq. \eqref{eq:sensormodels:pressure} \cite{physicshandbook}.
        The pressure sensor, as shown in the validation, is inherently
        noisy and especially so in an indoor environment where air conditioning
        causes disturbances in the relatively small - and thus pressure sensitive -
        environments that are available indoors.
        The pressure sensor is also affected by propeller turbulence.
        \begin{equation}
            \label{eq:sensormodels:pressure}
            p = p_{0} \left( 1 - \frac{L \cdot h}{T_{0}} \right)^{\frac{g \cdot M}{R\cdot L}} + e_{p}
        \end{equation}
        \begin{table}
            \begin{tabularx}{\tablewidth}{|c|X|c|c|}\hline
                \textbf{Parameter} & \textbf{Description} & \textbf{Value} & \textbf{Unit} \\\hline
                $L$       & Temperature lapse rate.            & $0.0065$ & $\frac{\text{K}}{\text{m}}$ \\\hline
                $M$       & Molar mass of dry air.             & $0.0289644$ & $\frac{\text{kg}}{\text{mol}}$ \\\hline
                $p_{0}$   & Atmospheric pressure at sea level. & $101325$ & $\text{Pa}$ \\\hline
                $R$       & Universal gas constant.            & $8.31447$ & $\frac{\text{J}}{\text{mol} \cdot \text{K}}$ \\\hline
                $T_{0}$   & Standard temperature at sea level. & $288.15$ & $\text{K}$ \\\hline
            \end{tabularx}
            \label{tos:pressuresensor}
            \caption{Table of Symbols used in the pressure equation, \eqref{eq:sensormodels:pressure}}.
        \end{table}

    \subsection{Camera}
\label{ssec:observer:sensormodels:camera}
    To estimate the position of the camera using the captured images,
    the PTAM-library is used.
    Because the main application of the PTAM library is reprojection of
    augmented reality into the image, consistency between a metric world-fixed
    coordinate frame (such as the NEDEF-system used on the LinkQuad), and the
    internally used coordinate system is not of importance - and thus not implemented in the PTAM positioning.

    The measurements from the camera consists of the transform from the PTAM ``world''-coordinates
    to the camera lens, in terms of
    \begin{itemize}
        \item translation, $X^{\text{PTAM}}$,
        \item and orientation, $q^{PTAM,c}$.
    \end{itemize}
    However, since the quite arbtirary\citep{klein07parallel} coordinate system
    of PTAM is neither of the same scale nor aligned with the quadrotor coordinate system,
    we need to estimate the affine transformation between the two in order to get useful results.

    The transformation is characterized by
    \begin{itemize}
        \item a translation T to the origin, $\varnothing_{\text{PTAM}}$,
        \item a rotation R by the quaternion $q^{Pw}$,
        \item and a scaling S by a factor $s$.
    \end{itemize}
    %~ \begin{equation}
        %~ x_{\text{world}} = \underbrace{R(q^{wb})}_{quadrotor orientation} * \underbrace{T(\varnothing_{\text{cam}}) R(q^{bc}) S(s)}_{quadrotor to PTAM transform} x_{\text{cam}}
    %~ \end{equation}
    These are collected to a single transformation in Eq.~\ref{eq:observer:sensormodels:camera:transformation},
    forming the full transformation from the global NEDEF system to the
    PTAM coordinate frame.
    \begin{equation}
        \label{eq:observer:sensormodels:camera:transformation}
        x^{\text{PTAM}} = \underbrace{S(s) R(q^{Pw}) T(-\varnothing_{\text{PTAM}})}_{\triangleq \mathcal{J}^{Pw}, \text{transformation from camera to PTAM}}
         x^{\text{NEDEF}}
    \end{equation}

    E.g. \citep{hayashi2010} studies this problem in the offline case,
    whereas the method used in this thesis extends the idea to the on-line
    case where no ground truth is available. This is performed by using the
    first measurement to construct an initial guess which then is filtered
    through time using the observer.

    While PTAM exhibit very stable positioning, it has a tendency to move
    its origin due to association errors. To provide stable position
    measurements, we need to detect these movements and adjust the
    camera transformation accordingly. Initialization and tracking is dealt with in
    Section~\ref{sssec:observer:sensormodels:camera:initialization} and \ref{sssec:observer:sensormodels:camera:refinement}
    respectively, whereas the teleportation problem is discussed in
    Section~\ref{sssec:observer:sensormodels:camera:teleportation}.

    \subsubsection{Initialization}
        \label{sssec:observer:sensormodels:camera:initialization}
        When the first camera measurement arrives, there is a need to construct a
        first guess of the transform. Since the PTAM initialization places
        the origin at what it considers the ground level, the most informed
        guess we can do without any information about the environment is
        to assume that this is a horizontal plane at zero height.

        First, however, the orientation of the PTAM coordinate system is calculated
        from the estimated quadrotor orientation and the measurement in the
        PTAM coordinate frame;
        \begin{equation}
            \label{eq:observer:sensormodels:camera:qpw}
            q^{Pw} = q^{PTAM,c} q^{cb} q^{bw}.
        \end{equation}

        $q^{bc}$, the inverse of $q^{cb}$ of Equation~\ref{eq:observer:sensormodels:camera:qpw},
        describes the rotation from camera coordinates to body-fixed coordinates, taking
        into account the differing definitions between the PTAM library
        camera coordinate system and that used in this thesis.
        With known camera pitch and yaw - $\Theta_{c}$ and $\Psi_{c}$ respectively -
        this  corresponds to four consecutive rotations, given in
        Eq.~\eqref{eq:observer:sensormodels:qbc} as rotations around gives axes.

        %~ Quaternion<double> qbc(
            %~ AngleAxis<double>(config["camera"]["tilt"].as<double>(), Vector3d::UnitY())
            %~ * AngleAxis<double>(config["camera"]["yaw"].as<double>(), Vector3d::UnitZ())
            %~ * AngleAxis<double>(M_PI_2, Vector3d::UnitZ())
            %~ * AngleAxis<double>(M_PI_2, Vector3d::UnitX()));
        \begin{equation}
        \label{eq:observer:sensormodels:qbc}
            q^{bc} = \text{rot}(\Theta_{c}, \hat{y}) * \text{rot}(\Psi_{c}, \hat{z}) * \text{rot}(\pi/2, \hat{z}) * \text{rot}(\pi/2, \hat{x})
        \end{equation}

        To determine the distance to this plane according to the current estimation,
        Equation \eqref{eq:observer:sensormodels:camera:origin} is solved for $\lambda$
        in accordance with Eq.~\ref{eq:observer:sensormodels:camera:lambda}.
        \begin{equation}
            \label{eq:observer:sensormodels:camera:origin}
            \left\lbrace
            \begin{array}{ll}
                \varnothing_{\text{PTAM}} &= \xi + R(q^{wb}) r_{\text{camera}/\mathcal{G}} + \lambda R(q^{wP}) \frac{X^{PTAM}}{|X^{PTAM}|} \\
                \varnothing_{\text{PTAM}} \cdot \hat{z} &= 0
            \end{array}\right.
        \end{equation}

        \begin{equation}
            \label{eq:observer:sensormodels:camera:lambda}
            \lambda = -\frac{\left(\xi + R(q^{wb}) r_{\text{camera}/\mathcal{G}} \right) \cdot \hat{z}}{\left( R(q^{wP}) \frac{X^{PTAM}}{|X^{PTAM}|} \right) \cdot \hat{z}}
        \end{equation}


        By comparing the approximated distance with the distance measured
        in the PTAM coordinate system, we obtain a starting guess for the scaling factor, $s$;
        \begin{equation}
            s = \frac{|X^{PTAM}|}{|\lambda|}.
        \end{equation}

        Together, these parameters form an initial estimate of the transformation
        connecting the PTAM reference frame. This estimate could be inserted into
        the global observer filter - introducing eight new states containing
        $q^{Pw}$, $s$ and $\varnothing_{\text{cam}}$ - although little benefit
        of this has been observed in simulated testing.
        Because the PTAM coordinate system is defined fixed in the global NEDEF coordinate system,
        the transform parameters are unchanged in the observer's time update.

    \subsubsection{Continuous Refinement}
        \label{sssec:observer:sensormodels:camera:refinement}
        The measurement update is made separate from the update of the inertial sensors, using
        the measurement equations in \eqref{eq:observer:measurement:camera}, expanding the
        equation derived in Eqs. \eqref{eq:observer:sensormodels:camera:transformation} and \eqref{eq:observer:sensormodels:camera:qpw}.

        \begin{subequations}
            \label{eq:observer:measurement:camera}
            \begin{align}
                \hat{X}^{\text{PTAM}} &= \mathcal{J}^{Pw} (\xi + R(q^{wb})r_{\text{camera}/\mathcal{G}}) + e_{\text{PTAM,X}} \\
                \hat{q}^{\text{PTAM,c}} &= q^{Pw} q^{wb} q^{bc} + e_{\text{PTAM,q}}
            \end{align}
        \end{subequations}

    \subsubsection{Teleportation}
        \label{sssec:observer:sensormodels:camera:teleportation}
        The PTAM tracking may sometimes exhibit a ``teleporting'' behaviour.
        That is, although tracking is overall stable, the origin may
        sometimes be misassociated and placed at a new position as the
        tracking gets lost.
        To detect this, the measurements may be monitored for sudden changes in position.
        If a teleportation is detected, a reinitialization would be needed,
        either performing a new initial estimation, or utilizing the previous state
        to recognize the new pose of the origin.
        The teleporting behaviour is detected using simple thresholding,
        as in Eq.~\eqref{eq:observer:sensormodels:camera:teleport},
        although no action is currently implemented to recover.
        In Eq.~\eqref{eq:observer:sensormodels:camera:teleport}, the
        estimated motion is scaled by the factor $s$ from the tranformation
        from PTAM coordinates, and thresholded by a configurable parameter $\epsilon$.
        \begin{equation}
            \label{eq:observer:sensormodels:camera:teleport}
            \left|X^{\text{PTAM}}_{t} - X^{\text{PTAM}}_{t-1}\right| \cdot s > \epsilon
        \end{equation}

