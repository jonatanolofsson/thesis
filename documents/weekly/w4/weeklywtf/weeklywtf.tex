\section{Summary}
This is the first report that I intend to try to write each friday to report
of the progress in the work of my master thesis.

This week's progress have been somewhat dampened by some problems regarding
the move to the gumstix. Due to this, the milestones that were due today
will have to be moved and hopefully completed in the middle of next week.

Until the internal communication has been properly set up on a LinkBoard,
I regard the work to be in a ''setup'' phase of sorts, in which it is
difficult to diverge from the main line of work - since everything is more
or less dependent on getting the fundamental framework to work.

The turning in of an outline of the report structure has been deliberately
moved to next week, when I will meet with Thomas to discuss a few details
regarding this.

Apart from these relatively minor setbacks this week, the thesis work
has started out well.

\section{Problem of the week}
There are two main reasons of this weeks delay in progress.
The first was that I together with my tutors decided that it would benefit
the LinkQuad as a project - and thereby my thesis - to develop a new C++-library for the
serial port communication. The (yet untested) library is very promising and
will contribute as expected, but was still an unplanned effort that I thought was
more or less done in the previous implentation.

The second, less appealing delay was that the requirement of running LAPACK/BLAS
on the gumstix, which turned out to be previously untested, or at least undocumented,
on the internet and the gumstix user mailing list. Failing to compile BLAS natively,
I sat out on a detour to try to cross-compile it, until finally succeeding using
the most basic (and inefficient) implementations of BLAS.

Having failed to set up a working cross-compilation environment, this
time could of course in retrospect have been better spent, but I gave it an
effort. Successfully, it would have provided a useful tool later, but
i feel I can't spend more time on it now.

\section{Successes of the week}
Apart from creating a templated serial port communication library worthy of attention,
I have finalized the work on implementing an UKF filter and an LQ-controller as well
as connecting those in the ROS-alike framework I have created for the internal communication.
Figure~\ref{fig:observer} contains a simulated control to bring a one-dimensional
position to the origin, controlling the velocity and measuring the velocity as
a normally distributed random variable.

\fig{observer}{UKF-observed, LQ-controlled, 2-dimensional linear model. The velocity is measured as random noise $v\sim\mathcal{N}(0, 1)$}{fig:observer}

The system developed and described above is, after this weeks effort, now compiling and running
on the gumstix. The model is also easily extended with more sensors and more advanced models.
This is indeed promising, and is the base for the continued thesis work.
