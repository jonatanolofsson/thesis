\section{Summary}
Over the past few weeks, I have been terribly busy trying to get the
helicopter to fly in the simulated environment.
The environment is setup as similar to the LinkQuad as possible, including
using virtual serial ports for the communication with the ground station interface,
and simulating real, noisy, sensor inputs. The observer still has a few issues, but then
it should soon be ready for some real data.

As for the future time schedule; After I get the observer fully working,
I need to see to the report and get it up to date. I also would like to prepare
the control system for real flight - first as a passive observer (perhaps with recorded data)
but then also interfaced with the control MCU.

\section{Problems}
Due to the unexpected difficulties of getting the flight simulation working,
I have not had time to work with the report in the desired extent.
Having gotten it to work however, I now have results to write about, as well
as verification of the theory used, making it more meaningful to write about.

It may be that the computational load of the estimation and control will be
a problem on the LinkBoard. I have not been able to give the computation time
a fair estimate, but it is notably heavier than what was first expected.
I'm working on the problem, but the modelling may need simplifications in the
future. In the worst case, this problem means that the modelling work done
so far will be made superfluous. In that case, it will still have been an
interesting excercise in quadrotor modelling and will as such definitely have its place
in the thesis.

\section{Successes}
The progress made during the last weeks has not been at all insignificant.
The SDRE LQ controller have been successfully implemented and tested with simulated data,
even with 3d control commands sent as serial data over a virtual serial port.

Also, the PTAM library is now working with the camera that will be used on board.
The library is compiled for the LinkBoard, and the software that will run on
the secondary gumstix is more or less finished, save for tuning the camera settings.
