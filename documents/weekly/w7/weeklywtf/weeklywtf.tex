\section{Summary}
During last week, I went deep into the theoretical modelling work of the
LinkQuad. I managed to make a non-linear model of the quadrotor which
I am now in progress of implementing in code, where it will be verified
and tested against the previously developed linear model.

During this work, there has been progress in the report, mainly in the
observer chapter which, among other things, deal with the modelling for the
sake of state estimation.

I have also begun some discussions regarding the control structure
which will be used in the thesis.

\section{Problem of the week}
The main troubles this week was the difficulty of
wrapping my head around the equations and coordinate transformations.
However, with the help of Dr. David Törnqvist, ISY, I managed to sort out
some of the theory and the choice of state variables.
Gianpaolo Conte has also been helpful in providing literature and discussions
on the modelling of propeller flapping.

I had wished to include plots of the model for this week's report, but
since it's still beeing implemented, it has still to be verified.

\section{Successes of the week}
With only minor difficulties, I got the PTAM library to compile and run on my laptop.
I now only wait for the camera to be able to start to try to get it running
on the gumstix.

The observer of the report needs to be slightly polished after I changed
a few choices of coordinate systems in the implementation, then I should
be able to send it for review.
