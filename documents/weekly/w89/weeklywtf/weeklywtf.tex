\section{Summary}
The last two weeks since the last ''weekly'' report, have seen a very interesting
development of the thesis work. While debugging the physical model, I decided
that a visualization of the simulation was not only desirable, but could also
serve as an invaluable tool in the later work of the thesis, providing visual
feedback on the actions of the control system.
I thus made a quick 3D visualization tool in OpenGL, which will do for a great demonstration
as soon the modelling is fully debugged.

The visualization tool is made as a stand-alone program intended to be running
on a laptop. The communication with the control system is made via a serial port,
implying that the control system could just as well run on the gumstix as locally on the laptop.

While implementing the communication for the visualization, I also added
control input from a 3D mouse, which is sent to the control system on the serial port.

\section{Problem of the week}
There are still problems with the physical model of the system, but I now
have new tools for debugging. I will probably need input on the report's
modelling sections while working with this, but preliminary results are
going in the right direction.

I still don't have access to the camera with which we will do the visual feedback.
This should be available very soon however, and I have still been able to
work with preparatory work for this.

\section{Successes of the week}
A list of some of the recent accomplishments;
\begin{itemize}
    \item A new visualization tool has been developed,
    \item Simulation model is beeing implemented (currently debugged),
    \item Serial communication allows for control input,
    \item All libraries that PTAM use have been compiled (not tested) for the gumstix,
    \item Thesis draft sent in for review.
\end{itemize}
\fig{visualization}{Sample from the visulization interface, running against the real control system. The bunny is flying, obviously.}{fig:visualization}
