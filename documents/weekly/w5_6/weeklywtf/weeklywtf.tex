\section{Summary}
Week 5-6 have been unfortunately clamped by some private commitments and
travels. Therefore I have chosen to combine the reports for these weeks
to this single report.
Despite the lack of available time, I have nonetheless begun to put together the introductory
parts of the thesis report. I have been in contact with both Thomas and
also Gianpaolo Conte (who unfortunately is currently in Italy),
and have with their help begun to look at developing a non-linear dynamical model
for the observer, emanating from and congruent with the existing linear model.

The practical work initially progressed fast, then to face some problems
with the serial communication lingering from the previous week.
This is now working however, and I can now sample, send, receive and plot
data from the real sensors.

A state machine engine has successfully been developed and run on the gumstix
computer. It is hierarchichal and dynamical by structure, and provide a simple
interface for the programmer. It will provide an extremely lightweight and
synoptical logic module for the LinkQuad.

\section{Problem of the week}
The progress have been somewhat slower than intended on getting started
with the modelling, so that have had to fall behind somewhat.
Also, there was a minor misunderstanding in my new implementation of the
serial communication that took a while to figure out.
This is now what has the highest priority, and might require some discussions
in the start of next week.

The intention was to interface my code with the existing simulator this week.
With Gianpaolo in Italy, however, that is unsuitable, so focus have been
put to getting started with the writing of the report instead.
It might be suitable to postpone the ROS/simulator integration until the
system is closer to a working state, to see what work is really nescessary there.

\section{Successes of the week}
The sensor readings seem to be working well now. The new serial implementation
used for this provides a cleaner interface when programming, which I personally value.

The state machine is also working fine as it seems, and should scale
well with the task I'm focusing on in my thesis.

\fig{acceleration}{Sampled and scaled raw accelerometer data}{fig:acceleration}
\fig{gyro}{Sampled and scaled gyro data. One of the gyros is defect on the development board.}{fig:gyro}
